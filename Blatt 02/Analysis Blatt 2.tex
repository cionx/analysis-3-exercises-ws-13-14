\documentclass[a4paper,10pt]{article}
%\documentclass[a4paper,10pt]{scrartcl}

\usepackage{xltxtra}
\usepackage[ngerman]{babel}
\usepackage{amsmath}
\usepackage{amssymb}
\usepackage{amsthm}
\usepackage{mathtools}
\usepackage{nicefrac}
\usepackage{enumerate}
\usepackage{leftidx}

\makeatletter
\g@addto@macro\th@definition{\thm@headpunct{:}}
\makeatother
\theoremstyle{definition}
\newtheorem*{lem}{Lemma}
\newtheorem*{beh}{Behauptung}
\newtheorem*{bem}{Bemerkung}
\newtheorem*{ia}{Induktionsanfang}
\newtheorem*{is}{Induktionsschritt}

\renewcommand{\thesection}{Aufgabe \arabic{section}.}
\renewcommand{\thesubsection}{\alph{subsection})}
\renewcommand{\thesubsubsection}{(\roman{subsubsection})}

\newcommand{\N}{\mathbb{N}}
\newcommand{\Z}{\mathbb{Z}}
\newcommand{\Q}{\mathbb{Q}}
\newcommand{\R}{\mathbb{R}}
\newcommand{\C}{\mathbb{C}}
\newcommand{\A}{\mathcal{A}}
\newcommand{\La}{\mathcal{L}}
\newcommand{\dx}{\,\text{d}x}
\newcommand{\dy}{\,\text{d}y}
\newcommand{\dt}{\,\text{d}t}
\newcommand{\du}{\,\text{d}u}
\newcommand{\mc}[1]{\mathcal{#1}}
\newcommand{\Img}{\operatorname{Im}}
\newcommand{\Real}{\operatorname{Re}}
\newcommand{\Imag}{\operatorname{Im}}
\newcommand{\sgn}{\operatorname{sgn}}
\newcommand{\limes}[2]{\lim_{#1 \rightarrow #2}}
\newcommand{\limessup}[1]{\limsup_{#1 \rightarrow \infty}}
\newcommand{\limesinf}[1]{\liminf_{#1 \rightarrow \infty}}
\newcommand{\vect}[1]{\begin{pmatrix}#1\end{pmatrix}}
\newcommand{\partd}[2]{\frac{\partial #1}{\partial #2}}
\newcommand{\op}[1]{\left\|#1\right\|_{\text{op}}}

\makeatletter
\renewcommand*\env@matrix[1][*\c@MaxMatrixCols c]{%
  \hskip -\arraycolsep
  \let\@ifnextchar\new@ifnextchar
  \array{#1}}
\makeatother

\setromanfont[Mapping=tex-text]{Linux Libertine O}
% \setsansfont[Mapping=tex-text]{DejaVu Sans}
% \setmonofont[Mapping=tex-text]{DejaVu Sans Mono}
\parindent 0pt

\title{\sc Analysis III \\ \Large 2. Aufgabenblatt}
\author{Jendrik Stelzner}
\date{\today}

\begin{document}
\maketitle





\section{(Monotone Klassen und $\sigma$-Algebren)}


\subsection{}
Angenommen, $\A$ ist eine Algebra. Nach der Definition einer Algebra ist damit $X \in \A$ und für alle $A \in \A$ auch $A^c \in \A$; es muss also nur noch die $\sigma$-Additivität gezeigt werden.

Sei hierfür $(A_n)_{n \in \N}$ eine Folge auf $\A$. Für alle $n \in N$ sei $B_n := \bigcup_{k=0}^n A_k$. Es ist offenbar $\bigcup_{n \in \N} A_n = \bigcup_{n \in \N} B_n$. Da $(B_n)_{n \in \N}$ eine wachsende Folge auf $\A$ ist, und $\A$ eine monotone Klasse, gilt
\[
 \bigcup_{n \in \N} A_n = \bigcup_{n \in \N} B_n \in \A.
\]
Dies zeigt die $\sigma$-Additivität.

Ist andererseits $\A$ eine $\sigma$-Algebra, so ist $\A$ auch eine Algebra, da jede $\sigma$-Algebra auf $X$ eine Algebra auf $X$ ist (bekannt aus der Vorlesung).


\subsection{}
Wie aus der Vorlesung bekannt ist jede $\sigma$-Algebra eine monotone Klasse. Damit ist $\sigma(\A)$ eine monotone Klasse, die $\A$ enthält. Da $m(\A)$ die kleinste monotone Klasse ist, die $\A$ enthält, ist daher $m(\A) \subseteq \sigma(\A)$.

Um zu zeigen, dass auch $m(\A) \supseteq \sigma(\A)$, genügt es zu zeigen, dass $m(\A)$ eine $\sigma$-Algebra ist: Da $\sigma(\A)$ die kleinste $\sigma$-Algebra ist, die $\A$ enthält, ist dann $m(\A) \supseteq \sigma(\A)$. Da $m(\A)$ eine monotone Klasse ist, genügt es nach \textbf{Aufgabenteil a)} zu zeigen, dass $m(\A)$ eine Algebra ist.

Es ist $X \in \A \subseteq m(\A)$, da $\A$ eine Algebra ist. Es muss noch die Abgeschlossenheit von $\A$ unter Komplementbildung und endlichen Vereinigungen gezeigt werden.

Es sei $\mc{K} := \{M \in \mc{P}(X) : M^c \in m(\A)\}$. $\mc{K}$ ist eine monotone Klasse: Ist $(K_n)_{n \in \N}$ ein wachsende Folge in $\mc{K}$, so ist $(K_n^c)_{n \in \N}$ eine fallende Folge in $m(\A)$, und somit
\[
 \left( \bigcup_{n \in \N} K_n \right)^c
 = \bigcap_{n \in \N} K_n^c \in m(\A),
\]
da $m(\A)$ eine monotone Klasse ist, also $\bigcup_{n \in \N} K_n \in \mc{K}$.
Ist $(L_n)_{n \in \N}$ eine fallende Folge in $\mc{K}$, so ist $(L_n^c)_{n \in \N}$ eine wachsende Folge in $m(\A)$, und somit
\[
 \left( \bigcap_{n \in \N} L_n \right)^c = \bigcup_{n \in \N} L_n^c \in m(\A),
\]
da $m(\A)$ eine monotone Klasse ist, also $\bigcap_{n \in \N} L_n \in \mc{K}$. Dies zeigt, dass $\mc{K}$ eine monotone Klasse ist.

Da $\A$ eine Algebra ist, ist $A^c \in \A \subseteq m(\A)$ für alle $A \in \A$, also $\A \subseteq \mc{K}$. Da $\mc{K}$ eine monotone Klasse ist, die $\A$ enthält, ist $m(\A) \subseteq \mc{K}$. Also ist $A^c \in m(\A)$ für alle $A \in m(\A)$. Dies zeigt die Abgeschlossenheit von $\mc{A}$ unter Komplementbildung.

Für $D \in m(\A)$ sei $\mc{V}_D := \{M \in P(X) : D \cup M \in m(\A)\}$. Auch $\mc{V}_D$ ist eine monotone Klasse: Ist $(V_n)_{n \in \N}$ eine wachsende Folge auf $\mc{V}_D$, so ist $(V_n \cup D)_{n \in \N}$ ein wachsende Folge auf $m(\A)$, und somit
\[
 \left( \bigcup_{n \in \N} V_n \right) \cup D
 = \bigcup_{n \in \N} (V_n \cup D)
 \in m(\A),
\]
da $m(\A)$ eine monotone Klasse ist, und daher $\bigcup_{n \in \N} V_n \in \mc{V}_D$.
Ist $(W_n)_{n \in \N}$ eine fallende Folge auf $\mc{V}_D$, so $(W_n \cup D)$ eine fallende Folge auf $m(\A)$, und somit
\[
 \left( \bigcap_{n \in \N} W_n \right) \cup D
 = \bigcap_{n \in \N} (W_n \cup D)
 \in m(\A),
\]
da $m(\A)$ eine monotone Klasse ist, und daher $\bigcap_{n \in \N} W_n \in \mc{V}_D$. Dies zeigt, dass $\mc{V}_D$ eine monotone Klasse ist.

Sei $A \in \A$. Für $B \in \A$ ist $A \cup B \in \A \subseteq m(\A)$, da $\A$ eine Algebra ist, und daher $B \in \mc{V}_A$, wegen der Beliebigkeit von $B$ also $\A \subseteq \mc{V}_A$. Da $m(\A)$ die kleinste monotone Klasse ist, die $\A$ enthält, ist daher $m(\A) \subseteq \mc{V}_A$. Also ist $A \cup B \in m(\A)$ für alle $B \in m(\A)$ und $A \in \A$. Dies bedeutet auch, dass $\A \subseteq \mc{V}_B$ für alle $B \in m(\A)$. Da $m(\A)$ die kleinste monotone Klasse ist, die $\A$ enthält, ist daher $m(\A) \subseteq \mc{V}_B$ für alle $B \in m(\A)$. Das bedeutet gerade, dass $A \cup B \in m(\A)$ für alle $A, B \in m(\A)$. Dies zeigt die Abgeschlossenheit bezüglich endlicher Schnitte.





\section{(Ein Zugang zur Borelschen $\sigma$-Algebra)}
\emph{Korrektur: Aufgabenteil a) ist viel zu umständlich und Aufgabenteil b) ist falsch.}

\subsection{}
Sei $U \subseteq \R^n$ offen beliebig aber fest. Es gilt zu zeigen, dass $U \in F_\sigma$, d.h. dass es eine abzählbar unendliche Familie $(K_n)_{n \in \N}$, mit $K_n \subseteq \R^n$ abgeschlossen für alle $n \in \N$, gibt, so dass $U = \bigcup_{n \in \N} K_n$.

Sei $x \in U$. Da $U$ offen ist, gibt es ein $\varepsilon > 0$, so dass $B_\varepsilon(x) \subseteq U$. Es sei $r(x) \in \Q$ mit $0 < r(x) < \frac{\varepsilon}{3}$ und $q(x) \in \Q^n$ mit $q(x) \in B_{r(x)}(x)$ (solche $r(x)$ und $q(x)$ existieren, dass $\Q$ dicht in $\R$, bzw. $\Q^n$ dicht in $\R^n$ liegt). Aufgrund der Symmetrie einer Metrik ist nun $x \in \overline{B_{r(x)}(q(x))}$. Auch ist $\overline{B_{r(x)}(q(x))} \subseteq U$, denn für alle $y \in \overline{B_{r(x)}(q(x))}$ ist nach der Dreiecksungleichung
\[
 |y - x| \leq |y - q(x)| + |q(x) - x| \leq 2 r(x) < \frac{2}{3}\varepsilon,
\]
also $y \in B_\varepsilon(x) \subseteq U$.

Da nun
\[
 \{ \overline{B_{r(x)}(q(x))} : x \in U \}
 \subseteq \{ \overline{B_r(q)} : r \in \Q, r > 0, q \in \Q^n\}
\]
abzählbar ist, gibt es eine Folge $(x_n)_{n \in \N}$ auf $U$ so dass
\[
 V := \bigcup_{n \in \N} \overline{B_{r(x_n)}(q(x_n))} = \bigcup_{x \in U} \overline{B_{r(x)}(q(x))}.
\]

Es gilt nun, dass $U = V$: Für alle $x \in U$ gilt
\[
 x \in \overline{B_{r(x)}(q(x))} \subseteq \bigcup_{x \in U} \overline{B_{r(x)}(q(x))} = V
\]
und für alle $x \in V = \bigcup_{n \in \N} B_{r(x_n)}(q(x_n))$ gibt es ein $n \in \N$ mit
\[
 x \in \overline{B_{r(x_n)}(q(x_n))} \subseteq U.
\]
Das zeigt, dass $U = \bigcup_{n \in \N} \overline{B_{r(x_n)}(q(x_n))} \in F_{\sigma}$.

Sei $K \subseteq \R^n$ abgeschlossen. Da $K$ abgeschlossen ist, ist $K^c$ offen. Wie eben gezeigt gibt es daher eine Folge abgeschlossener Mengen $(K_n)_{n \in \N}$ mit $K^c = \bigcup_{n \in \N} K_n$. Insbesondere ist $K_n^c$ für alle $n \in \N$ offen, und damit
\[
 K
 = (K^c)^c
 = \left( \bigcup_{n \in \N} K_n \right)^c
 = \bigcap_{n \in \N} K_n^c
 \in G_\delta.
\]


\subsection{}

\subsubsection*{$\sigma(\mc{G}_n) = \delta(\mc{G}_n)$}
Da bekanntermaßen für alle $U, V \in \mc{G}_n$ auch $U \cap V \in \mc{G}_n$, ist, wie aus der Vorlesung bekannt, $\sigma(\mc{G}_n) = \delta(\mc{G}_n)$.

\subsubsection*{$\sigma(\mc{G}_n) \supseteq m(\mc{G}_n)$}
Da $\mc{G}_n \subseteq \sigma(\mc{G}_n)$ und $\sigma(\mc{G}_n)$ als $\sigma$-Algebra eine monotone Klasse ist, und $m(\mc{G}_n)$ die kleinste monotone Klasse ist, die $\mc{G}_n$ enthält, ist auch $\sigma(\mc{G}_n) \supseteq m(\mc{G}_n)$.

\subsubsection*{$m(\mc{G}_n) \supseteq F_\sigma \cap G_\delta$}
Für $A \in F_\sigma \cap G_\delta$ ist $A \in G_\delta$, es gibt also eine Folge $(U_n)_{n \in \N}$ auf $\mc{G}_n$ mit $A = \bigcap_{n \in \N} U_n$. Für $n \in \N$ sei $B_n := \bigcap_{k=1}^n U_k$; $(B_n)_{n \in \N}$ ist eine fallende Folge auf $\mc{G}_n$, da endliche Schnitte offener Mengen offen sind, und es gilt $\bigcap_{n \in \N} B_n = \bigcap_{n \in \N} U_n = A$. Da $m(\mc{G}_n)$ ein monotone Klasse mit $\mc{G}_n \subseteq m(\mc{G}_n)$ ist, ist $(B_n)_{n \in \N}$ auch eine fallende Folge auf $\mc{G}_n$. Also ist $A = \bigcap_{n \in \N} B_n \in m(\mc{G}_n)$. Das zeigt, dass $F_\sigma \cap G_\delta \subseteq m(\mc{G}_n)$.

\subsubsection*{$F_\sigma \cap G_\delta \supseteq \sigma(\mc{G}_n)$}
Wegen der Minimalitätseigenschaft von $\sigma(\mc{G}_n)$ genügt es zu zeigen, dass $F_\sigma \cap G_\delta$ eine $\sigma$-Algebra mit $\mc{G}_n \in F_\sigma \cap G_\delta$ ist.

Wie aus dem Hinweis bekannt ist $F_\sigma \cap G_\delta$ bereits eine Algebra, es genügt also die Abgeschlossenheit bezüglich abzählbarer Vereinigungen zu zeigen. Sei $(A_n)_{n \in \N}$ eine Folge auf $F_\sigma \cap G_\delta$. Da $(A_n)_{n \in \N}$ auch eine Folge auf $F_\sigma$ ist, gibt es für alle $n \in \N$ eine Folge $(K^n_k)_{k \in \N}$ von abgeschlossenen Mengen von $\R^n$ mit $A_n = \bigcup_{k \in \N} K^n_k$. Es ist daher
\[
 \bigcup_{n \in \N} A_n
 = \bigcup_{n \in \N} \bigcup_{k \in \N} K^n_k
 = \bigcup_{n,k \in \N} K^n_k
 \in F_\sigma.
\]
Da $(A_n)_{n \in \N}$ auch eine Folge auf $G_\delta$ ist, gibt es für alle $n \in \N$ eine Folge $(U^n_k)_{k \in K}$ von offenen Mengen von $\R^n$ mit $A_n = \bigcap_{k \in \N} U^n_k$. Es ist daher auch
\[
 \bigcup_{n \in \N} A_n
 = \bigcup_{n \in \N} \bigcap_{k \in \N} U^n_k
 \subseteq \bigcap_{k \in \N} \underbrace{\bigcup_{n \in \N} U^n_k}_{\text{offen in }\R^n}
 \in G_\delta.
\]
Also ist $\bigcup_{n \in \N} A_n \in F_\sigma \cap G_\delta$.

Dass $\mc{G}_n \subseteq F_\sigma \cap G_\delta$ folgt daraus, dass für alle $U \in \mc{G}_n$, wie in Aufgabenteil \textbf{a)} gezeigt, $U \in F_\sigma$, und auch $U = U \cap \bigcap_{n \geq 1} \R^n \in G_\delta$.





\section{(Gegenbeispiele)}


\subsection{}
Sei $X = (0,1)$. Wie aus der Vorlesung bekannt ist
\[
 \mu: \mc{P}(X) \rightarrow [0,\infty] \text{ mit }
 \mu(A) =
 \begin{cases}
  0      & \text{falls } A = \emptyset, \\
  \infty & \text{sonst},
 \end{cases}
\]
ein Maß auf $(X,\mc{P}(X))$. Für die fallende Folge $(A_n)_{n \geq 1}$ mit $A_n := \left(0,\frac{1}{n}\right)$ ist jedoch
\[
 0
 = \mu(\emptyset)
 = \mu\left( \bigcap_{n \geq 1} A_n \right)
 \neq \limes{n}{\infty} \mu(A_n)
 = \limes{n}{\infty} \infty
 = \infty.
\]


\subsection{}
\begin{bem}
 Sei $\mu$ ein Maß auf $(X,\A)$. Dann ist für alle $c > 0$
 \[
  \mu': \A \rightarrow [0,\infty] \text{ mit } \mu'(A) = c\mu(A)
 \]
 ein Maß auf $(X,\A)$.
\end{bem}
\begin{proof}[Beweis der Bemerkung]
 Es ist $\mu(\emptyset) = c \cdot 0 = 0$. Für eine Folge $(A_n)_{n \in \N}$ disjunkter Mengen auf $\A$ ist
 \[
  \mu'\left( \bigcup_{n \in \N} A_n \right)
  = c \cdot \mu\left( \bigcup_{n \in \N} A_n \right)
  = c \sum_{n \in \N} \mu(A_n)
  = \sum_{n \in \N} c\mu(A_n)
  = \sum_{n \in \N} \mu'(A_n).
 \]
\end{proof}

Für alle $n \geq 1$ ist
\[
 \mu_n: \mc{P}(\N) \rightarrow [0,\infty] \text{ mit }
 \mu(A) =
 \begin{cases}
  \frac{|A|}{n} & \text{falls $A$ endlich ist} \\
  \infty        & \text{sonst}
 \end{cases}
\]
als skalares Vielfaches des Zählmaßes ($c = \frac{1}{n}$) ein Maß auf $(\N,\mc{P}(\N))$: Auch existiert für alle $A \in \mc{P}(\N)$
\[
 \mu(A)
 := \limes{n}{\infty} \mu_n(A)
 =
 \begin{cases}
  0      & \text{falls $A$ endlich ist} \\
  \infty & \text{sonst}
 \end{cases}.
\]
Es ist jedoch $\mu$ kein Maß auf $(\N,\mc{P}(\N))$, da $\mu$ wegen
\[
 \mu\left( \bigcup_{n \in \N} \{n\} \right)
 = \mu(\N)
 = \infty
 \neq 0
 = \sum_{n \in \N} 0
 = \sum_{n \in \N} \mu(\{n\}).
\]
nicht $\sigma$-additiv ist.


\subsection{}
Es sei $X := \{1,2\}$ und $\A := \mc{P}(X) = \{\emptyset, \{1\}, \{2\}, \{1,2\}\}$. Es sei $\mu : \A \rightarrow (-\infty,\infty)$ definiert als $\mu(\emptyset) := 0$, $\mu(\{1\}) := 1$, $\mu(\{2\}) := -1$ und $\mu(\{1,2\}) := 0$. Offenbar erfüllt $\mu$ die geforderten Eigenschaften auf $(X,\A)$. Es ist aber $|\mu|$ kein Maß auf $(X,\A)$, da
\[
 |\mu|(\{1,2\}) = 0 \neq 2 = |\mu(\{1\})|+|\mu(\{2\})| = |\mu|(\{1\}) + |\mu|(\{2\}),
\]
$|\mu|$ also nicht endlich additiv ist.





\section{(Monotone Folgen von Maßen)}


\subsection{}
$\mu$ ist wohldefiniert:
Für alle $A \in \A$ ist die Folge $(\mu_n(A))_{n \in \N}$ monoton steigend, und wie aus Analysis I bekannt ist eine monoton steigend Folge entweder beschränkt, und damit konvergent, oder divergent gegen $\infty$.

Für paarweise disjunkte $A_1, \ldots, A_N \in \A$ ist
\begin{align*}
 \mu\left( \bigcup_{k=1}^N A_k \right)
 &= \limes{n}{\infty} \mu_n\left( \bigcup_{k=1}^N A_k \right)
 = \limes{n}{\infty} \sum_{k=1}^N \mu_n(A_k) \\
 &= \sum_{k=1}^N \limes{n}{\infty} \mu_n(A_k)
 = \sum_{k=1}^N \mu(A_k),
\end{align*}
da jedes der $\mu_n$ als Maß endlich additiv ist und die einzelnen Grenzwerte alle existieren. $\mu$ ist also endlich additiv. Wie aus der Vorlesung bekannt ist $\mu$ daher genau dann ein Maß, wenn $\limes{n}{\infty} \mu(A_n) = \mu\left( \bigcup_{n \in \N} A_n \right)$ für jede wachsende Folge $(A_n)_{n \in \N}$ auf $\A$. Sei also $(A_n)_{n \in \N}$ ein wachsende Folge.

Die Monotonie von $\mu$ folgt bereits aus der endlichen Additivität, denn für $A, B \in \A$ mit $A \subseteq B$ ist $B = A\ \dot\cup\ (B \setminus A)$, also $\mu(B) = \mu(A) + \mu(B \setminus A)$, also $\mu(A) \leq \mu(B)$. Aufgrund der Monotonie von $\mu$ ist nun $\mu(A_m) \leq \mu\left(\bigcup_{n \in \N} A_n\right)$ für alle $m \in \N$, also auch $\limes{n}{\infty} \mu(A_n) \leq \mu\left(\bigcup_{n \in \N} A_n\right)$.

Da die Folge von Maßen $(\mu_m)_{m \in \N}$ monoton steigend im Sinne der Aufgabenstellung ist, ist $\mu_m(A_n) \leq \mu(A_n)$ für alle $m,n \in \N$, also auch $\limes{n}{\infty} \mu_m(A_n) \leq \limes{n}{\infty} \mu(A_n)$ für alle $m \in \N$. Da $\mu_m$ für alle $m \in \N$ ein Maß ist, ist dabei für alle $m \in \N$
\[
 \mu_m \left(\bigcup_{n \in \N} A_n\right) = \limes{n}{\infty} \mu_m(A_n) \leq \limes{n}{\infty} \mu(A_n),
\]
also auch
\[
 \mu \left(\bigcup_{n \in \N} A_n\right)
 = \limes{m}{\infty} \mu_m \left(\bigcup_{n \in \N} A_n\right)
 \leq \limes{n}{\infty} \mu(A_n).
\]

Dies zeigt die Gleichheit.


\subsection{}
\begin{bem}
 Sei $(\mu_m)_{m \in \N}$ eine Folge von Maßen auf $(X,\A)$. Dann ist auch $\mu := \sum_{m \in \N} \mu_m$ ein Maß auf $(X,\A)$.
\end{bem}
\begin{proof}[Beweis der Bemerkung]
 Es ist
 \[
  \mu(\emptyset) = \sum_{m \in \N} \mu_m(\emptyset) = \sum_{m \in \N} 0 = 0.
 \]
 Sei $(A_n)_{n \in \N}$ ein Folge disjunkter Mengen auf $\A$. Dann ist
 \begin{align*}
  \mu\left( \bigcup_{n \in \N} A_n \right)
  &= \sum_{m \in \N} \mu_m\left( \bigcup_{n \in \N} A_n \right)
  = \sum_{m \in \N} \sum_{n \in \N} \mu_m(A_n) \\
  &= \sum_{n \in \N} \sum_{m \in \N} \mu_m(A_n)
  = \sum_{n \in \N} \mu(A_n),
 \end{align*}
 wobei die entsprechenden Umordnungen wegen der Nichtnegativität aller Summanden gültig sind.
\end{proof}

Aus der Bemerkung folgt nun direkt, dass $\mu = \sum_{n \in \N} \delta_{x_n}$ ein Maß auf $(\R,\mc{P}(\R))$ definiert.

Ist die Folge $(x_n)_{n \in \N}$ eine Folge auf $\R$ so, dass $\mu$ auf beschränkten Mengen endlich ist, so ist insbesondere $\mu( [-r,r] ) < \infty$ für alle $r > 0$, d.h. für alle $r > 0$ gibt es nur endlich viele $n \in \N$ mit $x_n \in [-r, r]$. Für alle $r > 0$ ist daher $|x_n| > r$ für fast alle $n \in \N$, was genau dann der Fall ist, wenn $\limes{n}{\infty} |x_n| = \infty$.

Dies Bedingung ist nicht nur notwendig, sondern auch hinreichend: Ist $(x_n)_{n \in \N}$ ein Folge auf $\R$ mit $\limes{n}{\infty} |x_n| = \infty$, und $A \subseteq \R$ eine beschränkte Menge, so gibt es ein $r > 0$ mit $A \subseteq [-r,r]$. Da $\limes{n}{\infty} |x_n| = \infty$ gibt es ein $n_0 \in \N$ mit $|x_n| > r$, also $x_n \not\in [-r,r]$, für alle $n > n_0$. Es ist daher $\mu(A) \leq \mu([-r,r]) \leq n_0+1$ endlich.

$\mu$ ist genau dann nicht $\sigma$-endlich, wenn es ein $x \in \R$ gibt mit $x_n = x$ für unendlich viele $n \in \N$:

Gibt es ein solches $x$, so ist $\mu(A) = \infty$ für alle $A \subseteq \R$ mit $x \in A$. Notwendige Bedingung für $\sigma$-Endlichkeit ist aber, dass es ein $B \subseteq \R$ mit $x \in B$ und $\mu(B) < \infty$ gibt.

Gibt es kein solches $x$, so sei $X := \{x_n\}_{n \in \N}$ die Punktmenge der Folge $(x_n)_{n \in \N}$. Für $n \in \N$ sei $A_n := [n,n+1] \setminus X$. Aus der Konstruktion der $A_n$ geht direkt hervor, dass $\mu(A_n) = 0$ für alle $n \in \N$. Auch ist nach Annahme $\mu(\{x_n\}) < \infty$ für alle $n \in \N$. Die Folge $(B_n)_{n \in \N}$ sei nun definiert als $B_{2n} := A_n$ und $B_{2n+1} := x_n$ für je alle $n \in \N$; man bemerke, dass $\mu(B_n) < \infty$ für alle $n \in \N$. Da auch
\[
 \R
 = X \cup (\R \setminus X)
 = \left(\bigcup_{n \in \N} \{x_n\} \right) \cup \left( \bigcup_{n \in \N} A_n \right)
 = \bigcup_{n \in \N} B_n
\]
ist $\mu$ daher $\sigma$-endlich.








\end{document}
