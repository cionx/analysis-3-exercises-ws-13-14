\documentclass[a4paper,10pt]{article}
%\documentclass[a4paper,10pt]{scrartcl}

\usepackage{xltxtra}
\usepackage[ngerman]{babel}
\usepackage{amsmath}
\usepackage{amssymb}
\usepackage{amsthm}
\usepackage{mathtools}
\usepackage{nicefrac}
\usepackage{enumerate}
\usepackage{leftidx}

\makeatletter
\g@addto@macro\th@definition{\thm@headpunct{:}}
\makeatother
\newcounter{saetze}
\newtheorem{lem}[saetze]{Lemma}
\newtheorem{beh}[saetze]{Behauptung}
\newtheorem{bem}[saetze]{Bemerkung}
\newtheorem*{ia}{Induktionsanfang}
\newtheorem*{is}{Induktionsschritt}

\renewcommand{\thesection}{Aufgabe \arabic{section}.}
\renewcommand{\thesubsection}{\alph{subsection})}
\renewcommand{\thesubsubsection}{(\roman{subsubsection})}

\makeatletter
\def\moverlay{\mathpalette\mov@rlay}
\def\mov@rlay#1#2{\leavevmode\vtop{%
   \baselineskip\z@skip \lineskiplimit-\maxdimen
   \ialign{\hfil$\m@th#1##$\hfil\cr#2\crcr}}}
\newcommand{\charfusion}[3][\mathord]{
    #1{\ifx#1\mathop\vphantom{#2}\fi
        \mathpalette\mov@rlay{#2\cr#3}
      }
    \ifx#1\mathop\expandafter\displaylimits\fi}
\makeatother

\newcommand{\N}{\mathbb{N}}
\newcommand{\Z}{\mathbb{Z}}
\newcommand{\Q}{\mathbb{Q}}
\newcommand{\R}{\mathbb{R}}
\newcommand{\C}{\mathbb{C}}
\newcommand{\A}{\mathcal{A}}
\newcommand{\La}{\mathcal{L}}
\newcommand{\dr}{\,\text{d}r}
\newcommand{\dt}{\,\text{d}t}
\newcommand{\du}{\,\text{d}u}
\newcommand{\dx}{\,\text{d}x}
\newcommand{\dy}{\,\text{d}y}
\newcommand{\dmu}{\,\text{d}\mu}
\newcommand{\dlambda}{\,\text{d}\lambda}
\newcommand{\dvarphi}{\,\text{d}\varphi}
\newcommand{\Img}{\operatorname{Im}}
\newcommand{\Real}{\operatorname{Re}}
\newcommand{\Imag}{\operatorname{Im}}
\newcommand{\sgn}{\operatorname{sgn}}
\newcommand{\diag}{\operatorname{diag}}
\newcommand{\Vol}{\operatorname{Vol}}
\newcommand{\dotcup}{\ensuremath{\mathaccent\cdot\cup}}
\newcommand{\bigdotcup}{\charfusion[\mathop]{\bigcup}{\cdot}}
\newcommand{\ceil}[1]{\left\lceil{#1}\right\rceil}
\newcommand{\floor}[1]{\left\lfloor{#1}\right\rfloor}
\newcommand{\mc}[1]{\mathcal{#1}}
\newcommand{\limes}[2]{\lim_{#1 \rightarrow #2}}
\newcommand{\limessup}[1]{\limsup_{#1 \rightarrow \infty}}
\newcommand{\limesinf}[1]{\liminf_{#1 \rightarrow \infty}}
\newcommand{\vect}[1]{\begin{pmatrix}#1\end{pmatrix}}
\newcommand{\partd}[2]{\frac{\partial #1}{\partial #2}}
\newcommand{\op}[1]{\left\|#1\right\|_{\text{op}}}

\makeatletter
\renewcommand*\env@matrix[1][*\c@MaxMatrixCols c]{%
  \hskip -\arraycolsep
  \let\@ifnextchar\new@ifnextchar
  \array{#1}}
\makeatother

\setromanfont[Mapping=tex-text]{Linux Libertine O}
% \setsansfont[Mapping=tex-text]{DejaVu Sans}
% \setmonofont[Mapping=tex-text]{DejaVu Sans Mono}
\parindent 0pt

\title{\sc Analysis III \\ \Large 9. Aufgabenblatt}
\author{Jendrik Stelzner}
\date{\today}

\begin{document}
\maketitle





Da ich diese Woche keine Lust auf Analysis hatte, habe ich nur zwei der Aufgaben bearbeitet.





\section{(Riemann vs. Lebesgue)}


Wir gehen im Folgenden davon aus, dass $f$, bzw. der Ausdruck $\sin(x)/x$, an der Stelle $x = 0$ als $f(0) = 1$ definiert ist. Aus Analysis 1 ist bekannt, dass $f$ dann auf ganz $\R$ stetig differenzierbar ist.


\subsection{}
Für alle $n \in \N$ ist
\[
 \int_{(0,\infty)} |f| d\lambda
 \geq \int_{[0,n\pi]} |f| d\lambda
 = \int_0^{n\pi} |f(x)| \dx
 = \int_0^{n\pi} \frac{|\sin(x)|}{x} \dx
\]
Wir dürfen das Lebesgue- mit dem Riemann-Integral vertauschen, da $|f|$ auf $[0,n\pi]$ beschränkt und stetig ist. Da für alle $n \in \N$
\begin{align*}
 \int_0^{n\pi} \frac{|\sin(x)|}{x} \dx
 &= \sum_{k=1}^n \int_{(k-1)\pi}^{k\pi} \frac{|\sin(x)|}{x} \dx
 \geq \sum_{k=1}^n \frac{1}{k\pi} \int_{(k-1)\pi}^{k\pi} |\sin(x)| \dx \\
 &= \sum_{k=1}^n \frac{1}{k\pi} \int_0^\pi \sin(x) \dx
 = \sum_{k=1} ^n \frac{2}{k\pi},
\end{align*}
ist für alle $n \in \N$
\[
 \int_{(0,\infty)} |f| \dlambda
 \geq \frac{2}{\pi} \sum_{k=1}^n \frac{1}{k},
\]
also
\[
 \int_{(0,\infty)} |f| \dlambda
 \geq \frac{2}{\pi} \sum_{k=1}^\infty \frac{1}{k}
 = \infty.
\]
Es ist also $\int_{(0\infty)} |f| \dlambda = \infty$. Daher ist $|f|$ auf $(0,\infty)$ nicht Lebesgue-integrierbar, also auch $f$ nicht.


\subsection*{b) und c)}
Sei $R \geq 0$ beliebig aber fest. Wir betrachten die Funktion
\[
 g : [0,R] \times [0,\infty) \rightarrow [-1,1], (x,t) \mapsto \sin(x) e^{-xt}.
\]
Diese ist stetig, also Borell-messbar. Wir zeigen zunächst, dass $g$ integrierbar ist. Für $|g|$ lässt sich der Satz von Tonelli anwenden, weshalb
\begin{align*}
 \int_{[0,R] \times [0,\infty)} |g| \dlambda_2
 &= \int_{[0,R]} \int_{[0,\infty)} |g(x,t)| \dlambda(t) \dlambda(x) \\
 &= \int_{[0,R]} \int_{[0,\infty)} |\sin(x)|e^{-xt} \dlambda(t) \dlambda(x).
\end{align*}
Da $|\sin(x)|e^{-xt}$ für alle $x \in [0,R]$ auf jedem kompakten Teilintervall von $[0,\infty)$ Riemann-integrierbar ist, und auf $[0,\infty)$ uneigentlich Riemann-Integrierbar mit
\[
 \int_0^\infty |\sin(x)|e^{-xt} \dt
 = |\sin(x)| \int_0^\infty e^{-xt} \dt
 = \frac{|\sin(x)|}{x},
\]
ist für alle $x \in [0,R]$
\[
 \int_{[0,\infty)} |\sin(x)|e^{-xt} \dlambda(t)
 = \int_0^\infty |\sin(x)|e^{-xt} \dt
 = \frac{|\sin(x)|}{x}.
\]
Daher ist
\[
 \int_{[0,R] \times [0,\infty)} |g| \dlambda_2
 = \int_{[0,R]} \frac{|\sin(x)|}{x} \dlambda(x)
 < \infty,
\]
da $\lambda([0,R]) = R < \infty$ und $0 \leq |\sin(x)|/x \leq 1$. Es ist also $|g|$ integrierbar, und damit auch $g$.

Da $g$ integrierbar ist, lässt sich für $\int_{[0,R] \times [0,\infty)} g \dlambda_2$ der Satz von Fubini anwenden. Wir bemerken aber zunächst analog zur obigen Argumentation, dass für alle $x \in [0,R]$
\begin{align*}
 \int_{[0,\infty)} g(x,t) \dlambda(t)
 &= \int_{[0,\infty)} \sin(x) e^{-xt} \dlambda(t) \\
 &= \sin(x) \int_{[0,\infty)} e^{-xt} \dlambda(t)
 = \frac{\sin(x)}{x},
\end{align*}
wobei auch genutzt wird, dass $|g|$ auf $[0,\infty)$ uneigentlich Riemann-integrierbar ist. Daher ergibt sich mit dem Satz von Fubini zum einen, dass
\begin{align*}
 \int_{[0,R] \times [0,\infty)} g \dlambda_2
 &= \int_{[0,R]} \int_{[0,\infty)} g(x,t) \dlambda(t) \dlambda(x)
 = \int_{[0,R]} \frac{\sin(x)}{x} \dlambda(x) \\
 &= \int_{[0,R]} f(x) \dlambda(x)
 = \int_0^R f(x) \dx.
\end{align*}
Dabei haben wir im letzten Schritt genutzt, dass $f$ auf $[0,R]$ beschränkt und stetig ist, also Riemann- und Lebesgue-Integral übereinstimmen.

Um das andere, aus dem Satz von Fubini folgende, Integral zu bestimmen, bemerken wir, dass für alle $t \in [0,\infty)$
\[
 \int_{[0,R]} g(x,t) \dlambda(x)
 = \int_{[0,R]} \sin(x) e^{-xt} \dlambda(x)
 = \int_0^R \sin(x) e^{-xt} \dx.
\]
Dabei haben wir genutzt, dass $g(x,t)$ für alle $t \in [0,\infty)$ auf $[0,R]$ beschränkt und stetig ist, also Riemann- und Lebesgue-Integral übereinstimmen. Es ergibt sich für alle $t \in (0,\infty)$ durch partielle Integration, dass
\begin{align*}
 &\, \int_0^R \sin(x) e^{-xt} \dx
 = \left. -\frac{1}{t} \sin(x) e^{-xt} \right|_{x=0}^R + \frac{1}{t} \int_0^R \cos(x) e^{-xt} \dx \\
 =&\, \left.\left( -\frac{1}{t} \sin(x) e^{-xt} - \frac{1}{t^2} \cos(x) e^{-xt} \right)\right|_{x=0}^R - \frac{1}{t^2} \int_0^R \sin(x) e^{-xt} \dx,
\end{align*}
also
\begin{align*}
 \underbrace{\left(1+\frac{1}{t^2}\right)}_{=\frac{1+t^2}{t^2}} \int_0^R \sin(x) e^{-xt} \dx
 &= \left.\left( -\frac{1}{t} \sin(x) e^{-xt} - \frac{1}{t^2} \cos(x) e^{-xt} \right)\right|_{x=0}^R \\
 &= -\frac{1}{t} \sin(R) e^{-Rt} - \frac{1}{t^2} \cos(R) e^{-Rt} + \frac{1}{t^2},
\end{align*}
und daher für alle $t \in (0,\infty)$
\[
 \int_0^R \sin(x) e^{-xt} \dx
 = -\frac{t \sin(R) e^{-Rt}}{1+t^2} - \frac{\cos(R) e^{-Rt}}{1+t^2} + \frac{1}{1+t^2}.
\]
Man bemerke, dass diese Gleichung auch für $t=0$ gilt, da
\[
 \int_0^R \sin(x) \dx
 = \left. -\cos(x) \right|_{x=0}^R
 = -\cos(R) + 1.
\]
Nach dem Satz von Fubini gilt also auch
\begin{align*}
 \int_{[0,R] \times [0,\infty)} g \dlambda_2
 &= \int_{[0,\infty)} \int_{[0,R]} g(x,t) \dlambda(x) \dlambda(t) \\
 &= \int_{[0,\infty)} -\frac{t \sin(R) e^{-Rt}}{1+t^2} - \frac{\cos(R) e^{-Rt}}{1+t^2} + \frac{1}{1+t^2} \dlambda(t),
\end{align*}
wobei aus dem Satz auch folgt, dass dieses hässliche Integral existiert.
Zusammengefasst ergibt sich also aus dem Satz von Fubini
\[
 \int_0^R f(x) \dx
 = \int_{[0,\infty)} -\frac{t \sin(R) e^{-Rt}}{1+t^2} - \frac{\cos(R) e^{-Rt}}{1+t^2} + \frac{1}{1+t^2} \dlambda(t),
\]

Sei nun $(R_n)_{n \in \N}$ eine Folge auf $(0,\infty)$ mit $\limes{n}{\infty} R_n = \infty$. Um den Grenzwert $\limes{n}{\infty} \int_0^{R_n} f(x) \dx$ zu bestimmen (und ob dieser überhaupt existiert), betrachten wir
\[
 \limes{n}{\infty} \int_{[0,\infty)} -\frac{t \sin(R_n) e^{-{R_n}t}}{1+t^2} - \frac{\cos(R_n) e^{-{R_n}t}}{1+t^2} + \frac{1}{1+t^2} \dlambda(t).
\]
Wir wollen den Grenzwert und das Integral mithilfe des Satzes über dominierte Konvergenz vertauschen. Offenbar ist für alle $t \in [0,\infty)$
\[
 \limes{n}{\infty} -\frac{t \sin(R_n) e^{-{R_n}t}}{1+t^2} - \frac{\cos(R_n) e^{-{R_n}t}}{1+t^2} + \frac{1}{1+t^2}
 = \frac{1}{1+t^2}.
\]
Auch ist für alle $n \in \N$ für alle $t \in [0,\infty)$
\begin{align*}
 &\, \left| -\frac{t \sin(R_n) e^{-{R_n}t}}{1+t^2} - \frac{\cos(R_n) e^{-{R_n}t}}{1+t^2} + \frac{1}{1+t^2} \right| \\
 \leq&\, \frac{t |\sin(R_n)| e^{-{R_n}t}}{1+t^2} + \frac{|\cos(R_n)| e^{-{R_n}t}}{1+t^2} + \frac{1}{1+t^2} \\
 \leq&\, 2e^{-R_n t} + \frac{1}{1+t^2}.
\end{align*}
Diese Funktion ist auf $[0,\infty)$ Lebesgue-integrierbar: Sie ist auf jedem kompakten Teilintervall von $[0,\infty)$ Riemann-integrierbar, nichtnegativ, und das uneigentliche Riemanintegral
\begin{align*}
 \int_0^\infty 2e^{-R_n t} + \frac{1}{1+t^2} \dt
 &= \limes{s}{\infty} \int_0^s 2e^{-R_n t} + \frac{1}{1+t^2} \dt \\
 &= \limes{s}{\infty} \left.\left( -\frac{2}{R_n} e^{-R_n t} + \arctan(t) \right)\right|_{t=0}^s \\
 &= \limes{s}{\infty} -\frac{2}{R_n} e^{-R_n s} + \arctan(s) + \frac{2}{R_n}
 = \frac{\pi}{2} + \frac{2}{R_n}
\end{align*}
existiert. Daher ist die Funktion auf $[0,\infty)$ auch Lebesgue-integrierbar.

Wir können also den Satz über dominierte Konvergenz anwenden, und erhalten, dass
\begin{align*}
 &\, \limes{n}{\infty} \int_{[0,\infty)} -\frac{t \sin(R_n) e^{-{R_n}t}}{1+t^2} - \frac{\cos(R_n) e^{-{R_n}t}}{1+t^2} + \frac{1}{1+t^2} \dlambda(t) \\
 =&\, \int_{[0,\infty)} \limes{n}{\infty}  -\frac{t \sin(R_n) e^{-{R_n}t}}{1+t^2} - \frac{\cos(R_n) e^{-{R_n}t}}{1+t^2} + \frac{1}{1+t^2} \dlambda(t) \\
 =&\, \int_{[0,\infty)} \frac{1}{1+t^2} \dlambda(t).
\end{align*}
Die Funktion $1/(1+t^2)$ auf jedem kompakten Teilintervall von $[0,\infty)$ Riemann-integrierbar ist, nichtnegativ und auf $[0,\infty)$ uneigentlich Riemann-integrierbar mit
\[
 \int_0^\infty \frac{1}{1+t^2} \dt
 = \limes{s}{\infty} \int_0^s \frac{1}{1+t^2} \dt
 = \limes{s}{\infty} \left. \arctan(t) \right|_{t=0}^s
 = \limes{s}{\infty} \arctan(s)
 = \frac{\pi}{2}.
\]
Daher stimmen Riemann- und Lebesgue-Integral überein, d.h. es ist
\[
 \int_{[0,\infty)} \frac{1}{1+t^2} \dlambda(t) = \frac{\pi}{2}.
\]
Es ist also
\[
 \limes{n}{\infty} \int_0^{R_n} f(x) \dx
 = \frac{\pi}{2}.
\]
Aus der Beliebigkeit der Folge $R_n$ auf $(0,\infty)$ mit $\limes{n}{\infty} R_n = \infty$ folgt, dass bereits
\[
 \int_0^\infty f(x) \dx
 = \limes{R}{\infty} \int_0^R f(x) \dx
 = \frac{\pi}{2}.
\]





\addtocounter{section}{2}





\section{(Kugeln)}
Für alle $N \geq 1$ und $R \geq 0$ ist
\[
 \lambda_N\left( B_R^N \right) = R^N \alpha_N,
\]
denn für die Diagonalmatrix $T = \diag(R, \ldots, R) \in \R^{N \times N}$ gilt
\[
 \lambda_N\left( B_R^N \right)
 = \lambda_N\left( T(B_R^1) \right)
 = |\det(T)|\ \lambda_N(B_R^1)
 = R^N \alpha_N.
\]
Sei nun $N \geq 3$ beliebig aber fest. Für alle $(x_1, x_2) \in \R^2$ ist
\begin{align*}
 \left( B_1^N \right)_{(x_1, x_2)}
 &= \left\{ (x_3, \ldots, x_N) \in \R^{N-2} : \sum_{i=1}^{N-2} x^2_i \leq 1 \right\} \\
 &= \left\{ (x_3, \ldots, x_N) \in \R^{N-2} : \sum_{i=3}^{N-2} x^2_i \leq 1-(x_1^2+x_2^2) \right\} \\
 &= B_{1-\sqrt{1-(x_1^2+x_2^2)}}^{N-2}.
\end{align*}

Da $\mc{B}\left(\R^N\right) = \mc{B}\left(\R^2\right) \times \mc{B}\left(\R^{N-2}\right)$ und $\lambda_N = \lambda_2 \times \lambda_{N-2}$ ergibt sich aus der Definition des Produktmaßes, dass
\begin{align*}
 \alpha_N = \lambda_N\left(B_1^N\right)
 &= \int_{\R^2} \lambda_{N-2}\left(\left(B_1^N\right)_{(x,y)}\right) \dlambda_2(x,y) \\
 &= \int_A \lambda_{N-2}\left( B_{\sqrt{1-(x^2+y^2)}}^{N-2} \right) \dlambda_2(x,y) \\
 &= \alpha_{N-2} \int_A \sqrt{1-(x^2+y^2)}^{N-2} \dlambda_2(x,y)
\end{align*}
wobei
\[
 A = \{ (x,y) \in \R^2 : x^2 + y^2 \leq 1 \}.
\]
Umwandeln in Polarkoordinaten ergibt, dass
\begin{align*}
  &\,\int_A \sqrt{1-(x^2+y^2)}^{N-2} \dlambda_2(x,y) \\
 =&\, \int_{(0,1) \times (0,2\pi)} r \sqrt{1 - ((r \cos \varphi)^2 + (r \sin \varphi)^2}^{N-2} \dlambda_2(r,\varphi) \\
 =&\, \int_{(0,1) \times (0,2\pi)} r\sqrt{1-r^2}^{N-2} \dlambda_2(r,\varphi).
\end{align*} 
(Streng genommen müssten wir $A$ zuerst durch eine passende offene Menge ersetzen. Es ist jedoch klar, dass der Rand von $A$ für den Integral nicht von Bedeutung ist, da es sich um eine Nullmenge handelt, so dass wir
\[
 A' = \{(x,y) \in R^2 : x^2 + y^2 < 1\} = A^\circ
\]
wählen können.
)
Anwenden des Satzes von Tonelli ergibt, dass
\begin{align*}
 \int_{(0,1) \times (0,2\pi)} r\sqrt{1-r^2}^{N-2} \dlambda_2(r,\varphi)
 &= \int_{(0,1)} \int_{(0,2\pi)} r\sqrt{1-r^2}^{N-2} \dvarphi \dr \\
 &= 2 \pi \int_{(0,1)} r \sqrt{1-r^2}^{N-2} \dr.
\end{align*}
Da die Funktion $r\sqrt{1-r^2}^{N-2}$ auf $(0,1)$ beschränkt und stetig ist, entspricht das Lebesgue-Integral dem Riemann-Integral. Dabei ergibt sich mit den Substitutionen $r = \sin t$, $\dr = \cos t \dt$ und $\cos t = x$ und $-\sin t \dt = \dx$, dass
\begin{align*}
 \int_0^1 r \sqrt{1-r^2}^{N-2} \dr
 &= \int_0^{\pi/2} \sin(t) \cos^{N-1}(t) \dt
 = -\int_1^0 x^{N-1} \dx \\
 &= \int_0^1 x^{N-1} \dx
 = \left. \frac{x^N}{N} \right|_{x=0}^1
 = \frac{1}{N}.
\end{align*}
Zusammengefasst ergibt sich damit, dass
\[
 \alpha_N = 2 \pi N^{-1} \alpha_{N-2}.
\]



































\end{document}
