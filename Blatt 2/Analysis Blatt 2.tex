\documentclass[a4paper,10pt]{article}
%\documentclass[a4paper,10pt]{scrartcl}

\usepackage{xltxtra}
\usepackage[ngerman]{babel}
\usepackage{amsmath}
\usepackage{amssymb}
\usepackage{amsthm}
\usepackage{mathtools}
\usepackage{nicefrac}
\usepackage{enumerate}
\usepackage{leftidx}

\theoremstyle{definition}
\newtheorem*{lem}{Lemma}
\newtheorem*{beh}{Behauptung}
\newtheorem*{bem}{Bemerkung}
\newtheorem*{ia}{Induktionsanfang}
\newtheorem*{is}{Induktionsschritt}

\renewcommand{\thesection}{Aufgabe \arabic{section}.}
\renewcommand{\thesubsection}{\alph{subsection})}
\renewcommand{\thesubsubsection}{(\roman{subsubsection})}

\newcommand{\N}{\mathbb{N}}
\newcommand{\Z}{\mathbb{Z}}
\newcommand{\Q}{\mathbb{Q}}
\newcommand{\R}{\mathbb{R}}
\newcommand{\C}{\mathbb{C}}
\newcommand{\A}{\mathcal{A}}
\newcommand{\La}{\mathcal{L}}
\newcommand{\dx}{\,\text{d}x}
\newcommand{\dy}{\,\text{d}y}
\newcommand{\dt}{\,\text{d}t}
\newcommand{\du}{\,\text{d}u}
\newcommand{\mc}[1]{\mathcal{#1}}
\newcommand{\Img}{\operatorname{Im}}
\newcommand{\Real}{\operatorname{Re}}
\newcommand{\Imag}{\operatorname{Im}}
\newcommand{\sgn}{\operatorname{sgn}}
\newcommand{\limes}[2]{\lim_{#1 \rightarrow #2}}
\newcommand{\limessup}[1]{\limsup_{#1 \rightarrow \infty}}
\newcommand{\limesinf}[1]{\liminf_{#1 \rightarrow \infty}}
\newcommand{\vect}[1]{\begin{pmatrix}#1\end{pmatrix}}
\newcommand{\partd}[2]{\frac{\partial #1}{\partial #2}}
\newcommand{\op}[1]{\left\|#1\right\|_{\text{op}}}

\makeatletter
\renewcommand*\env@matrix[1][*\c@MaxMatrixCols c]{%
  \hskip -\arraycolsep
  \let\@ifnextchar\new@ifnextchar
  \array{#1}}
\makeatother

\setromanfont[Mapping=tex-text]{Linux Libertine O}
% \setsansfont[Mapping=tex-text]{DejaVu Sans}
% \setmonofont[Mapping=tex-text]{DejaVu Sans Mono}
\parindent 0pt

\title{Analysis 3 — Übung 2}
\author{Jendrik Stelzner}
\date{\today}

\begin{document}
\maketitle





\section{(Monotone Klassen und $\sigma$-Algebren)}


\subsection{}
Angenommen, $\A$ ist eine Algebra. Nach der Definition einer Algebra ist damit $X \in \A$ und für alle $A \in \A$ auch $A^c \in \A$; es muss also nur noch die $\sigma$-Additivität gezeigt werden.

Sei hierfür $(A_n)_{n \in \N}$ eine Folge auf $\A$. Für alle $n \in N$ sei $B_n := \bigcup_{k=0}^n A_k$. Es ist offenbar $\bigcup_{n \in \N} A_n = \bigcup_{n \in \N} B_n$. Da $(B_n)_{n \in \N}$ eine wachsende Folge auf $\A$ ist, und $\A$ eine monotone Klasse, gilt
\[
 \bigcup_{n \in \N} A_n = \bigcup_{n \in \N} B_n \in \A.
\]
Dies zeigt die $\sigma$-Additivität.

Ist andererseits $\A$ eine $\sigma$-Algebra, so ist $\A$ auch eine Algebra, da jede $\sigma$-Algebra auf $X$ eine Algebra auf $X$ ist (bekannt aus der Vorlesung).


\subsection{}
Wie aus der Vorlesung bekannt ist jede $\sigma$-Algebra eine monotone Klasse. Damit ist $\sigma(\A)$ eine monotone Klasse, die $\A$ enthält. Da $m(\A)$ die kleinste monotone Klasse ist, die $\A$ enthält, ist daher $m(\A) \subseteq \sigma(\A)$.

Um zu zeigen, dass auch $m(\A) \subseteq \sigma(\A)$, genügt es zu zeigen, dass $m(\A)$ eine $\sigma$-Algebra ist: Da $\sigma(\A)$ die kleinste $\sigma$-Algebra ist, die $\A$ enthält, ist dann $m(\A) \subseteq \sigma(\A)$. Da $m(\A)$ eine monotone Klasse ist, genügt es nach \textbf{Aufgabenteil a)} zu zeigen, dass $m(\A)$ eine Algebra ist.

Es ist $\Omega \in \A \subseteq m(\A)$, da $\A$ eine Algebra ist. Es muss noch die Abgeschlossenheit von $\A$ unter Komplementbildung und endlichen Vereinigungen gezeigt werden.

Es sei $\mc{K} := \{M \in \mc{P}(X) : M^c \in m(\A)\}$. $\mc{K}$ ist eine monotone Klasse: Ist $(K_n)_{n \in \N}$ ein wachsende Folge in $\mc{K}$, so ist $(K_n^c)_{n \in \N}$ eine fallende Folge in $m(\A)$, und somit
\[
 \left( \bigcup_{n \in \N} K_n \right)^c
 = \bigcap_{n \in \N} K_n^c \in m(\A),
\]
da $m(\A)$ eine monotone Klasse ist, also $\bigcup_{n \in \N} K_n \in \mc{K}$.
Ist $(L_n)_{n \in \N}$ eine fallende Folge in $\mc{K}$, so ist $(L_n^c)_{n \in \N}$ eine wachsende Folge in $m(\A)$, und somit
\[
 \left( \bigcap_{n \in \N} L_n \right)^c = \bigcup_{n \in \N} L_n^c \in m(\A),
\]
da $m(\A)$ eine monotone Klasse ist, also $\bigcap_{n \in \N} L_n \in \mc{K}$. Dies zeigt, dass $\mc{K}$ eine monotone Klasse ist.

Da $\A$ eine Algebra ist, ist $A^c \in \A \subseteq m(\A)$ für alle $A \in \A$, also $\A \subseteq \mc{K}$. Da $\mc{K}$ eine monotone Klasse ist, die $\A$ enthält, ist daher $m(\A) \subseteq \mc{K}$. Also ist $A^c \in m(\A)$ für alle $A \in m(\A)$. Dies zeigt die Abgeschlossenheit von $\mc{A}$ unter Komplementbildung.

Für $D \in m(\A)$ sei $\mc{V}_D := \{M \in P(X) : D \cup M \in m(\A)\}$. Auch $\mc{V}_D$ ist eine monotone Klasse: Ist $(V_n)_{n \in \N}$ eine wachsende Folge auf $\mc{V}_D$, so ist $(V_n \cup D)_{n \in \N}$ ein wachsende Folge auf $m(\A)$, und somit
\[
 \left( \bigcup_{n \in \N} V_n \right) \cup D
 = \bigcup_{n \in \N} (V_n \cup D)
 \in m(\A),
\]
da $m(\A)$ eine monotone Klasse ist, und daher $\bigcup_{n \in \N} V_n \in \mc{V}_D$.
Ist $(W_n)_{n \in \N}$ eine fallende Folge auf $\mc{V}_D$, so $(W_n \cup D)$ eine fallende Folge auf $m(\A)$, und somit
\[
 \left( \bigcap_{n \in \N} W_n \right) \cup D
 = \bigcap_{n \in \N} (W_n \cup D)
 \in m(\A),
\]
da $m(\A)$ eine monotone Klasse ist, und daher $\bigcap_{n \in \N} W_n \in \mc{V}_D$. Dies zeigt, dass $\mc{V}_D$ eine monotone Klasse ist.

Sei $A \in \A$. Für $B \in \A$ ist $A \cup B \in \A \subseteq m(\A)$, da $\A$ eine Algebra ist, und daher $B \in \mc{V}_A$, wegen der Beliebigkeit von $B$ also $\A \subseteq \mc{V}_A$. Da $m(\A)$ die kleinste monotone Klasse ist, die $\A$ enthält, ist daher $m(\A) \subseteq \mc{V}_A$. Also ist $A \cup B \in m(\A)$ für alle $B \in m(\A)$ und $A \in \A$. Dies bedeutet auch, dass $\A \subseteq \mc{V}_B$ für alle $B \in m(\A)$. Da $m(\A)$ die kleinste monotone Klasse ist, die $\A$ enthält, ist daher $m(\A) \subseteq \mc{V}_B$ für alle $B \in m(\A)$. Das bedeutet gerade, dass $A \cup B \in m(\A)$ für alle $A, B \in m(\A)$. Dies zeigt die Abgeschlossenheit bezüglich endlicher Schnitte.
















\end{document}
