\documentclass[a4paper,10pt]{article}
%\documentclass[a4paper,10pt]{scrartcl}

\usepackage{xltxtra}
\usepackage[ngerman]{babel}
\usepackage{amsmath}
\usepackage{amssymb}
\usepackage{amsthm}
\usepackage{mathtools}
\usepackage{nicefrac}
\usepackage{enumerate}
\usepackage{leftidx}

\makeatletter
\g@addto@macro\th@definition{\thm@headpunct{:}}
\makeatother
\newcounter{saetze}
\newtheorem{lem}[saetze]{Lemma}
\newtheorem{beh}[saetze]{Behauptung}
\newtheorem{bem}[saetze]{Bemerkung}
\newtheorem*{ia}{Induktionsanfang}
\newtheorem*{is}{Induktionsschritt}

\renewcommand{\thesection}{Aufgabe \arabic{section}.}
\renewcommand{\thesubsection}{\alph{subsection})}
\renewcommand{\thesubsubsection}{(\roman{subsubsection})}

\makeatletter
\def\moverlay{\mathpalette\mov@rlay}
\def\mov@rlay#1#2{\leavevmode\vtop{%
   \baselineskip\z@skip \lineskiplimit-\maxdimen
   \ialign{\hfil$\m@th#1##$\hfil\cr#2\crcr}}}
\newcommand{\charfusion}[3][\mathord]{
    #1{\ifx#1\mathop\vphantom{#2}\fi
        \mathpalette\mov@rlay{#2\cr#3}
      }
    \ifx#1\mathop\expandafter\displaylimits\fi}
\makeatother

\newcommand{\N}{\mathbb{N}}
\newcommand{\Z}{\mathbb{Z}}
\newcommand{\Q}{\mathbb{Q}}
\newcommand{\R}{\mathbb{R}}
\newcommand{\C}{\mathbb{C}}
\newcommand{\A}{\mathcal{A}}
\newcommand{\La}{\mathcal{L}}
\newcommand{\dx}{\,\text{d}x}
\newcommand{\dy}{\,\text{d}y}
\newcommand{\dt}{\,\text{d}t}
\newcommand{\du}{\,\text{d}u}
\newcommand{\Img}{\operatorname{Im}}
\newcommand{\Real}{\operatorname{Re}}
\newcommand{\Imag}{\operatorname{Im}}
\newcommand{\sgn}{\operatorname{sgn}}
\newcommand{\Vol}{\operatorname{Vol}}
\newcommand{\dotcup}{\ensuremath{\mathaccent\cdot\cup}}
\newcommand{\bigdotcup}{\charfusion[\mathop]{\bigcup}{\cdot}}
\newcommand{\ceil}[1]{\left\lceil{#1}\right\rceil}
\newcommand{\floor}[1]{\left\lfloor{#1}\right\rfloor}
\newcommand{\mc}[1]{\mathcal{#1}}
\newcommand{\limes}[2]{\lim_{#1 \rightarrow #2}}
\newcommand{\limessup}[1]{\limsup_{#1 \rightarrow \infty}}
\newcommand{\limesinf}[1]{\liminf_{#1 \rightarrow \infty}}
\newcommand{\vect}[1]{\begin{pmatrix}#1\end{pmatrix}}
\newcommand{\partd}[2]{\frac{\partial #1}{\partial #2}}
\newcommand{\op}[1]{\left\|#1\right\|_{\text{op}}}

\makeatletter
\renewcommand*\env@matrix[1][*\c@MaxMatrixCols c]{%
  \hskip -\arraycolsep
  \let\@ifnextchar\new@ifnextchar
  \array{#1}}
\makeatother

\setromanfont[Mapping=tex-text]{Linux Libertine O}
% \setsansfont[Mapping=tex-text]{DejaVu Sans}
% \setmonofont[Mapping=tex-text]{DejaVu Sans Mono}
\parindent 0pt

\title{\sc Analysis III \\ \Large 6. Aufgabenblatt}
\author{Jendrik Stelzner}
\date{\today}

\begin{document}
\maketitle





\section{Lipschitz-Stetigkeit und ein Lemma von Sard}


\addtocounter{subsection}{2}
\subsection*{a) und b)}
Für alle $a = \ltrans{(a_1, \ldots, a_n)} \in \R^n$ und $l \geq 0$ bezeichne
\[
 W_l(a) = \prod_{i=1}^n \left[a_i-\frac{l}{2}, a_i+\frac{l}{2}\right)
\]
den halboffenen Würfel mit Seitenlänge $l$ und Mittelpunkt $a$. Dieser ist bekanntermaßen Borell-, und damit auch Lebesgue-meßbar mit
\[
 \lambda_n( W_l(a) ) = l^n.
\]
Da jeder halboffene Würfel von dieser Form ist, gelten alle Aussagen, die im Folgenden für Würfel der form $W_l(a)$ gezeigt werden, für beliebige halboffene Würfel.

Zunächst zeigen wir, dass für alle $l \geq 0$ und $a \in \R$
\begin{equation}\label{eq: abschätzung würfel äußerem maß}
 \lambda^*_n( f(W_l(a)) ) \leq n^{n/2} L^n l^n = n^{n/2} L^n \lambda_n( W_l(a) ).
\end{equation}
Hierfür bemerken wir zunächst, dass die Hauptdiagonale von $W_l(a)$ genau $\sqrt{n l^2} = \sqrt{n}\, l$ lang ist, und daher
\[
 W_l(a) \subseteq B_{\sqrt{n}l/2}(a).
\]
Wegen der Libschitz-Stetigkeit von $f$ ist damit
\[
 f( W_l(a) )
 \subseteq f( B_{\sqrt{n}l/2}(a) )
 \subseteq B_{\sqrt{n}Ll/2}(f(a))
 \subseteq W_{\sqrt{n}Ll}(f(a)).
\]
Aus der Monotonie von $\lambda^*_n$ folgt damit
\[
 \lambda^*_n( f(W_l(a)) )
 \leq \lambda^*_n\left(W_{\sqrt{n}Ll}(f(a))\right)
 = n^{n/2} L^n l^n,
\]
was \eqref{eq: abschätzung würfel äußerem maß} zeigt.

Wir zeigen nun, dass das Bild von Lebesgue-messbaren Mengen unter $f$ Lebesgue-messbar ist. Sei hierfür $A \subseteq \R^n$ eine Lebesgue-messbare Menge. Wie aus einem früheren Übungszettel bekannt, ist die Lebesgue-Messbarkeit von $A$ äuqivalent dazu, dass es Mengen $F \in F_\sigma$ und $N \subseteq \R^n$ mit $A = F \cup N$ gibt, so dass $\lambda_n(N) = 0$, also $N$ eine Lebesgue-Nullmenge ist (diese ist mit Sicherheit Lebesgue-messbar, da $\lambda_n$ vollständig ist). Inbesondere ist
\[
 f(A) = f(F \cup N) = f(F) \cup f(N).
\]

\begin{beh}
 Es ist $f(F) \in F_\sigma$ sowie $\lambda^*_n(f(N)) = 0$. Insbesondere ist $f(N)$ wegen der Vollständigkeit des Lebesgue-Maßes $\lambda_n$ also Lebesgue-Nullmenge messbar.
\end{beh}
\begin{proof}
 Zunächst zeigen wir, dass $f(F) \in F_\sigma$. Da $F \in F_\sigma$ gibt es eine Folge $(C_k)_{k \in \N}$ abgeschlossener Mengen, so dass $F = \bigcup_{k \in \N} C_k$. Für alle $k \in \N$ definieren wir eine Folge $(K^k_l)_{l \in \N}$ kompakter Mengen durch
 \[
  K^k_l := C_k \cap \overline{B_l(0)}.
 \]
 Es ist $C_k = \bigcup_{l \in \N} K^k_l$ und damit $F = \bigcup_{k,l \in \N} K^k_l$. Da $f$ stetig ist, ist das Bild kompakter Mengen unter $f$ wieder kompakt, und damit auch abgeschlossen. Insbesondere ist $f\left(K^k_l\right)$ für alle $k,l \in \N$ abgeschlossen. Es ist daher
 \[
  f(F)
  = f\left( \bigcup_{k,l \in \N} K^k_l \right)
  = \bigcup_{k,l \in \N} f\left(K^k_l\right) \in F_\sigma.
 \]
 Nun zeigen wir, dass $\lambda^*_n( f(N) )$. Sei hierfür $\varepsilon > 0$ beliebig aber fest. Da $N$ eine Lebesgue-Nullmenge ist, gibt es, wie aus einem dem früheren Übungsblätter bekannt, eine Folge $(Q_k)_{k \in \N}$ von halboffenen Würfeln in $\R^n$ so dass $N \subseteq \bigcup_{k \in \N} Q_k$ und $\sum_{k \in \N} \Vol(Q_k) \leq \varepsilon$. Es ist also auch
 \[
  f(N)
  \subseteq f\left( \bigcup_{k \in \N} Q_k \right)
  = \bigcup_{k \in \N} f(Q_k).
 \]
 Aus der Monotonie und Subadditivität von $\lambda^*_n$, sowie \eqref{eq: abschätzung würfel äußerem maß} folgt damit, dass
 \begin{align*}
  \lambda^*_n( f(N) )
  &\leq \lambda^*_n\left( \bigcup_{k \in \N} f(Q_k) \right)
  \leq \sum_{k \in \N} \mu(f(Q_k)) \\
  &\leq \sum_{k \in \N} n^{n/2} L^n \mu(Q_k)
  = n^{n/2} L^n \sum_{k \in \N} \mu(Q_k)
  \leq n^{n/2} L^n \varepsilon.
 \end{align*}
 Mit der Beliebigkeit von $\varepsilon > 0$ ergibt dies, dass $\lambda^*_n(f(N)) = 0$.
\end{proof}

Aus dieser Behauptung folgt aufgrund der bereits oben genannten, zur Lebesgue-Messbarkeit äquivalenten, Bedingung nun direkt, dass $f(A) = f(F) \cup f(N)$ wieder Lebesgue-messbar ist.

Insbesondere ergibt sich damit, dass die in \eqref{eq: abschätzung würfel äußerem maß} für das äußere Maß aufgestellte Ungleichung bereits für das Lebesgue-Maß selbst gilt, dass also
\begin{equation}\label{eq: abschätzung würfel maß}
 \lambda_n( f(W_l(a)) ) \leq n^{n/2} L^n l^n = n^{n/2} L^n \lambda_n( W_l(a) ).
\end{equation}

Die Abschätzung \eqref{eq: abschätzung würfel maß} lässt sich nun direkt auf beliebige offene Mengen übertragen: Sei $U \subseteq \R^n$ eine offene Menge. Wie aus der Vorlesung bekannt, gibt es eine Folge $(Q_k)_{k \in \N}$ paarweise disjunkter, halboffener Würfel mit
\[
 U = \bigdotcup_{k \in \N} Q_k.
\]
Insbesondere sind $U$, sowie alle $Q_k$ Lebesgue-messbar. Aufgrund der $\sigma$-Additivität von $\lambda_n$ ergibt nun aus \eqref{eq: abschätzung würfel maß}, dass
\begin{align*}
 \lambda_n( f(U) )
 &= \lambda_n\left( f\left( \bigdotcup_{k \in \N} Q_k \right) \right)
 = \lambda_n\left( \bigcup_{k \in \N} f(Q_k) \right) \\
 &\leq \sum_{k \in \N} \lambda_n(f(Q_k))
 \leq n^{n/2} L^n \sum_{k \in \N} \lambda_n(Q_k) \\
 &= n^{n/2} L^n \lambda_n \left( \bigdotcup_{k \in \N} Q_k \right)
 = n^{n/2} L^n \lambda_n(U).
\end{align*}
Es ist also auch für alle offenen Mengen $U \subseteq \R^n$
\begin{equation}\label{eq: abschätzung offene menge}
 \lambda_n(f(U)) \leq n^{n/2} L^n \lambda_n(U).
\end{equation}

Nun werden wir die Regularität des Lebesgue-Maßes nutzen, um diese Abschätzung für alle Lebesgue-messbaren Mengen zu zeigen. Sei hierfür $A \subseteq \R^n$ eine Lebesgue-messbare Menge und $\varepsilon > 0$ beliebig aber fest. Aus der Regularität des Lebesgue-Maßes folgt, dass es eine offene Menge $U \subseteq \R^n$ mit $A \subseteq U$ und $\lambda_n(U) \leq \lambda_n(A) + \varepsilon$ gibt. Es ist also wegen der Monotonie des Lebesgue-Maßes und \eqref{eq: abschätzung offene menge}
\begin{align*}
 \lambda_n(f(A))
 &\leq \lambda_n(f(U))
 \leq n^{n/2} L^n \lambda_n(U) \\
 &\leq n^{n/2} L^n (\lambda_n(A) + \varepsilon)
 = n^{n/2} L^n \lambda_n(A) + n^{n/2} L^n \varepsilon.
\end{align*}
Aus der Beliebigkeit von $\varepsilon > 0$ folgt damit die Ungleichung
\[
 \lambda_n(f(A)) \leq n^{n/2} L^n \lambda_n(A).
\]


\subsection{}

\begin{lem}\label{lem: Lipschitz Ableitung}
 Seien $a,b \in \R$ mit $a < b$ und sei $f$ stetig auf $[a,b]$ und differenzierbar auf $(a,b)$. Dann ist $f$ auf $[a,b]$ genau dann Lipschitz-stetig mit Konstante $L$, wenn $|f'(x)| \leq L$ für alle $x \in (a,b)$.
\end{lem}
\begin{proof}
 Ist $f$ Lipschitz-stetig mit Konstante $L$, so folgt wegen der Stetigkeits des Betrags, dass für alle $x \in (a,b)$
 \[
  |f'(x)|
  = \left|\limes{h}{0} \frac{f(x+h)-f(x)}{h}\right|
  = \limes{h}{0} \frac{|f(x+h)-f(x)|}{|h|}
  \leq \limes{h}{0} \frac{L |h|}{|h|}
  = L.
 \]
 Ist andererseits $f'(x) \leq L$ für alle $x \in (a,b)$, so folgt aus der Mittelwertabschätzung direkt, dass für alle $x,y \in (a,b)$
 \[
  |f(x)-f(y)| \leq L |x-y|.
 \]
\end{proof}

\begin{lem}\label{lem: abzählbare Zusammenhangskomponenten}
 Sei $U \subseteq \R$ offen. Dann gibt es eine höchstens abzählbare Familie $I_k$ von offenen, nichtleeren Intervallen, so dass $U = \bigdotcup_n I_k$.
\end{lem}
\begin{proof}
 Aus Analysis 2 bekannt, lässt sich eine offene Menge $V \subseteq \R^n$ in offene, nichtleere, disjunkte Zusammenhangskomponenten zerlegen lässt. Im Falle von $\R$ sind dies gerade offene Intervalle. Es muss also nur noch gezeigt werden, dass hierfür höchstens abzählbar viele Intervalle benötigt werden.
 
 Wir bemerken, dass es für jedes $I_k$ eine rationale Zahl $r_k \in I_k$ gibt. Aus der Disjunktheit der $I_k$ folgt direkt, dass die Zuordnung $I_k \mapsto r_k$ injektiv ist. Da es nur abzählbar viele $r_k$ gibt, gibt es damit auch nur abzählbar viele $I_k$.
\end{proof}

Es sei
\[
 N := \{x \in \R : f'(x) = 0\}
\]
und für $a,b \in \R$
\[
 N_{a,b} := \{x \in (a,b) : f'(x) = 0\}.
\]
Um zu zeigen, dass $f(N)$ Lebesgue-messbar mit $\lambda_n( f(N) ) = 0$ ist, genügt es zu zeigen, dass $f(N_{a,b})$ für alle $a,b \in \R$ Lebesgue-messbar mit Maß $\lambda_n( f(N_{a,b}) ) = 0$ ist. Dann ergibt sich, dass
\[
  f(N)
  = f\left(\bigcup_{k \in \Z} N_{k/2, k/2 + 1} \right)
   = \bigcup_{k \in \Z} f\left(N_{k/2, k/2 + 1}\right)
\]
als abzählbare Vereinigung von Lebesgue-messbaren Mengen Lebesgue-messbar ist, und
\[
 \lambda_n( f(N) )
 = \lambda_n\left( \bigcup_{k \in \Z} f\left(N_{k/2, k/2 + 1}\right) \right)
 \leq \sum_{k \in \Z} \lambda_n \left( f\left(N_{k/2, k/2 + 1}\right) \right)
 = 0.
\]

Wir zeigen nun die entsprechende Aussage für die $N_{a,b}$. Seien hierfür $a,b \in \R$ beliebig aber fest. Wir betrachten nur den Fall $a < b$, da sonst $N_{a,b} = \emptyset$ und die Aussage trivialerweise stimmt.

Zunächst zeigen wir, dass $f(N_{a,b})$ Lebesgue-messbar ist. Hierfür bemerken wir zum einen, dass $\tilde{f} := f_{|(a,b)}$ stetig differenzierbar ist, also $\tilde{f}' = f'_{|(a,b)}$ als stetige Funktion Borell-messbar. Daher ist
\[
 N_{a,b} = \{x \in (a,b) : \tilde{f}'(x) = 0\}
\]
Borell-, und damit auch Lebesgue-messbar. Zum anderen bemerken wir, dass die Einschränkung $\tilde{f}$ Lipschitz-stetig ist: Da $f'$ auf $[a,b]$ stetig ist, ist $f'$ auf $[a,b]$, und somit auch auf $(a,b)$, beschränkt. Aus Lemma \ref{lem: Lipschitz Ableitung} folgt damit, $f$ auf $[a,b]$ Lipschitz-stetig ist. Damit ist auch $\tilde{f}$ auf $(a,b)$ Lipschitz-stetig. Aus den vorherigen Aufgabenteilen folgt damit, dass $f(N_{a,b})$ als Bild einer Lebesgue-messbaren Menge unter einer Lipschitz-stetigen Funktion ebenfalls Lebesgue-messbar ist.

Nun zeigen wir, dass $\lambda_n(f(N_{a,b})) = 0$. Sei hierfür $\varepsilon > 0$ beliebig aber fest. Wegen der Stetigkeit von $\tilde{f}'$ gibt es für alle $x \in N_{a,b}$ ein $\delta_x > 0$, so dass $|\tilde{f}'(y)| < \varepsilon$ für alle $y \in B_{\delta_x}(x)$. Da $(a,b)$ offen ist, kann dabei $\delta_x > 0$ so klein gewählt werden, dass $B_{\delta_x}(x) \subseteq (a,b)$. Es sei 
\[
 U := \bigcup_{x \in N_{a,b}} B_{\delta_x}(x).
\]
$U$ ist als Vereinigung offener Mengen offen, und es ist
\[
 N_{a,b} \subseteq U \subseteq (a,b).
\]
Es seien $U_k = (a_k, b_k)$ die Zusammenhangskomponenten von $U$. Nach Lemma \ref{lem: abzählbare Zusammenhangskomponenten} sind diese höchstens abzählbar viele. Für alle $k$ ist $U_k$ als offenes Intervall Borell- und damit auch Lebesgue-messbar, wegen der Lipschitz-Stetigkeit von $\tilde{f}$ ist also auch $\tilde{f}(U_k)$ für alle $k$ Lebesgue-messbar. Dabei folgt aus der Konstruktion von $U$, dass $\tilde{f}'(x) < \varepsilon$ für alle $x \in U \supseteq U_k$, aus Lemma \ref{lem: Lipschitz Ableitung} also, dass $\tilde{f}_{|U_k}$ für alle $k$ Lipschitz-stetig mit Konstante $\varepsilon$ ist. Aus den vorherigen Aufgabenteilen folgt damit, dass für alle $k$
\[
 \lambda_n(f(U_k)) \leq n^{n/2} \varepsilon^n \lambda_n(U_k).
\]
Wegen der Monotonie und $\sigma$-Additivität von $\lambda_n$ ergibt sich nun, dass
\begin{align*}
 \lambda_n( f(N_{a,b}) )
 &\leq \lambda_n( f(U) )
 = \lambda_n\left( f\left(\bigdotcup_k U_k\right) \right) 
 = \sum_k \lambda_n( f(U_k) ) \\
 &\leq n^{n/2} L^n \sum_k \lambda_n(U_k)
 = n^{n/2} \varepsilon^n \lambda_n \left( \bigdotcup_k U_k \right) \\
 &= n^{n/2} \varepsilon^n \lambda_n(U)
 \leq n^{n/2} \varepsilon^n \lambda_n((a,b))
 = n^{n/2} \varepsilon^n (b-a).
\end{align*}
Aus der Beliebigkeit von $\varepsilon > 0$ folgt damit, dass
\[
 \lambda_n( f(N_{a,b} ) = 0.
\]





\section{Funktionen als Maße}





\section{Positive Funktionen}





\section{Konvergenz von Funktionen}


\subsection{}
Nach Definition ist $f_n \rightarrow f$ fast überall genau dann, wenn die Menge $N$ der Stellen, an denen $f_n$ nicht punktweise gegen $f$ konvergiert, eine $\mu$-Nullmenge ist. Für alle $\varepsilon > 0$ sei
\begin{align*}
 N_\varepsilon
 &:= \{x \in X : \text{ für alle } n \geq \N \text{ gibt es } m \geq n : |f_m(x)-f(x)| \geq \varepsilon \} \\
 &= \bigcap_{n \in \N} \bigcup_{m \geq n} \{x: |f_m(x) - f(x)| \geq \varepsilon \}.
\end{align*}
Wir bemerken, dass $N_\varepsilon \in \A$ für alle $\varepsilon > 0$: Aus der Messbarkeit von $f$ und der $f_m$ für alle $m \in \N$ folgt, dass auch $|f_m-f|$ für alle $m \in \N$ messbar ist. Daher ist
\[
 \{x : |f_m(x) - f(x)| \geq \varepsilon \} \in \A
\]
für alle $m \in \N$ und $\varepsilon > 0$. Die Messbarkeit von $N_\varepsilon$ ergibt sich daher aus der Abgeschlossenheit von $\A$ bezüglich abzählbarer Schnitte und Vereinigungen.

$N$ lässt sich nun schreiben als
\[
 N = \bigcup_{\varepsilon > 0} N_\varepsilon = \bigcup_{\substack{q \in \Q \\ q > 0}} N_q.
\]
Da $N_\varepsilon \in \A$ für alle $\varepsilon > 0$ ist auch $N$ als abzählbare Vereinigung der $N_q$ in $\A$ enthalten.

Es ist nun einfach zu sehen, dass $N$ genau dann eine $\mu$-Nullmenge ist, wenn jedes der $N_\varepsilon$ eine $\mu$-Nullmenge ist: Dies ergibt sich direkt aus der Monotonie von $\mu$ und daraus, dass für alle $\varepsilon > 0$
\[
 N_\varepsilon \subseteq N = \bigcup_{\substack{q \in \Q \\ q > 0}} N_q.
\]


\subsection{}
Wie bereits im vorherigen Aufgabenteil erläutert ist für alle $m \in \N$ und $\varepsilon > 0$
\[
 \{x : |f_m(x) - f(x)| \geq \varepsilon \} \in \A.
\]
Wegen der Abgeschlossenheit von $\A$ bezüglich abzählbarer Vereinigungen ist daher für alle $\varepsilon > 0$
\[
 \left( \bigcup_{m \geq n} \{x : |f_m(x) - f(x)| \geq \varepsilon \}\right)_{n \in \N}
\]
eine monoton fallende Folge auf $\A$. Ist $\mu$ ein endlich Maß, so ergibt sich damit, dass
\begin{align*}
 &\mu\left( \bigcap_{n \in \N} \bigcup_{m \geq n} \{x: |f_m(x) - f(x)| \geq \varepsilon \} \right) \\
 = \limes{n}{\infty} &\mu\left( \bigcup_{m \geq n} \{x: |f_m(x) - f(x)| \geq \varepsilon \} \right).
\end{align*}
Die zu zeigend Aussage ergibt sich daher direkt aus dem vorherigen Aufgabenteil.


\subsection{}
Es sei $f := 0$, und für alle $n \in \N$ sei $f_n := \chi_{[n,n+1]}$. Dann ist $\limes{n}{\infty} f_n(x) = f(x)$ für alle $x \in \R$, d.h. es ist $f_n \rightarrow f$ punktweise auf ganz $\R$. Für beliebiges $0 < \varepsilon \leq 1$ ist dann für alle $n \in \N$
\[
  \bigcup_{m \geq n} \{x : |f_m(x) - f(x)| \geq \varepsilon \} = [n,\infty),
\]
und damit insbesondere
\[
 \limes{n}{\infty} \lambda_1\left( \bigcup_{m \geq n} \{x : |f_m(x) - f(x)| \geq \varepsilon \} \right)
 = \limes{n}{\infty} \lambda_1([n,\infty))
 = \infty.
\]


\subsection{}
Zunächst nehmen wir an, dass $f_n \rightarrow f$ fast überall. Für alle $n \in \N$ und $\delta > 0$ sei
\[
 A_{n,\delta} := \bigcup_{m \geq n} \{x : |f_m(x) - f(x)| \geq \delta \}.
\]
Man bemerke, dass $A_{n,\delta} \in \A$ für alle $n \in \N$ und $\delta > 0$, und dass
\begin{equation}\label{eq: A^c gleichmäßig}
 A^c_{n,\delta} = \{x : |f_m(x) - f(x)| < \delta \text{ für alle } m \geq n\}.
\end{equation}

Sei nun $\varepsilon > 0$ beliebig aber fest. Aus Aufgabenteil \textbf{b)} folgt, dass für alle $\delta > 0$
\[
 \limes{n}{\infty} \mu(A_{n,\delta}) = 0.
\]
Inbesondere gibt es für alle $k \in \N$ ein $N_k \in \N$ so dass
\[
 \mu(A_{N_k, \varepsilon/2^{k+1}}) < \frac{\varepsilon}{2^{k+1}}.
\]
Es sei
\[
 A := \bigcup_{k \in \N} A_{N_k, \varepsilon/2^{k+1}}.
\]

Wir bemerken nun, dass $A \in \A$, da $\A$ unter abzählbaren Vereinigungen abgeschlossen ist, und das wegen der Monotonie von $\mu$
\[
 \mu(A)
 = \left( \bigcup_{k \in \N} A_{N_k, \varepsilon/2^{k+1}} \right)
 \leq \sum_{k \in \N} \mu(A_{N_k, \varepsilon/2^{k+1}})
 < \sum_{k \in \N} \frac{\varepsilon}{2^{k+1}}
 = \varepsilon.
\]
Andererseits ist
\[
 X \setminus A
 = A^c
 = \left( \bigcup_{k \in \N} A_{N_k, \varepsilon/2^{k+1}} \right)^c
 = \bigcap_{k \in \N} A^c_{N_k, \varepsilon/2^{k+1}}.
\]
Es folgt daher aus \eqref{eq: A^c gleichmäßig}, dass für alle $k \in \N$ und $m \geq N_k$
\[
 |f_m(x) - f(x)| < \frac{\varepsilon}{2^{k+1}} \text{ für alle } x \in X \setminus A.
\]
Dies zeigt, dass $f_n \rightarrow f$ gleichmäßig auf $X \setminus A$.

Nun nehmen wir an, dass es für alle $\varepsilon > 0$ eine Menge $A_\varepsilon \in \A$ gibt, so dass $\mu(A_\varepsilon) < \varepsilon$ und $f_n \rightarrow f$ gleichmäßig auf $A^c_\varepsilon$. Wie zuvor bezeichne $N$ die Menge aller Stellen, an denen nicht $f_n \rightarrow f$ punktweise.
Da $(f_n)_{n \in \N}$ für alle $\varepsilon > 0$ auf $A^c_\varepsilon$ gleichmäßig, und damit insbesondere punktweise gegen $f$ konvergiert, muss $N \subseteq A_\varepsilon$ für alle $\varepsilon > 0$. Also ist
\[
 N \subseteq \bigcap_{\substack{q \in \Q \\ q > 0}} A_q =: B.
\]
Dabei ist $B \in \A$, da $\A$ unter abzählbaren Schnitten abgeschlossen ist. Aus der Monotonie von $\mu$ ergibt sich auch, dass
\[
 \mu(B)
 = \mu\left( \bigcap_{\substack{q \in \Q \\ q > 0}} A_q \right)
 \leq \mu(A_q) = q
\]
für alle $q \in \Q$ mit $q > 0$. Also ist $\mu(B) = 0$. Dies zeigt, dass $N$ vernachlässigbar ist, also $f_n \rightarrow f$ punktweise fast überall.











































\end{document}
