\documentclass[a4paper,10pt]{article}
%\documentclass[a4paper,10pt]{scrartcl}

\usepackage{xltxtra}
\usepackage[ngerman]{babel}
\usepackage{amsmath}
\usepackage{amssymb}
\usepackage{amsthm}
\usepackage{mathtools}
\usepackage{nicefrac}
\usepackage{enumerate}
\usepackage{leftidx}

\makeatletter
\g@addto@macro\th@definition{\thm@headpunct{:}}
\makeatother
\theoremstyle{definition}
\newcounter{saetze}
\newtheorem{lem}[saetze]{Lemma}
\newtheorem{beh}[saetze]{Behauptung}
\newtheorem{bem}[saetze]{Bemerkung}
\newtheorem*{ia}{Induktionsanfang}
\newtheorem*{is}{Induktionsschritt}

\renewcommand{\thesection}{Aufgabe \arabic{section}.}
\renewcommand{\thesubsection}{\alph{subsection})}
\renewcommand{\thesubsubsection}{(\roman{subsubsection})}

\makeatletter
\def\moverlay{\mathpalette\mov@rlay}
\def\mov@rlay#1#2{\leavevmode\vtop{%
   \baselineskip\z@skip \lineskiplimit-\maxdimen
   \ialign{\hfil$\m@th#1##$\hfil\cr#2\crcr}}}
\newcommand{\charfusion}[3][\mathord]{
    #1{\ifx#1\mathop\vphantom{#2}\fi
        \mathpalette\mov@rlay{#2\cr#3}
      }
    \ifx#1\mathop\expandafter\displaylimits\fi}
\makeatother

\newcommand{\N}{\mathbb{N}}
\newcommand{\Z}{\mathbb{Z}}
\newcommand{\Q}{\mathbb{Q}}
\newcommand{\R}{\mathbb{R}}
\newcommand{\C}{\mathbb{C}}
\newcommand{\A}{\mathcal{A}}
\newcommand{\La}{\mathcal{L}}
\newcommand{\dx}{\,\text{d}x}
\newcommand{\dy}{\,\text{d}y}
\newcommand{\dt}{\,\text{d}t}
\newcommand{\du}{\,\text{d}u}
\newcommand{\Img}{\operatorname{Im}}
\newcommand{\Real}{\operatorname{Re}}
\newcommand{\Imag}{\operatorname{Im}}
\newcommand{\sgn}{\operatorname{sgn}}
\newcommand{\Vol}{\operatorname{Vol}}
\newcommand{\dotcup}{\ensuremath{\mathaccent\cdot\cup}}
\newcommand{\bigdotcup}{\charfusion[\mathop]{\bigcup}{\cdot}}
\newcommand{\ceil}[1]{\left\lceil{#1}\right\rceil}
\newcommand{\floor}[1]{\left\lfloor{#1}\right\rfloor}
\newcommand{\mc}[1]{\mathcal{#1}}
\newcommand{\limes}[2]{\lim_{#1 \rightarrow #2}}
\newcommand{\limessup}[1]{\limsup_{#1 \rightarrow \infty}}
\newcommand{\limesinf}[1]{\liminf_{#1 \rightarrow \infty}}
\newcommand{\vect}[1]{\begin{pmatrix}#1\end{pmatrix}}
\newcommand{\partd}[2]{\frac{\partial #1}{\partial #2}}
\newcommand{\op}[1]{\left\|#1\right\|_{\text{op}}}

\makeatletter
\renewcommand*\env@matrix[1][*\c@MaxMatrixCols c]{%
  \hskip -\arraycolsep
  \let\@ifnextchar\new@ifnextchar
  \array{#1}}
\makeatother

\setromanfont[Mapping=tex-text]{Linux Libertine O}
% \setsansfont[Mapping=tex-text]{DejaVu Sans}
% \setmonofont[Mapping=tex-text]{DejaVu Sans Mono}
\parindent 0pt

\title{\sc Analysis III \\ \Large 6. Aufgabenblatt}
\author{Jendrik Stelzner}
\date{\today}

\begin{document}
\maketitle





\section{Lipschitz-Stetigkeit und ein Lemma von Sard}


\addtocounter{subsection}{2}
\subsection*{a) und b)}
Für alle $a = \ltrans{(a_1, \ldots, a_n)} \in \R^n$ und $l \geq 0$ bezeichne
\[
 W_l(a) = \prod_{i=1}^n \left[a_i-\frac{l}{2}, a_i+\frac{l}{2}\right)
\]
den halboffenen Würfel mit Seitenlänge $l$ und Mittelpunkt $a$. Dieser ist bekanntermaßen Borell-, und damit auch Lebesgue-meßbar mit
\[
 \lambda_n( W_l(a) ) = l^n.
\]
Da jeder halboffene Würfel von dieser Form ist, gelten alle Aussagen, die im Folgenden für Würfel der form $W_l(a)$ gezeigt werden, für beliebige halboffene Würfel.

Zunächst zeigen wir, dass für alle $l \geq 0$ und $a \in \R$
\begin{equation}\label{eq: abschätzung würfel äußerem maß}
 \lambda^*_n( f(W_l(a)) ) \leq n^{n/2} L^n l^n = n^{n/2} L^n \lambda_n( W_l(a) ).
\end{equation}
Hierfür bemerken wir zunächst, dass die Hauptdiagonale von $W_l(a)$ genau $\sqrt{n l^2} = \sqrt{n}\, l$ lang ist, und daher
\[
 W_l(a) \subseteq B_{\sqrt{n}l/2}(a).
\]
Wegen der Libschitz-Stetigkeit von $f$ ist damit
\[
 f( W_l(a) )
 \subseteq f( B_{\sqrt{n}l/2}(a) )
 \subseteq B_{\sqrt{n}Ll/2}(f(a))
 \subseteq W_{\sqrt{n}Ll}(f(a)).
\]
Aus der Monotonie von $\lambda^*_n$ folgt damit
\[
 \lambda^*_n( f(W_l(a)) )
 \leq \lambda^*_n\left(W_{\sqrt{n}Ll}(f(a))\right)
 = n^{n/2} L^n l^n,
\]
was \eqref{eq: abschätzung würfel äußerem maß} zeigt.

Wir zeigen nun, dass das Bild von Lebesgue-messbaren Mengen unter $f$ Lebesgue-messbar ist. Sei hierfür $A \subseteq \R^n$ eine Lebesgue-messbare Menge. Wie aus einem früheren Übungszettel bekannt, ist die Lebesgue-Messbarkeit von $A$ äuqivalent dazu, dass es Mengen $F \in F_\sigma$ und $N \subseteq \R^n$ mit $A = F \cup N$ gibt, so dass $\lambda_n(N) = 0$, also $N$ eine Lebesgue-Nullmenge ist (diese ist mit Sicherheit Lebesgue-messbar, da $\lambda_n$ vollständig ist). Inbesondere ist
\[
 f(A) = f(F \cup N) = f(F) \cup f(N).
\]

\begin{beh}
 Es ist $f(F) \in F_\sigma$ sowie $\lambda^*_n(f(N)) = 0$. Insbesondere ist $f(N)$ wegen der Vollständigkeit des Lebesgue-Maßes $\lambda_n$ also Lebesgue-Nullmenge messbar.
\end{beh}
\begin{proof}
 Zunächst zeigen wir, dass $f(F) \in F_\sigma$. Da $F \in F_\sigma$ gibt es eine Folge $(C_k)_{k \in \N}$ abgeschlossener Mengen, so dass $F = \bigcup_{k \in \N} C_k$. Für alle $k \in \N$ definieren wir eine Folge $(K^k_l)_{l \in \N}$ kompakter Mengen durch
 \[
  K^k_l := C_k \cap \overline{B_l(0)}.
 \]
 Es ist $C_k = \bigcup_{l \in \N} K^k_l$ und damit $F = \bigcup_{k,l \in \N} K^k_l$. Da $f$ stetig ist, ist das Bild kompakter Mengen unter $f$ wieder kompakt, und damit auch abgeschlossen. Insbesondere ist $f\left(K^k_l\right)$ für alle $k,l \in \N$ abgeschlossen. Es ist daher
 \[
  f(F)
  = f\left( \bigcup_{k,l \in \N} K^k_l \right)
  = \bigcup_{k,l \in \N} f\left(K^k_l\right) \in F_\sigma.
 \]
 Nun zeigen wir, dass $\lambda^*_n( f(N) )$. Sei hierfür $\varepsilon > 0$ beliebig aber fest. Da $N$ eine Lebesgue-Nullmenge ist, gibt es, wie aus einem dem früheren Übungsblätter bekannt, eine Folge $(Q_k)_{k \in \N}$ von halboffenen Würfeln in $\R^n$ so dass $N \subseteq \bigcup_{k \in \N} Q_k$ und $\sum_{k \in \N} \Vol(Q_k) \leq \varepsilon$. Es ist also auch
 \[
  f(N)
  \subseteq f\left( \bigcup_{k \in \N} Q_k \right)
  = \bigcup_{k \in \N} f(Q_k).
 \]
 Aus der Monotonie und Subadditivität von $\lambda^*_n$, sowie \eqref{eq: abschätzung würfel äußerem maß} folgt damit, dass
 \begin{align*}
  \lambda^*_n( f(N) )
  &\leq \lambda^*_n\left( \bigcup_{k \in \N} f(Q_k) \right)
  \leq \sum_{k \in \N} \mu(f(Q_k)) \\
  &\leq \sum_{k \in \N} n^{n/2} L^n \mu(Q_k)
  = n^{n/2} L^n \sum_{k \in \N} \mu(Q_k)
  \leq n^{n/2} L^n \varepsilon.
 \end{align*}
 Mit der Beliebigkeit von $\varepsilon > 0$ ergibt dies, dass $\lambda^*_n(f(N)) = 0$.
\end{proof}

Aus dieser Behauptung folgt aufgrund der bereits oben genannten, zur Lebesgue-Messbarkeit äquivalenten, Bedingung nun direkt, dass $f(A) = f(F) \cup f(N)$ wieder Lebesgue-messbar ist.

Insbesondere ergibt sich damit, dass die in \eqref{eq: abschätzung würfel äußerem maß} für das äußere Maß aufgestellte Ungleichung bereits für das Lebesgue-Maß selbst gilt, dass also
\begin{equation}\label{eq: abschätzung würfel maß}
 \lambda_n( f(W_l(a)) ) \leq n^{n/2} L^n l^n = n^{n/2} L^n \lambda_n( W_l(a) ).
\end{equation}

Die Abschätzung \eqref{eq: abschätzung würfel maß} lässt sich nun direkt auf beliebige offene Mengen übertragen: Sei $U \subseteq \R^n$ eine offene Menge. Wie aus der Vorlesung bekannt, gibt es eine Folge $(Q_k)_{k \in \N}$ paarweise disjunkter, halboffener Würfel mit
\[
 U = \bigdotcup_{k \in \N} Q_k.
\]
Insbesondere sind $U$, sowie alle $Q_k$ Lebesgue-messbar. Aufgrund der $\sigma$-Additivität von $\lambda_n$ ergibt nun aus \eqref{eq: abschätzung würfel maß}, dass
\begin{align*}
 \lambda_n( f(U) )
 &= \lambda_n\left( f\left( \bigdotcup_{k \in \N} Q_k \right) \right)
 = \lambda_n\left( \bigcup_{k \in \N} f(Q_k) \right) \\
 &\leq \sum_{k \in \N} \lambda_n(f(Q_k))
 \leq n^{n/2} L^n \sum_{k \in \N} \lambda_n(Q_k) \\
 &= n^{n/2} L^n \lambda_n \left( \bigdotcup_{k \in \N} Q_k \right)
 = n^{n/2} L^n \lambda_n(U).
\end{align*}
Es ist also auch für alle offenen Mengen $U \subseteq \R^n$
\begin{equation}\label{eq: abschätzung offene menge}
 \lambda_n(f(U)) \leq n^{n/2} L^n \lambda_n(U).
\end{equation}

Nun werden wir die Regularität des Lebesgue-Maßes nutzen, um diese Abschätzung für alle Lebesgue-messbaren Mengen zu zeigen. Sei hierfür $A \subseteq \R^n$ eine Lebesgue-messbare Menge und $\varepsilon > 0$ beliebig aber fest. Aus der Regularität des Lebesgue-Maßes folgt, dass es eine offene Menge $U \subseteq \R^n$ mit $A \subseteq U$ und $\lambda_n(U) \leq \lambda_n(A) + \varepsilon$ gibt. Es ist also wegen der Monotonie des Lebesgue-Maßes und \eqref{eq: abschätzung offene menge}
\begin{align*}
 \lambda_n(f(A))
 &\leq \lambda_n(f(U))
 \leq n^{n/2} L^n \lambda_n(U) \\
 &\leq n^{n/2} L^n (\lambda_n(A) + \varepsilon)
 = n^{n/2} L^n \lambda_n(A) + n^{n/2} L^n \varepsilon.
\end{align*}
Aus der Beliebigkeit von $\varepsilon > 0$ folgt damit die Ungleichung
\[
 \lambda_n(f(A)) \leq n^{n/2} L^n \lambda_n(A).
\]


\subsection{}

\begin{lem}\label{lem: Lipschitz Ableitung}
 Seien $a,b \in \R$ mit $a < b$ und $f \in C^1((a,b),\R)$. Dann ist $f$ genau dann Lipschitz-stetig mit Konstante $L$, wenn $|f'(x)| \leq L$ für alle $x \in (a,b)$.
\end{lem}
\begin{proof}
 Ist $f$ Lipschitz-stetig mit konstante $L$, so folgt wegen der Stetigkeits des Betrags für alle $x \in (a,b)$, dass
 \[
  |f'(x)|
  = \left|\limes{h}{0} \frac{f(x+h)-f(x)}{h}\right|
  = \limes{h}{0} \frac{|f(x+h)-f(x)|}{|h|}
  \leq \limes{h}{0} \frac{L |h|}{|h|}
  = L.
 \]
 Ist andererseits $f'(x) \leq L$ für alle $x \in (a,b)$, so folgt mit der Mittelwertabschätzung direkt dass für alle $x,y \in (a,b)$
 \[
  |f(x)-f(y)| \leq L |x-y|.
 \]
\end{proof}

\begin{lem}\label{lem: abzählbare Zusammenhangskomponenten}
 Sei $U \subseteq \R$ offen. Dann gibt es eine höchstens abzählbare Familie $I_k$ von offenen, nichtleeren Intervallen, so dass $U = \bigdotcup_n I_k$.
\end{lem}
\begin{proof}
 Aus Analysis 2 bekannt, lässt sich eine offene Menge $V \subseteq \R^n$ in offene, nichtleere, disjunkte Zusammenhangskomponenten zerlegen lässt. Im Falle von $\R$ sind dies gerade offene Intervalle. Es muss also nur noch gezeigt werden, dass hierfür höchstens abzählbar viele Intervalle benötigt werden.
 
 Wir bemerken, dass es für jedes $I_k$ eine rationale Zahl $r_k \in I_k$ gibt. Aus der Disjunktheit der $I_k$ folgt direkt, dass die Zuordnung $I_k \mapsto r_k$ injektiv ist. Da es nur abzählbar viele $r_k$ gibt, gibt es damit auch nur abzählbar viele $I_k$.
\end{proof}





































\end{document}
