\documentclass[a4paper,10pt]{article}
%\documentclass[a4paper,10pt]{scrartcl}

\usepackage{xltxtra}
\usepackage[ngerman]{babel}
\usepackage{amsmath}
\usepackage{amssymb}
\usepackage{amsthm}
\usepackage{mathtools}
\usepackage{nicefrac}
\usepackage{enumerate}
\usepackage{leftidx}

\makeatletter
\g@addto@macro\th@definition{\thm@headpunct{:}}
\makeatother
\theoremstyle{definition}
\newtheorem{lem}{Lemma}
\newtheorem{beh}{Behauptung}
\newtheorem{bem}{Bemerkung}
\newtheorem*{ia}{Induktionsanfang}
\newtheorem*{is}{Induktionsschritt}

\renewcommand{\thesection}{Aufgabe \arabic{section}.}
\renewcommand{\thesubsection}{\alph{subsection})}
\renewcommand{\thesubsubsection}{(\roman{subsubsection})}

\makeatletter
\def\moverlay{\mathpalette\mov@rlay}
\def\mov@rlay#1#2{\leavevmode\vtop{%
   \baselineskip\z@skip \lineskiplimit-\maxdimen
   \ialign{\hfil$\m@th#1##$\hfil\cr#2\crcr}}}
\newcommand{\charfusion}[3][\mathord]{
    #1{\ifx#1\mathop\vphantom{#2}\fi
        \mathpalette\mov@rlay{#2\cr#3}
      }
    \ifx#1\mathop\expandafter\displaylimits\fi}
\makeatother

\newcommand{\N}{\mathbb{N}}
\newcommand{\Z}{\mathbb{Z}}
\newcommand{\Q}{\mathbb{Q}}
\newcommand{\R}{\mathbb{R}}
\newcommand{\C}{\mathbb{C}}
\newcommand{\A}{\mathcal{A}}
\newcommand{\La}{\mathcal{L}}
\newcommand{\dx}{\,\text{d}x}
\newcommand{\dy}{\,\text{d}y}
\newcommand{\dt}{\,\text{d}t}
\newcommand{\du}{\,\text{d}u}
\newcommand{\Img}{\operatorname{Im}}
\newcommand{\Real}{\operatorname{Re}}
\newcommand{\Imag}{\operatorname{Im}}
\newcommand{\sgn}{\operatorname{sgn}}
\newcommand{\dotcup}{\ensuremath{\mathaccent\cdot\cup}}
\newcommand{\bigdotcup}{\charfusion[\mathop]{\bigcup}{\cdot}}
\newcommand{\ceil}[1]{\left\lceil{#1}\right\rceil}
\newcommand{\floor}[1]{\left\lfloor{#1}\right\rfloor}
\newcommand{\mc}[1]{\mathcal{#1}}
\newcommand{\limes}[2]{\lim_{#1 \rightarrow #2}}
\newcommand{\limessup}[1]{\limsup_{#1 \rightarrow \infty}}
\newcommand{\limesinf}[1]{\liminf_{#1 \rightarrow \infty}}
\newcommand{\vect}[1]{\begin{pmatrix}#1\end{pmatrix}}
\newcommand{\partd}[2]{\frac{\partial #1}{\partial #2}}
\newcommand{\op}[1]{\left\|#1\right\|_{\text{op}}}

\makeatletter
\renewcommand*\env@matrix[1][*\c@MaxMatrixCols c]{%
  \hskip -\arraycolsep
  \let\@ifnextchar\new@ifnextchar
  \array{#1}}
\makeatother

\setromanfont[Mapping=tex-text]{Linux Libertine O}
% \setsansfont[Mapping=tex-text]{DejaVu Sans}
% \setmonofont[Mapping=tex-text]{DejaVu Sans Mono}
\parindent 0pt

\title{\sc Analysis III \\ \Large 4. Aufgabenblatt}
\author{Jendrik Stelzner}
\date{\today}

\begin{document}
\maketitle





\section{(Lebesgue-Stieltjes-Maße)}


\subsection{}
Da $\mu$ endlich und monoton ist, ist $\mu$ auch beschränkt, denn für alle $A \in \mc{B}(\R)$ ist
\[
 \mu(A) \leq \mu(\R) < \infty.
\]
Aus der Beschränktheit von $\mu$ folgt die Beschränktheit von $F_\mu$, da damit für alle $x \in \R$
\[
 F_\mu(x) = \mu((-\infty,x]) < \infty.
\]
Die Monotonie von $F_\mu$ ergibt sich daraus, dass für $x, y \in \R$ mit $x \leq y$ offenbar $(-\infty, x] \subseteq (-\infty, y]$ und wegen der Monotonie von $\mu$ daher
\[
 F_\mu(x) = \mu((-\infty,  x]) \leq \mu((-\infty, y]) = F_\mu(y).
\]

Zum Nachweis der Rechtsstetigkeit sei $(x_n)_{n \in \N}$ eine motonon fallende Folge auf $\R$ mit $\limes{n}{\infty} x_n = x \in \R$. Es gilt zu zeigen, dass $\limes{n}{\infty} F_\mu(x_n) = F_\mu(x)$. Aus der Monotonie von $F_\mu$ folgt direkt, dass $F_\mu(x) \leq F_\mu(x_n)$ für alle $n \in \N$, also auch
\begin{equation}\label{eq:grenzwertkleiner}
 F_\mu(x) \leq \limes{n}{\infty} F_\mu(x_n).
\end{equation}
Zum Beweis der umgekehrten Ungleichung sei $\varepsilon > 0$ beliebig aber fest. Wir bemerken, dass die Folge $( (x,x+\frac{1}{n}] )_{n \in \N}$ auf $\mc{B}(\R)$ fallend ist, und daher
\[
 \limes{n}{\infty} \mu\left( \left(x,x+\frac{1}{n}\right] \right)
 = \mu\left( \bigcap_{n \in \N} \left(x,x+\frac{1}{n}\right] \right)
 = \mu( \emptyset ) = 0,
\]
da $\mu$ ein Maß ist. Es gibt also ein $N \in \N$ mit
\[
 \mu\left( \left(x,x+\frac{1}{n}\right] \right) < \varepsilon \text{ für alle } n \geq N.
\]
Da $\limes{n}{\infty} x_n = x$ gibt es ein $n_0 \in \N$ mit $x \leq x_n \leq x+\frac{1}{N}$ für alle $n \geq n_0$. Aufgrund der Monotonie und endlichen Additivität von $\mu$ ergibt sich, dass für alle $n \geq n_0$
\begin{align*}
 F_\mu(x_n)
 &= \mu( (-\infty, x_n] )
 = \mu( (-\infty, x]\ \dotcup\ (x,x_n] )
 = \mu( (-\infty, x] ) + \mu( (x,x_n] ) \\
 &= F_\mu(x) + \mu( (x,x_n] )
 \leq F_\mu(x) + \mu\left( \left(x,x+\frac{1}{N}\right] \right)
 < F_\mu(x) + \varepsilon.
\end{align*}
Also ist auch
\[
 \limes{n}{\infty} F_\mu(x_n) \leq F_\mu(x) + \varepsilon.
\]
Aus der Beliebigkeit von $\varepsilon > 0$ folgt daraus, dass
\[
 \limes{n}{\infty} F_\mu(x_n) \leq F_\mu(x).
\]
Zusammen mit \eqref{eq:grenzwertkleiner} zeigt dies die Rechtsstetigkeit von $F_\mu$.

Dass $\limes{x}{\infty} F_\mu(x) = 0$ ergibt sich ähnlich: Die Folge $((-\infty,-n])_{n \in \N}$ auf $\mc{B}(\R)$ ist fallend, also gilt
\[
 \limes{n}{\infty} F_\mu(-n)
 = \limes{n}{\infty} \mu((-\infty,-n])
 = \mu\left( \bigcap_{n \in \N} (-\infty,-n] \right)
 = \mu(\emptyset)
 = 0.
\]
Es gibt daher für alle $\varepsilon > 0$ ein $N \in \N$ mit $F_\mu(-n) < \varepsilon$ für alle $n \geq N$; dabei folgt aus der Monotonie von $F_\mu$, dass $F_\mu(x) < \varepsilon$ für alle $x \geq N$. Aus der Beliebigkeit von $\varepsilon > 0$ folgt damit, dass $\limes{x}{\infty} F_\mu(x) = 0$.





\section{(Mengen)}
Es sei $0 < \alpha < 1$ beliebig aber fest und $\beta := 1-\alpha$.

\subsection{}\label{ssec:cantoranders}
Die Folge $(M_n)_{n \in \N}$ sei wie folgt definiert: Man beginne mit $M_0 := [0,1]$. $M_1$ konstruiert man aus $M_0$ indem man in der Mitte von $M_0$ das offene Intervall $\left(\frac{1}{2}-\frac{\beta}{4},\frac{1}{2}+\frac{\beta}{4}\right)$ mit Länge $\frac{\beta}{2}$ entfernt, d.h.
\[
 M_1
 = [0,1] \setminus \left(\frac{1}{2}-\frac{\beta}{4},\frac{1}{2}+\frac{\beta}{4}\right)
 = \left[0,\frac{1}{2}-\frac{\beta}{4}\right] \cup \left[\frac{1}{2}+\frac{\beta}{4},1\right].
\]
$M_2$ konstruiert man aus $M_1$ indem man aus jedem der beiden Intervalle in $M_1$ das jeweils mittige offene Intervall der Länge $\frac{\beta/4}{2} = \frac{\beta}{8}$ entfernt. $M_3$ ergibt sich aus $M_2$, indem man aus jedem der vier Intervalle in $M_2$ das jeweils mittige offene Intervall der Länge $\frac{\beta/8}{4} = \frac{\beta}{32}$ entfernt. Nach dem gleichen Prinzip konstruiert man rekursiv alle weiteren $M_n$.

Offenbar ist $(M_n)_{n \in \N}$ eine fallende Folge, wobei für alle $n \in \N$ die Menge $M_n$ aus $2^n$ disjunkten, abgeschlossenen, gleichlangen Intervallen besteht; insbesondere ist $M_n$ als endliche Vereinigung abgeschlossener Mengen abgeschlossen. Die Intervalle haben zusammen eine Gesamtlänge von $1-\beta\sum_{k=1}^n 2^{-k}$ also jedes einzelne eine Länge von
\[
 \frac{1-\beta\sum_{k=1}^n 2^{-k}}{2^n}.
\]

Es sei nun
\[
 M := \bigcap_{n \in \N} M_n.
\]
Als Schnitt abgeschlossener Mengen ist $M$ ebenfalls abgeschlossen. Da $M_n \subseteq [0,1]$ für alle $n \in \N$ ist auch $M \subseteq [0,1]$. $M$ enthält keine nichtleere offene Menge: Für alle $x \in M$ ist $x \in M_n$ für alle $n \in \N$, also $x$ für alle $n \in \N$ in einem Intervall der Länge
\[
 \frac{1-\beta\sum_{k=1}^n 2^{-k}}{2^n}
\]
enthalten. Da
\[
 \limes{n}{\infty} \frac{1-\beta\sum_{k=1}^n 2^{-k}}{2^n} = 0
\]
gibt es in $M$ keinen $\varepsilon$-Ball um $x$ in $M$.

Es ist $\lambda(M) = \alpha$: Da alle $M_n$, sowie auch $M$, als abgeschlossene Mengen $\lambda$-messbar sind, und $(M_n)_{n \in \N}$ ein fallende Folge auf $M_\lambda$ mit $\lambda(M_n) \leq \lambda([0,1]) = 1$ für alle $n \in \N$ ist, folgt daraus, dass $\lambda$ ein Maß ist, dass
\[
 \mu(M)
 = \mu\left( \bigcap_{n \in \N} M_n \right)
 = \limes{n}{\infty} \mu(M_n)
 = \limes{n}{\infty} 1-\beta\sum_{k=1}^n 2^{-k}
 = 1-\beta
 = \alpha.
\]
Dabei ergibt sich $\mu(M_n)$ aus der endlichen Additivität von $\lambda$, sowie der Tatsache, dass $M_n$ eine endliche Vereinigung disjunkter, abgeschlossener Intervalle ist, deren Maß genau ihre Länge ist.


\subsection{}
Wie in Aufgabenteil \textbf{\ref{ssec:cantoranders}} gezeigt enthält $[0,1]$ eine abgeschlossene Teilmenge $A$, die keine nichtleeren offenen Mengen enthält, und für die $\lambda(A) = \beta$ (da $0 < \alpha < 1$ ist auch $0 < \beta < 1$). Es sei
\[
 B := (0,1) \setminus A = (0,1) \cap A^c.
\]
Da $A$ abgeschlossen ist, ist $B$ als endlicher Schnitt offener Mengen offen. Wegen der endlichen Additivität von $\mu$ ist
\[
 \lambda(B) = \lambda((0,1))-\lambda(A) = 1-\beta = \alpha.
\]
$B$ liegt dicht in $[0,1]$: Gebe es $x,y \in [0,1]$ mit $x < y$ und $z \not\in B$ für alle $x \leq z \leq y$, so ist $(x,y) \subseteq A$ ein nichtleeres offenes Intervall, im Widerspruch zur Annahme, dass $A$ kein solches enthält.





\section{(Lebesgue-messbare Mengen)}\label{sec:äuqivalentlebesquemessbar}


\subsection*{$(i) \Rightarrow (ii)$}
Sei $A \subseteq \R^n$ Lebesgue-messbar. Wie aus der Vorlesung bekannt ist
\[
 A = \sup\{\lambda^*_n(K) : K \subseteq A, K \text{ ist kompakt}\}.
\]
Für alle $k \in \N, k \geq 1$ gibt es daher eine abgeschlossene Teilmenge $K_k \subseteq A$ mit
\begin{equation}\label{eq:supgreater}
 \lambda^*_n(K_k)+\frac{1}{k} \geq \lambda^*_n(A).
\end{equation}
Es sei $K := \bigcup_{k \in \N} K_n$. Da alle $K_k$ kompakt und damit auch abgeschlossen sind, ist nach Definition $K \in F_\sigma$. Da alle $K_k$ abgeschlossen sind, also $K_k \in \mc{B}(\R)$ für alle $k \in \N$, ist wegen der Abgeschlossenheit von $\mc{B}(\R^n)$ unter abzählbaren Vereinigungen auch $K \in \mc{B}(\R)$. Es ist $\mu(K) = \mu(A)$: Da $K \subseteq A$ folgt $\mu(K) \leq \mu(A)$ aus der Monotonie von $\mu$. Wegen der Monotonie von $\mu$ folgt aus \textbf{\eqref{eq:supgreater}} auch, dass für alle $k \geq 1$
\[
 \mu(K) + \frac{1}{k} \geq \mu(K_k) + \frac{1}{k} \geq \lambda^*_n(A)
\]
Für $k \rightarrow \infty$ ergibt sich damit, dass $\mu(K) \geq \mu(A)$.

Sei $N := A \setminus K$. Aus der endlichen Additivität von $\mu$ ergibt sich aus $\lambda^*_n(A) = \lambda^*_n(K)$, dass
\[
 \lambda^*_n(N) = \lambda^*_n(A) - \lambda^*_n(K) = 0.
\]
Es ist also $K \in F_\sigma$, $N \subseteq \R^n$ mit $\lambda^*_n(N) = 0$ und $A = K \cup N$.


\subsection*{$(ii) \Rightarrow (iii)$}
Es ist $A_1 \in F_\sigma$ sowie
\begin{align*}
 \lambda^*_n( (A \setminus A_1) \cup (A_1 \setminus A) )
 &\leq \lambda^*_n(A \setminus A_1) + \lambda^*_n(A_1 \setminus A) \\
 &\leq \lambda^*(A_2) + \lambda^*_n(\emptyset) = 0.
\end{align*}


\subsection*{$(iii) \Rightarrow (i)$}
Da $B \in F_\sigma$ ist $B = \bigcup_{k \in \N} A_k$ für abgeschlossene Mengen $A_k \subseteq \R^n$. Es ist also auch $B \in \mc{B}(\R^n)$, also $B$ messbar. Da
\[
 \lambda^*_n( \underbrace{(A \setminus B) \cup (B \setminus A)}_{=: C} ) = 0
\]
ist auch $C$ messbar. Da $\mc{M}_{\lambda^*_n}$ eine $\sigma$-Algebra ist, ist damit auch
\[
 A = (B \cup C) \setminus (B \cap C) = (B \setminus C) \cup (C \setminus B) \in \mc{M}_{\lambda^*_n},
\]
also $A$ messbar.





\section{(Projektion des Lebesguemaßes)}


\subsection{}
Es gilt die Axiome eines äußeren Maßes zu überprüfen: Es ist
\[
 \mu^*(\emptyset) = \lambda^*_1(\pi(\emptyset)) = \lambda^*_1(\emptyset) = 0.
\]
Für $A \subseteq B \subseteq \R^2$ ist $\pi(A) \subseteq \pi(B)$, also wegen der Monotonie von $\lambda^*_1$
\[
 \mu^*(A) = \lambda^*_1(\pi(A)) \leq \lambda^*_1(\pi(B)) = \mu^*(B),
\]
was die Monotonie von $\mu^*$ zeigt. $\mu^*$ ist subadditiv, denn $\lambda^*_1$ ist als äußeres Maß auf $\R$ subadditiv, weshalb für jede Folge $(A_n)_{n \in \N}$ auf $\R^2$
\begin{align*}
 \mu^*\left( \bigcup_{n \in \N} A_n \right)
 &= \lambda^*_1\left( \pi\left( \bigcup_{n \in \N} A_n\right) \right)
 = \lambda^*_1\left(  \bigcup_{n \in \N} \pi(A_n) \right) \\
 &\leq \sum_{n \in \N} \lambda^*_1(\pi(A_n))
 = \sum_{n \in \N} \mu^*(A_n).
\end{align*}


\subsection{}
Sei $B \subseteq \R^2$. Die folgenden Aussagen sind äquivalent:
\begin{enumerate}[(i)]
 \item $B$ ist $\mu^*$-messbar.\label{item:B ist mu* messbar}
 \item $\pi(B)$ ist $\lambda^*_1$-messbar. \label{item:pi(B) lambda messbar}
 \item Es gibt $\lambda^*_1$-messbare Mengen $B_0, B_1 \subseteq \R$ mit $B_0 \subseteq B_1$, $\lambda^*_1(B_1 \setminus B_0) = 0$ und $B_0 \times \R \subseteq B \subseteq B_1 \times \R$.\label{item:es gibt b0 b1}
\end{enumerate}

\subsubsection*{$\eqref{item:B ist mu* messbar} \Rightarrow \eqref{item:pi(B) lambda messbar}$}
Sei $A \subseteq \R$ beliebig aber fest. Es gilt zu zeigen, dass
\[
 \lambda^*_1(A)
 = \lambda^*_1(A \cap \pi(B)) + \lambda(A \cap \pi(B)^c).
\]
Aufgrund der Subadditivität von $\lambda^*_1$ genügt es hierfür zu zeigen, dass
\[
 \lambda^*_1(A)
 \geq \lambda^*_1(A \cap \pi(B)) + \lambda^*_1(A \cap \pi(B)^c).
\]

Wer bemerken zunächst, dass $\mu(B)^c \subseteq \mu(B^c)$: Für $y \in \mu(B)^c$ ist $y \not\in \mu(B)$. Aus der Surjektivität von $\pi$ folgt, dass es $x \in \R^2$ mit $\pi(x) = y$ gibt. Da $y \not\in \pi(B)$ ist $x \not\in B$. Also ist $x \in B^c$ und daher $y \in \mu(B^c)$.

Auch ist für alle $C \in \R^2$
\begin{equation}\label{eq:schnitt pi vertauschen}
 A \cap \pi(C) = \pi((A \times \R) \cap C),
\end{equation}
da
\begin{align*}
 A \cap \pi(C)
 &= \{x \in \R : x \in A \text{ und } x \in \pi(C)\} \\
 &= \{x \in \R : x \in A \text{ und } (x,y) \in C \text{ für ein } y \in \R\} \\
 &= \{x \in \R : (x,y) \in A\times\R \text{ und } (x,y) \in C \text{ für ein } y \in \R\} \\
 &= \{x \in \R : (x,y) \in (A\times\R) \cap C \text{ für ein } y \in \R\} \\
 &= \pi((A\times\R) \cap C).
\end{align*}
Zusammen ergibt sich damit, dass
\begin{align*}
 &\, \lambda_1^*(A \cap \pi(B)) + \lambda_1^*(A \cap \pi(B)^c)
 \underset{\mu \text{ monoton}}{\leq} \lambda_1^*(A \cap \pi(B)) + \lambda_1^*(A \cap \pi(B^c)) \\
 =&\, \lambda_1^*( \pi((A\times\R)\cap B) ) + \lambda_1^*( \pi((A\times\R)\cap B^c) ) \\
 =&\, \mu^*( (A\times\R) \cap B) + \mu^*( (A\times\R) \cap B^c )
 \underset{B\, \mu^*-\text{messbar}}{=} \mu^*(A \times \R) \\
 =&\, \lambda_1^*(\pi(A \times \R))
 = \lambda_1^*(A).
\end{align*}
Dies zeigt, dass $\mu(B)$ $\lambda_1^*$-messbar ist.

\subsubsection*{$\eqref{item:pi(B) lambda messbar} \Rightarrow \eqref{item:es gibt b0 b1}$}
Da $\pi(B)$ $\lambda_1^*$-messbar ist, folgt aus \textbf{Aufgabe 3}, dass es $A_1 \in F_\sigma$ und $A_2 \subseteq \R$ mit $\lambda_1^*(A_2) = 0$ gibt, so dass $\pi(B) = A_1 \cup A_2$. $A_1$ ist messbar: Nach Definition ist $A_1 = \bigcup_{n \in \N} K_n$ für eine Familie $(K_n)_{n \in \N}$ abgeschlossener Mengen auf $\R$. Da alle $K_n$ als abgeschlossene Mengen $\lambda_1^*$-messbar sind, und $\mc{M}_{\lambda_1^*}$ eine $\sigma$-Algebra ist, ist wegen der $\sigma$-Additivität von $\mc{M}_{\lambda_1^*}$ damit auch $A_1$ $\lambda_1^*$-messbar.

Sei nun $B_0 := A_1$ und $B_1 := A_1 \cup A_2 = \pi(B)$. $B_0$ und $B_1$ sind $\lambda_1^*$-messbar, und wegen der Monotonie von $\lambda_1^*$ ist
\[
 \lambda_1^*(B_1 \setminus B_0) \leq \lambda_1^*(A_2) = 0.
\]
Da $B_0 \subseteq \pi(B) = B_1$ ist auch
\[
 B_0 \times \R = \pi^{-1}(B_0) \subseteq B \subseteq \pi^{-1}(B_1) = B_1 \times \R.
\]

\subsubsection*{$\eqref{item:es gibt b0 b1} \Rightarrow \eqref{item:B ist mu* messbar}$}
Sei $A \subseteq \R^2$ beliebig aber fest. Es gilt zu zeigen, dass
\[
 \mu^*(A) = \mu^*(A \cap B) + \mu^*(A \cap B^c).
\]
Wegen der Subadditivität von $\mu^*$ genügt es zu zeigen, dass
\[
 \mu^*(A) \geq \mu^*(A \cap B) + \mu^*(A \cap B^c).
\]

Es ist
\[
 A \cap (B_0 \times \R) \subseteq A \cap B \subseteq A \cap (B_1 \times \R).
\]
Wegen \eqref{eq:schnitt pi vertauschen} ist daher
\begin{equation}\label{eq: A cap B nach oben}
 \pi(A \cap B)
 \subseteq \pi(A \cap (B_1 \times \R))
 = \pi(A) \cap B_1,
\end{equation}
Analog ergibt sich wegen
\[
 (B_0 \times \R)^c
 = \pi^{-1}(B_0)^c
 = \pi^{-1}(B_0^c)
 = (B_0^c \times \R)
\]
auch, dass
\[
 A \cap B^c
 \subseteq A \cap (B_0 \times \R)^c
 = A \cap (B_0^c \times \R),
\]
und wegen \eqref{eq:schnitt pi vertauschen} daher
\begin{equation}\label{eq: A cap B^c nach oben}
 \pi(A \cap B^c)
 \subseteq \pi(A \cap (B_0 \times \R)^c)
 = \pi(A \cap (B_0^c \times \R))
 = \pi(A) \cap B_0^c.
\end{equation}
Zusammengefasst ergibt sich, dass
\begin{align}
 &\; \mu^*(A \cap B) + \mu^*(A \cap B^c) \notag \\
 =&\; \lambda_1^*(\pi(A \cap B)) + \lambda_1^*(\pi(A \cap B^c)) \notag \\
 \leq&\; \lambda_1^*(\pi(A) \cap B_1) + \lambda_1^*(\pi(A) \cap B_0^c) \label{align: monotonie}\\
 =&\; \lambda_1^*(\pi(A) \cap B_1) + \lambda_1^*(\pi(A) \cap B_0^c \cap B_1) + \lambda_1^*(\pi(A) \cap B_0^c \cap B_1^c) \label{align: B1 messbar} \\
 =&\; \lambda_1^*(\pi(A) \cap B_1) + \lambda_1^*(\pi(A) \cap B_1^c) \label{align: umformungen} \\
 =&\; \lambda_1^*(\pi(A)) = \mu^*(A) \label{align: B1 messbar 2},
\end{align}
wobei bei \eqref{align: monotonie} die Monotonie von $\lambda_1^*$ in Kombination mit \eqref{eq: A cap B nach oben} und \eqref{eq: A cap B^c nach oben} genutzt wird, bei \eqref{align: B1 messbar} die $\lambda_1^*$-messbarkeit von $B_1$, bei \eqref{align: umformungen} die Monotonie von $\lambda_1^*$ und dass
\[
 B_0^c \cap B_1 = B_1 \setminus B_0
\]
eine Nullmenge ist, sowie dass
\[
 B_0^c \cap B_1^c = (B_0 \cup B_1)^c = B_1^c,
\]
und bei \eqref{align: B1 messbar 2} noch einmal die $\lambda_1^*$-Messbarkeit von $B_1$. Dies zeigt die $\mu^*$-Messbarkeit von $B$.


















\end{document}
