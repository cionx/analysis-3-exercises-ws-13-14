\documentclass[a4paper,10pt]{article}
%\documentclass[a4paper,10pt]{scrartcl}

\usepackage{xltxtra}
\usepackage[ngerman]{babel}
\usepackage{amsmath}
\usepackage{amssymb}
\usepackage{amsthm}
\usepackage{mathtools}
\usepackage{nicefrac}
\usepackage{enumerate}
\usepackage{leftidx}

\makeatletter
\g@addto@macro\th@definition{\thm@headpunct{:}}
\makeatother
\theoremstyle{definition}
\newtheorem{lem}{Lemma}
\newtheorem{beh}{Behauptung}
\newtheorem{bem}{Bemerkung}
\newtheorem*{ia}{Induktionsanfang}
\newtheorem*{is}{Induktionsschritt}

\renewcommand{\thesection}{Aufgabe \arabic{section}.}
\renewcommand{\thesubsection}{\alph{subsection})}
\renewcommand{\thesubsubsection}{(\roman{subsubsection})}

\makeatletter
\def\moverlay{\mathpalette\mov@rlay}
\def\mov@rlay#1#2{\leavevmode\vtop{%
   \baselineskip\z@skip \lineskiplimit-\maxdimen
   \ialign{\hfil$\m@th#1##$\hfil\cr#2\crcr}}}
\newcommand{\charfusion}[3][\mathord]{
    #1{\ifx#1\mathop\vphantom{#2}\fi
        \mathpalette\mov@rlay{#2\cr#3}
      }
    \ifx#1\mathop\expandafter\displaylimits\fi}
\makeatother

\newcommand{\N}{\mathbb{N}}
\newcommand{\Z}{\mathbb{Z}}
\newcommand{\Q}{\mathbb{Q}}
\newcommand{\R}{\mathbb{R}}
\newcommand{\C}{\mathbb{C}}
\newcommand{\A}{\mathcal{A}}
\newcommand{\La}{\mathcal{L}}
\newcommand{\dx}{\,\text{d}x}
\newcommand{\dy}{\,\text{d}y}
\newcommand{\dt}{\,\text{d}t}
\newcommand{\du}{\,\text{d}u}
\newcommand{\Img}{\operatorname{Im}}
\newcommand{\Real}{\operatorname{Re}}
\newcommand{\Imag}{\operatorname{Im}}
\newcommand{\sgn}{\operatorname{sgn}}
\newcommand{\dotcup}{\ensuremath{\mathaccent\cdot\cup}}
\newcommand{\bigdotcup}{\charfusion[\mathop]{\bigcup}{\cdot}}
\newcommand{\ceil}[1]{\left\lceil{#1}\right\rceil}
\newcommand{\floor}[1]{\left\lfloor{#1}\right\rfloor}
\newcommand{\mc}[1]{\mathcal{#1}}
\newcommand{\limes}[2]{\lim_{#1 \rightarrow #2}}
\newcommand{\limessup}[1]{\limsup_{#1 \rightarrow \infty}}
\newcommand{\limesinf}[1]{\liminf_{#1 \rightarrow \infty}}
\newcommand{\vect}[1]{\begin{pmatrix}#1\end{pmatrix}}
\newcommand{\partd}[2]{\frac{\partial #1}{\partial #2}}
\newcommand{\op}[1]{\left\|#1\right\|_{\text{op}}}

\makeatletter
\renewcommand*\env@matrix[1][*\c@MaxMatrixCols c]{%
  \hskip -\arraycolsep
  \let\@ifnextchar\new@ifnextchar
  \array{#1}}
\makeatother

\setromanfont[Mapping=tex-text]{Linux Libertine O}
% \setsansfont[Mapping=tex-text]{DejaVu Sans}
% \setmonofont[Mapping=tex-text]{DejaVu Sans Mono}
\parindent 0pt

\title{\sc Analysis III \\ \Large 4. Aufgabenblatt}
\author{Jendrik Stelzner}
\date{\today}

\begin{document}
\maketitle





\section{(Lebesgue-Stieltjes-Maße)}


\subsection{}
Da $\mu$ endlich und monoton ist, ist $\mu$ auch beschränkt, denn für alle $A \in \mc{B}(\R)$ ist
\[
 \mu(A) \leq \mu(\R) < \infty.
\]
Aus der Beschränktheit von $\mu$ folgt die Beschränktheit von $F_\mu$, da damit für alle $x \in \R$
\[
 F_\mu(x) = \mu((-\infty,x]) < \infty.
\]
Die Monotonie von $F_\mu$ ergibt sich daraus, dass für $x, y \in \R$ mit $x \leq y$ offenbar $(-\infty, x] \subseteq (-\infty, y]$ und wegen der Monotonie von $\mu$ daher
\[
 F_\mu(x) = \mu((-\infty,  x]) \leq \mu((-\infty, y]) = F_\mu(y).
\]

Zum Nachweis der Rechtsstetigkeit sei $(x_n)_{n \in \N}$ eine motonon fallende Folge auf $\R$ mit $\limes{n}{\infty} x_n = x \in \R$. Es gilt zu zeigen, dass $\limes{n}{\infty} F_\mu(x_n) = F_\mu(x)$. Aus der Monotonie von $F_\mu$ folgt, dass $F_\mu(x) \leq F_\mu(x_n)$ für alle $n \in \N$, also auch
\[
 F_\mu(x) \leq \limes{n}{\infty} F_\mu(x_n).
\]
Zum Beweis der anderen Ungleichung sei $\varepsilon > 0$ beliebig aber fest. Wir bemerken, dass die Folge $( (x,x+\frac{1}{n}] )_{n \in \N}$ fallend ist, und da $\mu$ ein beschränktes Maß ist daher
\[
 \limes{n}{\infty} \mu\left( \left(x,x+\frac{1}{n}\right] \right)
 = \mu\left( \bigcap_{n \in \N} \left(x,x+\frac{1}{n}\right] \right)
 = \mu( \emptyset ) = 0.
\]
Es gibt also ein $N \in \N$ mit
\[
 \mu\left( \left(x,x+\frac{1}{n}\right] \right) < \varepsilon \text{ für alle } n \geq N.
\]
Da $\limes{n}{\infty} x_n = x$ gibt es ein $n_0 \in \N$ mit $x \leq x_n \leq x+\frac{1}{N}$ für alle $n \geq n_0$. Aufgrund der endlichen Additivität von $\mu$ ergibt sich daher, dass für $n \geq n_0$
\begin{align*}
 F_\mu(x_n)
 &= \mu( (-\infty, x_n] )
 = \mu( (-\infty, x]\ \dotcup\ (x,x_n] )
 = \mu( (-\infty, x] ) + \mu( (x,x_n] ) \\
 &= F_\mu(x) + \mu( (x,x_n] )
 \leq F_\mu(x) + \mu\left( \left(x,x+\frac{1}{N}\right] \right)
 \leq F_\mu(x) + \varepsilon.
\end{align*}
Also ist auch
\[
 \limes{n}{\infty} F_\mu(x_n) \leq F_\mu(x) + \varepsilon.
\]
Aus der Beliebigkeit von $\varepsilon > 0$ folgt, dass
\[
 \limes{n}{\infty} F_\mu(x_n) \leq F_\mu(x).
\]





\section{(Mengen)}





\section{(Lebesgue-messbare Mengen)}





\section{(Projektion des Lebesguemaßes)}












\end{document}
