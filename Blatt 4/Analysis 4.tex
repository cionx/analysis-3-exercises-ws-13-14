\documentclass[a4paper,10pt]{article}
%\documentclass[a4paper,10pt]{scrartcl}

\usepackage{xltxtra}
\usepackage[ngerman]{babel}
\usepackage{amsmath}
\usepackage{amssymb}
\usepackage{amsthm}
\usepackage{mathtools}
\usepackage{nicefrac}
\usepackage{enumerate}
\usepackage{leftidx}

\makeatletter
\g@addto@macro\th@definition{\thm@headpunct{:}}
\makeatother
\theoremstyle{definition}
\newtheorem{lem}{Lemma}
\newtheorem{beh}{Behauptung}
\newtheorem{bem}{Bemerkung}
\newtheorem*{ia}{Induktionsanfang}
\newtheorem*{is}{Induktionsschritt}

\renewcommand{\thesection}{Aufgabe \arabic{section}.}
\renewcommand{\thesubsection}{\alph{subsection})}
\renewcommand{\thesubsubsection}{(\roman{subsubsection})}

\makeatletter
\def\moverlay{\mathpalette\mov@rlay}
\def\mov@rlay#1#2{\leavevmode\vtop{%
   \baselineskip\z@skip \lineskiplimit-\maxdimen
   \ialign{\hfil$\m@th#1##$\hfil\cr#2\crcr}}}
\newcommand{\charfusion}[3][\mathord]{
    #1{\ifx#1\mathop\vphantom{#2}\fi
        \mathpalette\mov@rlay{#2\cr#3}
      }
    \ifx#1\mathop\expandafter\displaylimits\fi}
\makeatother

\newcommand{\N}{\mathbb{N}}
\newcommand{\Z}{\mathbb{Z}}
\newcommand{\Q}{\mathbb{Q}}
\newcommand{\R}{\mathbb{R}}
\newcommand{\C}{\mathbb{C}}
\newcommand{\A}{\mathcal{A}}
\newcommand{\La}{\mathcal{L}}
\newcommand{\dx}{\,\text{d}x}
\newcommand{\dy}{\,\text{d}y}
\newcommand{\dt}{\,\text{d}t}
\newcommand{\du}{\,\text{d}u}
\newcommand{\Img}{\operatorname{Im}}
\newcommand{\Real}{\operatorname{Re}}
\newcommand{\Imag}{\operatorname{Im}}
\newcommand{\sgn}{\operatorname{sgn}}
\newcommand{\dotcup}{\ensuremath{\mathaccent\cdot\cup}}
\newcommand{\bigdotcup}{\charfusion[\mathop]{\bigcup}{\cdot}}
\newcommand{\ceil}[1]{\left\lceil{#1}\right\rceil}
\newcommand{\floor}[1]{\left\lfloor{#1}\right\rfloor}
\newcommand{\mc}[1]{\mathcal{#1}}
\newcommand{\limes}[2]{\lim_{#1 \rightarrow #2}}
\newcommand{\limessup}[1]{\limsup_{#1 \rightarrow \infty}}
\newcommand{\limesinf}[1]{\liminf_{#1 \rightarrow \infty}}
\newcommand{\vect}[1]{\begin{pmatrix}#1\end{pmatrix}}
\newcommand{\partd}[2]{\frac{\partial #1}{\partial #2}}
\newcommand{\op}[1]{\left\|#1\right\|_{\text{op}}}

\makeatletter
\renewcommand*\env@matrix[1][*\c@MaxMatrixCols c]{%
  \hskip -\arraycolsep
  \let\@ifnextchar\new@ifnextchar
  \array{#1}}
\makeatother

\setromanfont[Mapping=tex-text]{Linux Libertine O}
% \setsansfont[Mapping=tex-text]{DejaVu Sans}
% \setmonofont[Mapping=tex-text]{DejaVu Sans Mono}
\parindent 0pt

\title{\sc Analysis III \\ \Large 4. Aufgabenblatt}
\author{Jendrik Stelzner}
\date{\today}

\begin{document}
\maketitle





\section{(Lebesgue-Stieltjes-Maße)}


\subsection{}
Da $\mu$ endlich und monoton ist, ist $\mu$ auch beschränkt, denn für alle $A \in \mc{B}(\R)$ ist
\[
 \mu(A) \leq \mu(\R) < \infty.
\]
Aus der Beschränktheit von $\mu$ folgt die Beschränktheit von $F_\mu$, da damit für alle $x \in \R$
\[
 F_\mu(x) = \mu((-\infty,x]) < \infty.
\]
Die Monotonie von $F_\mu$ ergibt sich daraus, dass für $x, y \in \R$ mit $x \leq y$ offenbar $(-\infty, x] \subseteq (-\infty, y]$ und wegen der Monotonie von $\mu$ daher
\[
 F_\mu(x) = \mu((-\infty,  x]) \leq \mu((-\infty, y]) = F_\mu(y).
\]

Zum Nachweis der Rechtsstetigkeit sei $(x_n)_{n \in \N}$ eine motonon fallende Folge auf $\R$ mit $\limes{n}{\infty} x_n = x \in \R$. Es gilt zu zeigen, dass $\limes{n}{\infty} F_\mu(x_n) = F_\mu(x)$. Aus der Monotonie von $F_\mu$ folgt, dass $F_\mu(x) \leq F_\mu(x_n)$ für alle $n \in \N$, also auch
\[
 F_\mu(x) \leq \limes{n}{\infty} F_\mu(x_n).
\]
Zum Beweis der anderen Ungleichung sei $\varepsilon > 0$ beliebig aber fest. Wir bemerken, dass die Folge $( (x,x+\frac{1}{n}] )_{n \in \N}$ fallend ist, und da $\mu$ ein beschränktes Maß ist daher
\[
 \limes{n}{\infty} \mu\left( \left(x,x+\frac{1}{n}\right] \right)
 = \mu\left( \bigcap_{n \in \N} \left(x,x+\frac{1}{n}\right] \right)
 = \mu( \emptyset ) = 0.
\]
Es gibt also ein $N \in \N$ mit
\[
 \mu\left( \left(x,x+\frac{1}{n}\right] \right) < \varepsilon \text{ für alle } n \geq N.
\]
Da $\limes{n}{\infty} x_n = x$ gibt es ein $n_0 \in \N$ mit $x \leq x_n \leq x+\frac{1}{N}$ für alle $n \geq n_0$. Aufgrund der endlichen Additivität von $\mu$ ergibt sich daher, dass für $n \geq n_0$
\begin{align*}
 F_\mu(x_n)
 &= \mu( (-\infty, x_n] )
 = \mu( (-\infty, x]\ \dotcup\ (x,x_n] )
 = \mu( (-\infty, x] ) + \mu( (x,x_n] ) \\
 &= F_\mu(x) + \mu( (x,x_n] )
 \leq F_\mu(x) + \mu\left( \left(x,x+\frac{1}{N}\right] \right)
 \leq F_\mu(x) + \varepsilon.
\end{align*}
Also ist auch
\[
 \limes{n}{\infty} F_\mu(x_n) \leq F_\mu(x) + \varepsilon.
\]
Aus der Beliebigkeit von $\varepsilon > 0$ folgt, dass
\[
 \limes{n}{\infty} F_\mu(x_n) \leq F_\mu(x).
\]





\section{(Mengen)}
Es sei $0 < \alpha < 1$ beliebig aber fest und es sei $0 < \beta < 1$ definiert als $\beta := 1-\alpha$.

\subsection{}\label{ssec:cantoranders}
Die Folge $(M_n)_{n \in \N}$ sei wie folgt definiert: Man beginne mit $M_0 := [0,1]$. $M_1$ konstruiert man aus $M_0$ indem man in der Mitte von $M_0$ das offene Intervall $\left(\frac{1}{2}-\frac{\beta}{4},\frac{1}{2}+\frac{\beta}{4}\right)$ mit Länge $\frac{\beta}{2}$ entfernt, d.h.
\[
 M_1
 = [0,1] \setminus \left(\frac{1}{2}-\frac{\beta}{4},\frac{1}{2}+\frac{\beta}{4}\right)
 = \left[0,\frac{1}{2}-\frac{\beta}{4}\right] \cup \left[\frac{1}{2}+\frac{\beta}{4},1\right].
\]
$M_2$ konstruiert man aus $M_1$ indem man aus jedem der beiden Intervalle in $M_1$ das jeweils offene mittige Intervall der Länge $\frac{\beta/4}{2} = \frac{\beta}{8}$ entfernt. $M_3$ ergibt sich aus $M_2$, indem man aus jedem der vier Intervalle in $M_2$ das jeweils mittige offene Intervall der Länge $\frac{\beta/8}{4} = \frac{\beta}{32}$ entfernt. Nach dem gleichen Prinzip konstruiert man rekursiv alle weiteren $M_n$.

Offenbar ist $(M_n)_{n \in \N}$ eine fallende Folge, wobei für alle $n \in \N$ die Menge $M_n$ aus $2^n$ disjunkten, abgeschlossenen, gleichlangen Intervallen besteht; insbesondere ist $M_n$ als endliche Vereinigung abgeschlossener Mengen abgeschlossen. Die Intervalle haben zusammen eine Gesamtlänge von $1-\beta\sum_{k=1}^n 2^{-k}$ also jedes einzelne eine Länge von $\frac{1-\beta\sum_{k=1}^n 2^{-k}}{2^n}$.

Sei $M := \bigcap_{n \in \N} M_n$. Als Schnitt abgeschlossener Mengen ist $M$ auch abgeschlossen. Da $M_n \subseteq [0,1]$ für alle $n \in \N$ ist auch $M \subseteq [0,1]$. $M$ enthält keine nichtleere offene Menge: Für alle $x \in M$ ist $x \in M_n$ für alle $n \in N$, also für alle $n \in \N$ in einem Intervall der Länge $\frac{1-\beta\sum_{k=1}^n 2^{-k}}{2^n}$. Da
\[
 \limes{n}{\infty} \frac{1-\beta\sum_{k=1}^n 2^{-k}}{2^n} = 0
\]
kann es daher keine $\varepsilon$-Ball um $x$ in $M$ geben. Es ist $\lambda(M) = 0$: Da $(M_n)_{n \in \N}$ ein fallende Folge mit $\lambda(M_n) \leq \lambda([0,1]) = 1$ für alle $n \in \N$ ist, und $\lambda$ ein Maß, ist
\[
 \mu(M)
 = \mu\left( \bigcap_{n \in \N} M_n \right)
 = \limes{n}{\infty} \mu(M_n)
 = \limes{n}{\infty} 1-\beta\sum_{k=1}^n 2^{-k}
 = 1-\beta
 = \alpha.
\]


\subsection{}
Wie in Aufgabenteil \textbf{\ref{ssec:cantoranders}} gezeigt enthält $[0,1]$ eine abgeschlossene Teilmenge $A$, die keine nichtleeren offenen Mengen enthält, und für die $\lambda(A) = \beta$. Es sei
\[
 B := (0,1) \setminus A = (0,1) \cap A^c.
\]
Da $A$ abgeschlossen ist, ist $B$ offen. Wegen der endlichen Additivität von $\mu$ ist daher
\[
 \mu(B) = \mu((0,1))-\mu(a) = 1-\beta = \alpha.
\]
$B$ liegt dicht in $[0,1]$: Gebe es $x,y \in [0,1]$ mit $x < y$ und $z \not\in B$ für alle $x \leq z \leq y$, so ist $(x,y) \subseteq A$ ein nichtleeres offenes Intervall, im Widerspruch zur Annahme, dass $A$ kein solches enthält.





\section{(Lebesgue-messbare Mengen)}


\subsection*{$(i) \Rightarrow (ii)$}
Sei $A \subseteq \R^n$ Lebesgue-messbar. Wie aus der Vorlesung bekannt ist
\[
 A = \sup\{\lambda^*_n(K) : K \subseteq A, K \text{ ist kompakt}\}.
\]
Für alle $k \in \N, k \geq 1$ gibt es daher eine abgeschlossene Teilmenge $K_k \subseteq A$ mit
\begin{equation}\label{eq:supgreater}
 \lambda^*_n(K_k)+\frac{1}{k} \geq \lambda^*_n(A).
\end{equation}
Es sei $K := \bigcup_{k \in \N} K_n$. Da alle $K_k$ kompakt und damit auch abgeschlossen sind, ist nach Definition $K \in F_\sigma$, sowie wegen der $\sigma$-Additiv von $\mc{B}(\R^n)$ auch $K \in \mc{B}(\R)$. Es ist $\mu(K) = \mu(A)$: Da $K \subseteq A$ ist $\mu(K) \leq \mu(A)$ wegen der Monotonie von $\mu$. Wegen der Monotonie von $\mu$ folgt aus \textbf{\eqref{eq:supgreater}} auch, dass für alle $k \geq 1$
\[
 \mu(K) \geq \mu(K_k) \geq \lambda^*_n(A) - \frac{1}{k}.
\]
Für $k \rightarrow \infty$ ergibt sich damit, dass $\mu(K) \geq \mu(A)$.

Sei $N := A \setminus K$. Aus der endlichen Additivität von $\mu$ ergibt sich aus $\lambda^*_n(A) = \lambda^*_n(K)$, dass
\[
 \lambda^*_n(N) = \lambda^*_n(A) - \lambda^*_n(K) = 0.
\]
Es ist also $K \in F_\sigma$, $N \subseteq \R^n$ mit $\lambda^*_n(N) = 0$ und $A = K \cup N$.


\subsection*{$(ii) \Rightarrow (iii)$}
Es ist $A_1 \in F_\sigma$ sowie
\begin{align*}
 \lambda^*_n( (A \setminus A_1) \cup (A_1 \setminus A) )
 &\leq \lambda^*_n(A \setminus A_1) + \lambda^*_n(A_1 \setminus A) \\
 &\leq \lambda^*(A_2) + \lambda^*_n(\emptyset) = 0.
\end{align*}


\subsection*{$(iii) \Rightarrow (i)$}
Da $B \in F_\sigma$ ist $B = \bigcup_{k \in \N} A_k$ für abgeschlossene Mengen $A_k \subseteq \R^n$. Es ist also auch $B \in \mc{B}(\R^n)$, also $B$ messbar. Da
\[
 \lambda^*_n( \underbrace{(A \setminus B) \cup (B \setminus A)}_{=: C} ) = 0
\]
ist auch $C$ messbar. Da $\mc{M}_{\lambda^*_n}$ eine $\sigma$-Algebra ist, ist damit auch
\[
 A = (B \cup C) \setminus (B \cap C) = (B \setminus C) \cup (C \setminus B) \in \mc{M}_{\lambda^*_n},
\]
also $A$ messbar.











\section{(Projektion des Lebesguemaßes)}












\end{document}
