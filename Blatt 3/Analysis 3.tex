\documentclass[a4paper,10pt]{article}
%\documentclass[a4paper,10pt]{scrartcl}

\usepackage{xltxtra}
\usepackage[ngerman]{babel}
\usepackage{amsmath}
\usepackage{amssymb}
\usepackage{amsthm}
\usepackage{mathtools}
\usepackage{nicefrac}
\usepackage{enumerate}
\usepackage{leftidx}

\makeatletter
\g@addto@macro\th@definition{\thm@headpunct{:}}
\makeatother
\theoremstyle{definition}
\newtheorem{lem}{Lemma}
\newtheorem{beh}{Behauptung}
\newtheorem{bem}{Bemerkung}
\newtheorem*{ia}{Induktionsanfang}
\newtheorem*{is}{Induktionsschritt}

\renewcommand{\thesection}{Aufgabe \arabic{section}.}
\renewcommand{\thesubsection}{\alph{subsection})}
\renewcommand{\thesubsubsection}{(\roman{subsubsection})}

\newcommand{\N}{\mathbb{N}}
\newcommand{\Z}{\mathbb{Z}}
\newcommand{\Q}{\mathbb{Q}}
\newcommand{\R}{\mathbb{R}}
\newcommand{\C}{\mathbb{C}}
\newcommand{\A}{\mathcal{A}}
\newcommand{\La}{\mathcal{L}}
\newcommand{\dx}{\,\text{d}x}
\newcommand{\dy}{\,\text{d}y}
\newcommand{\dt}{\,\text{d}t}
\newcommand{\du}{\,\text{d}u}
\newcommand{\Img}{\operatorname{Im}}
\newcommand{\Real}{\operatorname{Re}}
\newcommand{\Imag}{\operatorname{Im}}
\newcommand{\sgn}{\operatorname{sgn}}
\newcommand{\dcup}{\dot{\cup}}
\newcommand{\ceil}[1]{\left\lceil{#1}\right\rceil}
\newcommand{\floor}[1]{\left\lfloor{#1}\right\rfloor}
\newcommand{\mc}[1]{\mathcal{#1}}
\newcommand{\limes}[2]{\lim_{#1 \rightarrow #2}}
\newcommand{\limessup}[1]{\limsup_{#1 \rightarrow \infty}}
\newcommand{\limesinf}[1]{\liminf_{#1 \rightarrow \infty}}
\newcommand{\vect}[1]{\begin{pmatrix}#1\end{pmatrix}}
\newcommand{\partd}[2]{\frac{\partial #1}{\partial #2}}
\newcommand{\op}[1]{\left\|#1\right\|_{\text{op}}}

\makeatletter
\renewcommand*\env@matrix[1][*\c@MaxMatrixCols c]{%
  \hskip -\arraycolsep
  \let\@ifnextchar\new@ifnextchar
  \array{#1}}
\makeatother

\setromanfont[Mapping=tex-text]{Linux Libertine O}
% \setsansfont[Mapping=tex-text]{DejaVu Sans}
% \setmonofont[Mapping=tex-text]{DejaVu Sans Mono}
\parindent 0pt

\title{\sc Analysis III \\ \Large 3. Aufgabenblatt}
\author{Jendrik Stelzner}
\date{\today}

\begin{document}
\maketitle





\begin{lem} \label{produktdisjunkt} 
 Sei $(A_n)_{n \in I}$ eine Familie von Mengen über einer Indexmenge $I \subseteq \N$ und $A_n = \dot{\bigcup}_{m \in I_n} B^n_m$ für eine Familie von Mengen $(B^n_m)_{m \in I_n}$ über einer Indexmenge $I_n \subseteq \N$  für alle $n \in \N$. Dann gilt
 \[
  \prod_{n \in I} A_n
  = \prod_{n \in I} \dot{\bigcup_{m \in I_n}} B^n_m
  = \dot{\bigcup_{(b_n) \in \prod_{n \in I} I_n}}\ \prod_{n \in I} B^n_{b_n}.
 \]
 (Die Indexmengen werden auf Teilmengen von $\N$ eingeschränkt, und nicht etwa beliebig gewählt, um das Auswahlaxiom nicht nutzen zu müssen.)
\end{lem}
\begin{proof}[Beweis des Lemmas:]
 Sei $(a_n)_{n \in I} \in \prod_{n \in I} A_n$. Da $a_n \in A_n$ für alle $n \in I$, gibt es für alle $n \in I$ ein $b_n \in I_n$ mit $a_n \in B^n_{b_n}$. Daher ist \[(a_n)_{n \in I} \in \prod_{n \in I} B^n_{b_n} \subseteq \bigcup_{(b_n) \in \prod_{n \in I} I_n}\ \prod_{n \in I} B^n_{b_n},\]
 also $\prod_{n \in I} A_n \subseteq \bigcup_{(b_n) \in \prod_{n \in I} I_n} \prod_{n \in I} B^n_{b_n}$.
 
 Ist $(a_n)_{n \in I} \in \bigcup_{(b_n) \in \prod_{n \in I} I_n} \prod_{n \in \N} B^n_{b_n}$, so gibt es $(b_n)_{n \in I} \in \prod_{n \in I}$ so dass $a_n \in B^n_{b_n} \subseteq A_n$ für alle $n \in I$, und somit $(a_n)_{n \in I} \in \prod_{n \in I} A_n$. Also ist $\prod_{n \in I} A_n \supseteq \bigcup_{(b_n) \in \prod_{n \in I} I_n}\ \prod_{n \in I} B^n_{b_n}$.
 
 Die Disjunktheit der Vereinigung ergibt sich aus der paarweisen Disjunktheit der $B^m_n$, $m \in I_n$, für alle $n \in I$: Sind $(b_n)_{n \in I}, (b'_n)_{n \in I} \in \prod_{n \in I} I_n$ so dass ein $(a_n)_{n \in I} \in \prod_{n \in I} B^n_{b_n} \cap \prod_{n \in I} B^n_{b'_n}$ existiert, so muss $a_n \in B^n_{b_n}$ und $a_n \in B^n_{b'_n}$, für alle $n \in I$. Also muss $B^n_{b_n} \cap B^n_{b'_n} \neq \emptyset$, und damit $b_n = b'_n$ für alle $n \in I$.
\end{proof}





\section{(Volumen auf Quadern)}


\subsection{}
Seien $q, Q \in \mc{Q}$ mit $q \subseteq Q$ beliebig aber fest. Es ist $Q = \prod_{i=1}^n [A_i,B_i)$ mit $A_i \leq B_i$ für $i=1,\ldots,n$ und $q = \prod_{i=1}^n [a_i,b_i)$ mit $A_i \leq a_i \leq b_i \leq B_i$ für $i=1,\ldots,n$. Dass $[a_i,b_i) \subseteq [A_i, B_i)$ folgt daher, dass sich wegen $q \subseteq Q$ mithilfe der Projektion $\pi_i : \R^n \rightarrow \R, (x_1, \ldots, x_n) \mapsto x_i$ ergibt, dass $[a_i,b_i) = \pi_i(q) \leq \pi(Q) = [A_i,B_i)$ für je $i=1, \ldots, n$. Mit Lemma \eqref{produktdisjunkt} ergibt sich nun, dass
\begin{align*}
 Q
 &= \prod_{i=1}^n [A_i,B_i)
 = \prod_{i=1}^n \left( \underbrace{[A_i,a_i)}_{:= C^i_1}\ \dcup\ \underbrace{[a_i,b_i)}_{:= C^i_2}\ \dcup\ \underbrace{[b_i,B_i)}_{:= C^i_3}\ \right) \\
 &= \dot{\bigcup_{(c_1, \ldots, c_n) \in \{1,2,3\}^n}} \prod_{i=1}^n C^i_{c_i}
\end{align*}
die disjunkte Vereinigung von $3^n$ Quadern $q_1, \ldots, q_{3^n}$ aus $\mc{Q}$ ist, wobei $q = \prod_{i=1}^n C^i_2$ einer von diesen ist. Aufgrund der Additivität von $\mu$ ist daher
\[
 \mu(Q)
 = \mu\left(\bigcup_{i=1}^{3^n} q_i\right)
 = \sum_{i=1}^{3^n} \mu(q_i)
 \geq \mu(q),
\]
weshalb $\mu$ monoton ist.





\subsection{}
Wegen der Invarianz von $\mu$ kann für alle $Q \in \mc{Q}$ o.b.d.A. davon ausgegangen werden, dass $Q = \prod_{i=1}^n [0,a_i)$ mit $a_i \in \R_{\geq 0}$ für $i = 1,\ldots,n$. Auch folgt aus der Invarianz von $\mu$ bezüglich Transposition, der endlichen Additivität von $\mu$ und Lemma \ref{produktdisjunkt}, dass für alle $Q = \prod_{i=1}^n [0,a_i) \in \mc{Q}$ und $m_1, \ldots, m_n \in \N$
\[
 \mu\left( \prod_{i=1}^n [0,m_i a_i) \right) = m_1 \cdots m_n\ \mu(Q),
\]
da
\begin{align*}
 &\, \mu\left( \prod_{i=1}^n [0,m_i a_i) \right)
 = \mu\left( \prod_{i=1}^n \dot{\bigcup}_{j=1}^{m_i} \underbrace{[(j-1) a_i,j a_i)}_{:=C^i_j} \right) \\
 &= \mu\left( \dot{\bigcup}_{(c_1, \ldots, c_n) \in \prod_{i=1}^n \{1,\ldots,m_i\}} \prod_{i=1}^n C^i_{c_i} \right) 
 = \sum_{(c_1, \ldots, c_n) \in \prod_{i=1}^n \{1,\ldots,m_i\}} \mu\left(\prod_{i=1}^n C^i_{c_i}\right) \\
 &= \sum_{(c_1, \ldots, c_n) \in \prod_{i=1}^n \{1,\ldots,m_i\}} \mu\left(\prod_{i=1}^n [(c_i-1) a_i,c_i a_i)\right)
 = \sum_{(c_1, \ldots, c_n) \in \prod_{i=1}^n \{1,\ldots,m_i\}} \mu(Q) \\
 &= m_1 \cdots m_n\ \mu(Q).
\end{align*}
Insbesondere folgt daher aus der Normierung, dass $\mu\left(\prod_{i=1}^n [0,m_i)\right) = \prod_{i=1}^n m_i$ für alle $m_1, \ldots, m_n \in \N$. Sei nun $Q = \prod_{i=1}^n [0,a_i) \in \mc{Q}$ beliebig aber fest. Für alle $m \in \N$ gilt
\[
 \mu\left(\prod_{i=1}^n [0,ma_i)\right) = m^n \mu\left(\prod_{i=1}^n [0,a_i)\right) = m^n \mu(Q),
\]
sowie wegen der Monotonie von $\mu$ auch
\begin{align*}
 &\prod_{i=1}^n \floor{ma_i}
 = \mu\left(\prod_{i=1}^n [0,\floor{ma_i})\right)
 \leq \mu\left(\prod_{i=1}^n [0,ma_i)\right) \text{ und}\\
 &\mu\left(\prod_{i=1}^n [0,ma_i)\right)
 \leq \mu\left(\prod_{i=1}^n [0,\ceil{ma_i})\right)
 = \prod_{i=1}^n \ceil{ma_i},
\end{align*}
also für $m \geq 1$
\begin{equation}\label{eq:rundungsprodukte}
 \prod_{i=1}^n \frac{\floor{ma_i}}{m}
 = \frac{1}{m^n}\prod_{i=1}^n \floor{ma_i}
 \leq \mu(Q)
 \leq \frac{1}{m^n}\prod_{i=1}^n \ceil{ma_i}
 = \prod_{i=1}^n \frac{\ceil{ma_i}}{m}.
\end{equation}
Nun ist $\limes{m}{\infty} \frac{\floor{mx}}{m} = m$ für alle $x \in \R$, da
\[
 x-\frac{1}{m} = \frac{mx-1}{m} \leq \frac{\floor{mx}}{m} \leq \frac{mx}{m} = x
\]
für alle $m \geq 1$ und daher $x \leq \limes{m}{\infty} \frac{\floor{mx}}{m} \leq x$, also $\limes{m}{\infty} \frac{\floor{mx}}{m} = x$. Analog ergibt sich, dass auch $\limes{m}{\infty} \frac{\ceil{mx}}{m} = x$. Aus \eqref{eq:rundungsprodukte} ergibt sich damit, dass
\begin{align*}
 &\prod_{i=1}^n a_i
 = \prod_{i=1}^n \limes{m}{\infty} \frac{\floor{ma_i}}{m}
 = \limes{m}{\infty} \prod_{i=1}^n \frac{\floor{ma_i}}{m}
 \leq \mu(Q) \text{ und} \\
 &\mu(Q)
 \leq \limes{m}{\infty} \prod_{i=1}^n \frac{\ceil{ma_i}}{m}
 \leq \prod_{i=1}^n \limes{m}{\infty} \frac{\ceil{ma_i}}{m}
 = \prod_{i=1}^n a_i,
\end{align*}
also $\mu(Q) = \prod_{i=1}^n a_i = \prod_{i=1}^n (a_i-0)$.










\section{(Nullmengen)}


\subsection{}
Aus der Aufgabenstellung geht meiner Meinung nach nicht klar hervor, ob $A$ als
\begin{align*}
 \tilde{A}_1 := \{x \in X: x \in A_k \text{ für unendlich viele } k \in \N\} \text{ oder als}\\
 \tilde{A}_2 := \{x \in X: x \in M \text{ für unendlich viele } M \in \{A_k\}_{k \in \N}  \}
\end{align*}
definiert wird.
Für den Beweis diesen Aufgabenteiles genügt es allerdings davon auszugehen, dass $A = \tilde{A}_1$, da $\tilde{A}_2 \subseteq \tilde{A}_1$: Ist $x \in \tilde{A}_2$, so gibt es unendlich viele $M \in \{A_k\}_{k \in \N}$ mit $x \in M$; insbesondere gibt es daher unendlich viele $n \in \N$ mit $x \in A_n$. Man bemerke jedoch, dass im Allgemeinen $\tilde{A}_1 \subsetneq \tilde{A}_2$: Ist etwa $(A_n)_{n \in \N}$ konstant mit $A_1 \neq \emptyset$, so ist $\tilde{A}_1 = A_1 \neq \emptyset$, aber $\tilde{A}_2 = \emptyset$.

Sei $\varepsilon > 0$ beliebig aber fest. Da $\mu(A_k) \geq 0$ für alle $k \in \N$ ist die Folge der Partialsummen $\left(\sum_{k=1}^n \mu(A_k)\right)_{n \in \N}$ monoton steigend. Da sie nach Annahme nach oben beschränkt ist, ist sie konvergent. Es gibt daher ein $N \in \N$ mit $\sum_{k=N}^\infty A_k < \varepsilon$. Da $\mu$ ein Maß ist folgt daraus, dass \[ \mu\left( \bigcup_{k = N}^\infty A_k \right) \leq \sum_{k=N}^\infty \mu(A_k) < \varepsilon. \] Da $A = \tilde{A}_1$ gibt es für alle $x \in A$ ein $n \in \N$ mit $n > N$ und $x \in A_n$. Insbesondere ist daher $A \subseteq \bigcup_{k=N}^\infty A_k$. Aufgrund der Monotonie von $\mu$ folgt, dass
\[
 0 \leq \mu(A) \leq \mu\left( \bigcup_{k=N}^\infty A_k \right) < \varepsilon.
\]
Aus der Beliebigkeit von $\varepsilon$ folgt damit, dass $\mu(A) = 0$.


\subsection{}
Es sei $(X,\A) := (\N,\mc{P}(\N))$ und $\mu$ das Zählmaß auf $\mc{P}(\N)$, sowie $A_k := \{1,\ldots,k\}$ für alle $k \in \N$. Es ist dann $\{1\} \subseteq A$, unabhängig davon ob $A = \tilde{A}_1$ oder $A = \tilde{A}_2$. Wegen der Monotonie des Maßes ist daher $\mu(A) \geq \mu(\{1\}) = 1$, also $\mu(A) \neq 0$. (Ist $A = \tilde{A}_1$, so ist sogar $A = \N$ und somit $\mu(A) = \infty$.)





\section{(Vom äußeren Maß zum Maß)}





\section{(Verschiedene äußere Maße)}





\end{document}
