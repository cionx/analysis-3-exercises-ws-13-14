\documentclass[a4paper,10pt]{article}
%\documentclass[a4paper,10pt]{scrartcl}

\usepackage{xltxtra}
\usepackage[ngerman]{babel}
\usepackage{amsmath}
\usepackage{amssymb}
\usepackage{amsthm}
\usepackage{mathtools}
\usepackage{nicefrac}
\usepackage{enumerate}
\usepackage{leftidx}

\makeatletter
\g@addto@macro\th@definition{\thm@headpunct{:}}
\makeatother
\theoremstyle{definition}
\newtheorem*{lem}{Lemma}
\newtheorem*{beh}{Behauptung}
\newtheorem*{bem}{Bemerkung}
\newtheorem*{ia}{Induktionsanfang}
\newtheorem*{is}{Induktionsschritt}

\renewcommand{\thesection}{Aufgabe \arabic{section}.}
\renewcommand{\thesubsection}{\alph{subsection})}
\renewcommand{\thesubsubsection}{(\roman{subsubsection})}

\newcommand{\N}{\mathbb{N}}
\newcommand{\Z}{\mathbb{Z}}
\newcommand{\Q}{\mathbb{Q}}
\newcommand{\R}{\mathbb{R}}
\newcommand{\C}{\mathbb{C}}
\newcommand{\A}{\mathcal{A}}
\newcommand{\La}{\mathcal{L}}
\newcommand{\dx}{\,\text{d}x}
\newcommand{\dy}{\,\text{d}y}
\newcommand{\dt}{\,\text{d}t}
\newcommand{\du}{\,\text{d}u}
\newcommand{\Img}{\operatorname{Im}}
\newcommand{\Real}{\operatorname{Re}}
\newcommand{\Imag}{\operatorname{Im}}
\newcommand{\sgn}{\operatorname{sgn}}
\newcommand{\dcup}{\dot{\cup}}
\newcommand{\mc}[1]{\mathcal{#1}}
\newcommand{\limes}[2]{\lim_{#1 \rightarrow #2}}
\newcommand{\limessup}[1]{\limsup_{#1 \rightarrow \infty}}
\newcommand{\limesinf}[1]{\liminf_{#1 \rightarrow \infty}}
\newcommand{\vect}[1]{\begin{pmatrix}#1\end{pmatrix}}
\newcommand{\partd}[2]{\frac{\partial #1}{\partial #2}}
\newcommand{\op}[1]{\left\|#1\right\|_{\text{op}}}

\makeatletter
\renewcommand*\env@matrix[1][*\c@MaxMatrixCols c]{%
  \hskip -\arraycolsep
  \let\@ifnextchar\new@ifnextchar
  \array{#1}}
\makeatother

\setromanfont[Mapping=tex-text]{Linux Libertine O}
% \setsansfont[Mapping=tex-text]{DejaVu Sans}
% \setmonofont[Mapping=tex-text]{DejaVu Sans Mono}
\parindent 0pt

\title{\sc Analysis III \\ \Large 3. Aufgabenblatt}
\author{Jendrik Stelzner}
\date{\today}

\begin{document}
\maketitle





\section{(Volumen auf Quadern)}
Im Folgenden nutze ich für die Familie halboffener Quader in $\R^n$ die Notation
\[
 \mc{Q}_n := \left\{\prod_{i=1}^n [a_i, b_i) : a_i \leq b_i \text{ für } i=1,\ldots,n \right\} \subseteq \mc{P}(\R^n)
\]
und bezeichne mit $\mu_n$ die entsprechende Abbildung auf $\mc{Q}_n$.


\subsection{}
Für alle $n \in \N, n \geq 1$ gibt es für alle $q, Q \in \mc{Q}_n$ mit $q \subseteq Q$ paarweise disjunkte Quader $q_1, \ldots, q_{2n+1} \in \mc{Q}_n$ mit $q = q_i$ für ein $i \in \{1, \ldots, 2n+1\}$ und $Q = \bigcup_{k=1}^{2n+1} q_k$.

\begin{proof}[Beweis:]
 Die Aussage lässt sich per Induktion über $n$ zeigen.
 \begin{ia}
  Sei $n = 1$. Dann ist $Q = [A,B)$ mit $A \leq B$ und $q = [a,b)$ mit $A \leq a \leq b \leq B$. Es ist dann $Q = [A,a)\ \dcup\ q\ \dcup\ [b,B)$.
 \end{ia}
 \begin{is}
  Sei $n > 2$ und gelte die Aussage für $n-1$. Es ist $Q = \prod_{i=1}^n [A_i,B_i)$ mit $A_i \leq B_i$ für $i=1,\ldots,n$ und $q = \prod_{i=1}^n [a_i,b_i)$ mit $A_i \leq a_i \leq b_i \leq B_i$ für $i=1,\ldots,n$. Es sei $Q' := \prod_{i=1}^{n-1} [A_i, B_i)$ und $q' := \prod_{i=1}^{n-1} [a_i, b_i)$. Da $q \subseteq Q$ ist $q' \subseteq Q'$, nach Induktionsvoraussetzung gibt es daher paarweise disjunkte Quader $q'_1, \ldots, q'_{2n-1} \in \mc{Q}_{n-1}$ mit $Q' = \bigcup_{i=1}^{2n-1} q'_1$ und $q' = q'_j$ für ein $j \in \{1, \ldots, n-1\}$. Für $i=1,\ldots,2n-1$ sei $q_i := q'_i \times [A_n,B_n) \in \mc{Q}_n$. Es ist $Q = \bigcup_{i=1}^{2n-1} q_i$, und aus der Disjunktheit der $q'_i$ folgt die Disjunktheit der $q_i$. Mit $q_{2n} := q'_j \times [A_n, a_n)$ und $q_{2n+1} := q'_j \times [b_n, B_n)$ ist $q_j = q_{2n}\ \dcup\ q\ \dcup\ q_{2n+1}$, also $Q = q \cup \bigcup_{\substack{i=1\\i \neq j}}^{2n+1} q_i$ als disjunkte Vereinigung.\qedhere
 \end{is}
\end{proof}

Sei $n \in \N$ beliebig aber fest und seien $Q, q \in \mc{Q}_n$ mit $q \subseteq Q$. Wie oben gezeigt gibt es paarweise disjunkte Quader $q_1, \ldots, q_{2n+1} \in \mc{Q}_n$ mit $Q = \bigcup_{i=1}^{2n+1} q_i$ und $q = q_i$ für ein $i \in \{1, \ldots, 2n+1\}$. Aufgrund der Additivität von $\mu_n$ ist daher
\[
 \mu_n(Q)
 = \mu_n\left(\bigcup_{i=1}^{2n+1} q_i\right)
 = \sum_{i=1}^{2n+1} \mu_n(q_i)
 \geq \mu_n(q),
\]
d.h. $\mu_n$ ist monoton.





\section{(Nullmengen)}


\subsection{}
Aus der Aufgabenstellung geht meiner Meinung nach nicht klar hervor, ob $A$ als
\begin{align*}
 \tilde{A}_1 := \{x \in X: x \in A_k \text{ für unendlich viele } k \in \N\} \text{ oder als}\\
 \tilde{A}_2 := \{x \in X: x \in M \text{ für unendlich viele } M \in \{A_k\}_{k \in \N}  \}
\end{align*}
definiert wird.
Für den Beweis diesen Aufgabenteiles genügt es allerdings davon auszugehen, dass $A = \tilde{A}_1$, da $\tilde{A}_2 \subseteq \tilde{A}_1$: Ist $x \in \tilde{A}_2$, so gibt es unendlich viele $M \in \{A_k\}_{k \in \N}$ mit $x \in M$; insbesondere gibt es daher unendlich viele $n \in \N$ mit $x \in A_n$. Man bemerke jedoch, dass im Allgemeinen $\tilde{A}_1 \subsetneq \tilde{A}_2$: Ist etwa $(A_n)_{n \in \N}$ konstant mit $A_1 \neq \emptyset$, so ist $\tilde{A}_1 = A_1 \neq \emptyset$, aber $\tilde{A}_2 = \emptyset$.

Sei $\varepsilon > 0$ beliebig aber fest. Da $\mu(A_k) \geq 0$ für alle $k \in \N$ ist die Folge der Partialsummen $\left(\sum_{k=1}^n \mu(A_k)\right)_{n \in \N}$ monoton steigend. Da sie nach Annahme nach oben beschränkt ist, ist sie konvergent. Es gibt daher ein $N \in \N$ mit $\sum_{k=N}^\infty A_k < \varepsilon$. Da $\mu$ ein Maß ist folgt daraus, dass \[ \mu\left( \bigcup_{k = N}^\infty A_k \right) \leq \sum_{k=N}^\infty \mu(A_k) < \varepsilon. \] Da $A = \tilde{A}_1$ gibt es für alle $x \in A$ ein $n \in \N$ mit $n > N$ und $x \in A_n$. Insbesondere ist daher $A \subseteq \bigcup_{k=N}^\infty A_k$. Aufgrund der Monotonie von $\mu$ folgt, dass
\[
 0 \leq \mu(A) \leq \mu\left( \bigcup_{k=N}^\infty A_k \right) < \varepsilon.
\]
Aus der Beliebigkeit von $\varepsilon$ folgt damit, dass $\mu(A) = 0$.


\subsection{}
Es sei $(X,\A) := (\N,\mc{P}(\N))$ und $\mu$ das Zählmaß auf $\mc{P}(\N)$, sowie $A_k := \{1,\ldots,k\}$ für alle $k \in \N$. Es ist dann $\{1\} \subseteq A$, unabhängig davon ob $A = \tilde{A}_1$ oder $A = \tilde{A}_2$. Wegen der Monotonie des Maßes ist daher $\mu(A) \geq \mu(\{1\}) = 1$, also $\mu(A) \neq 0$. (Ist $A = \tilde{A}_1$, so ist sogar $A = \N$ und somit $\mu(A) = \infty$.)





\section{(Vom äußeren Maß zum Maß)}





\section{(Verschiedene äußere Maße)}





\end{document}
