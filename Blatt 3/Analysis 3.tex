\documentclass[a4paper,10pt]{article}
%\documentclass[a4paper,10pt]{scrartcl}

\usepackage{xltxtra}
\usepackage[ngerman]{babel}
\usepackage{amsmath}
\usepackage{amssymb}
\usepackage{amsthm}
\usepackage{mathtools}
\usepackage{nicefrac}
\usepackage{enumerate}
\usepackage{leftidx}

\theoremstyle{definition}
\newtheorem*{lem}{Lemma}
\newtheorem*{beh}{Behauptung}
\newtheorem*{bem}{Bemerkung}
\newtheorem*{ia}{Induktionsanfang}
\newtheorem*{is}{Induktionsschritt}

\renewcommand{\thesection}{Aufgabe \arabic{section}.}
\renewcommand{\thesubsection}{\alph{subsection})}
\renewcommand{\thesubsubsection}{(\roman{subsubsection})}

\newcommand{\N}{\mathbb{N}}
\newcommand{\Z}{\mathbb{Z}}
\newcommand{\Q}{\mathbb{Q}}
\newcommand{\R}{\mathbb{R}}
\newcommand{\C}{\mathbb{C}}
\newcommand{\A}{\mathcal{A}}
\newcommand{\La}{\mathcal{L}}
\newcommand{\dx}{\,\text{d}x}
\newcommand{\dy}{\,\text{d}y}
\newcommand{\dt}{\,\text{d}t}
\newcommand{\du}{\,\text{d}u}
\newcommand{\mc}[1]{\mathcal{#1}}
\newcommand{\Img}{\operatorname{Im}}
\newcommand{\Real}{\operatorname{Re}}
\newcommand{\Imag}{\operatorname{Im}}
\newcommand{\sgn}{\operatorname{sgn}}
\newcommand{\limes}[2]{\lim_{#1 \rightarrow #2}}
\newcommand{\limessup}[1]{\limsup_{#1 \rightarrow \infty}}
\newcommand{\limesinf}[1]{\liminf_{#1 \rightarrow \infty}}
\newcommand{\vect}[1]{\begin{pmatrix}#1\end{pmatrix}}
\newcommand{\partd}[2]{\frac{\partial #1}{\partial #2}}
\newcommand{\op}[1]{\left\|#1\right\|_{\text{op}}}

\makeatletter
\renewcommand*\env@matrix[1][*\c@MaxMatrixCols c]{%
  \hskip -\arraycolsep
  \let\@ifnextchar\new@ifnextchar
  \array{#1}}
\makeatother

\setromanfont[Mapping=tex-text]{Linux Libertine O}
% \setsansfont[Mapping=tex-text]{DejaVu Sans}
% \setmonofont[Mapping=tex-text]{DejaVu Sans Mono}
\parindent 0pt

\title{Analysis 3 — Übung 3}
\author{Jendrik Stelzner}
\date{\today}

\begin{document}
\maketitle





\section{(Volumen auf Quadern)}





\section{(Nullmengen)}


\subsection{}
Aus der Aufgabenstellung geht meiner Meinung nach nicht klar hervor, ob $A$ als
\begin{align*}
 \tilde{A}_1 := \{x \in X: x \in A_k \text{ für unendlich viele } k \in \N\} \text{ oder als}\\
 \tilde{A}_2 := \{x \in X: x \in M \text{ für unendlich viele } M \in \{A_k\}_{k \in \N}  \}
\end{align*}
definiert wird.
Für den Beweis diesen Aufgabenteiles genügt es allerdings davon auszugehen, dass $A = \tilde{A}_1$, da $\tilde{A}_2 \subseteq \tilde{A}_1$: Ist $x \in \tilde{A}_2$, so gibt es unendlich viele $M \in \{A_k\}_{k \in \N}$ mit $x \in M$; insbesondere gibt es daher unendlich viele $n \in \N$ mit $x \in A_n$. Man bemerke jedoch, dass im Allgemeinen $\tilde{A}_1 \subsetneq \tilde{A}_2$: Ist etwa $(A_n)_{n \in \N}$ konstant mit $A_1 \neq \emptyset$, so ist $\tilde{A}_1 = A_1 \neq \emptyset$, aber $\tilde{A}_2 = \emptyset$.

Sei $\varepsilon > 0$ beliebig aber fest. Da $\mu(A_k) \geq 0$ für alle $k \in \N$ ist die Folge der Partialsummen $\left(\sum_{k=1}^n \mu(A_k)\right)_{n \in \N}$ monoton steigend. Da sie nach Annahme nach oben beschränkt ist, ist sie konvergent. Es gibt daher ein $N \in \N$ mit $\sum_{k=N}^\infty A_k < \varepsilon$. Da $\mu$ ein Maß ist folgt daraus, dass \[ \mu\left( \bigcup_{k = N}^\infty A_k \right) \leq \sum_{k=N}^\infty \mu(A_k) < \varepsilon. \] Da $A = \tilde{A}_1$ gibt es für alle $x \in A$ ein $n \in \N$ mit $n > N$ und $x \in A_n$. Insbesondere ist daher $A \subseteq \bigcup_{k=N}^\infty A_k$. Aufgrund der Monotonie von $\mu$ folgt, dass
\[
 0 \leq \mu(A) \leq \mu\left( \bigcup_{k=N}^\infty A_k \right) < \varepsilon.
\]
Aus der Beliebigkeit von $\varepsilon$ folgt damit, dass $\mu(A) = 0$.


\subsection{}
Es sei $(X,\A) := (\N,\mc{P}(\N))$ und $\mu$ das Zählmaß auf $\mc{P}(\N)$, sowie $A_k := \{1,\ldots,k\}$ für alle $k \in \N$. Es ist dann $\{1\} \subseteq A$, unabhängig davon ob $A = \tilde{A}_1$ oder $A = \tilde{A}_2$. Wegen der Monotonie des Maßes ist daher $\mu(A) \geq \mu(\{1\}) = 1$, also $\mu(A) \neq 0$. (Ist $A = \tilde{A}_1$, so ist sogar $A = \N$ und somit $\mu(A) = \infty$.)





\section{(Vom äußeren Maß zum Maß)}





\section{(Verschiedene äußere Maße)}





\end{document}
