\documentclass[a4paper,10pt]{article}
%\documentclass[a4paper,10pt]{scrartcl}

\usepackage{xltxtra}
\usepackage[ngerman]{babel}
\usepackage{amsmath}
\usepackage{amssymb}
\usepackage{amsthm}
\usepackage{mathtools}
\usepackage{nicefrac}
\usepackage{enumerate}
\usepackage{leftidx}

\makeatletter
\g@addto@macro\th@definition{\thm@headpunct{:}}
\makeatother
\theoremstyle{definition}
\newtheorem{lem}{Lemma}
\newtheorem{beh}{Behauptung}
\newtheorem{bem}{Bemerkung}
\newtheorem*{ia}{Induktionsanfang}
\newtheorem*{is}{Induktionsschritt}

\renewcommand{\thesection}{Aufgabe \arabic{section}.}
\renewcommand{\thesubsection}{\alph{subsection})}
\renewcommand{\thesubsubsection}{(\roman{subsubsection})}

\makeatletter
\def\moverlay{\mathpalette\mov@rlay}
\def\mov@rlay#1#2{\leavevmode\vtop{%
   \baselineskip\z@skip \lineskiplimit-\maxdimen
   \ialign{\hfil$\m@th#1##$\hfil\cr#2\crcr}}}
\newcommand{\charfusion}[3][\mathord]{
    #1{\ifx#1\mathop\vphantom{#2}\fi
        \mathpalette\mov@rlay{#2\cr#3}
      }
    \ifx#1\mathop\expandafter\displaylimits\fi}
\makeatother

\newcommand{\N}{\mathbb{N}}
\newcommand{\Z}{\mathbb{Z}}
\newcommand{\Q}{\mathbb{Q}}
\newcommand{\R}{\mathbb{R}}
\newcommand{\C}{\mathbb{C}}
\newcommand{\A}{\mathcal{A}}
\newcommand{\La}{\mathcal{L}}
\newcommand{\dx}{\,\text{d}x}
\newcommand{\dy}{\,\text{d}y}
\newcommand{\dt}{\,\text{d}t}
\newcommand{\du}{\,\text{d}u}
\newcommand{\Img}{\operatorname{Im}}
\newcommand{\Real}{\operatorname{Re}}
\newcommand{\Imag}{\operatorname{Im}}
\newcommand{\sgn}{\operatorname{sgn}}
\newcommand{\dotcup}{\ensuremath{\mathaccent\cdot\cup}}
\newcommand{\bigdotcup}{\charfusion[\mathop]{\bigcup}{\cdot}}
\newcommand{\ceil}[1]{\left\lceil{#1}\right\rceil}
\newcommand{\floor}[1]{\left\lfloor{#1}\right\rfloor}
\newcommand{\mc}[1]{\mathcal{#1}}
\newcommand{\limes}[2]{\lim_{#1 \rightarrow #2}}
\newcommand{\limessup}[1]{\limsup_{#1 \rightarrow \infty}}
\newcommand{\limesinf}[1]{\liminf_{#1 \rightarrow \infty}}
\newcommand{\vect}[1]{\begin{pmatrix}#1\end{pmatrix}}
\newcommand{\partd}[2]{\frac{\partial #1}{\partial #2}}
\newcommand{\op}[1]{\left\|#1\right\|_{\text{op}}}

\makeatletter
\renewcommand*\env@matrix[1][*\c@MaxMatrixCols c]{%
  \hskip -\arraycolsep
  \let\@ifnextchar\new@ifnextchar
  \array{#1}}
\makeatother

\setromanfont[Mapping=tex-text]{Linux Libertine O}
% \setsansfont[Mapping=tex-text]{DejaVu Sans}
% \setmonofont[Mapping=tex-text]{DejaVu Sans Mono}
\parindent 0pt

\title{\sc Analysis III \\ \Large 3. Aufgabenblatt}
\author{Jendrik Stelzner}
\date{\today}

\begin{document}
\maketitle



Betrachtet man zwei Mengen $A$ und $A'$ mit $A = B_1\ \dotcup\ B_2$ und $A' = B'_1\ \dotcup\ B'_2$ für $B_1, B_2 \subseteq A$ und $B'_1, B'_2 \subseteq A'$, so ist es einfach so sehen, dass
\[
 A \times A' = (B_1 \times B'_1)\ \dotcup\ (B_1 \times B'_2)\ \dotcup\ (B_2 \times B'_1)\ \dotcup\ (B_2 \times B'_2).
\]
Diese Beobachtung wird von dem folgenden Lemma verallgemeinert:

\begin{lem} \label{produktdisjunkt} 
 Sei $(A_i)_{i \in I}$ eine Familie von Mengen über einer Indexmenge $I$ und
 \[
  A_i = \bigdotcup_{j \in J_i} B^i_j
 \]
für eine Familie von Mengen $(B^i_j)_{j \in J_i}$ über einer Indexmenge $J_i$ für alle $i \in I$. Dann gilt
 \[
  \prod_{i \in I} A_i
  = \prod_{i \in I} \bigdotcup_{j \in J_i} B^i_j
  = \bigdotcup_{(j_i) \in \prod_{i \in I} J_i}\ \prod_{i \in I} B^i_{j_i}.
 \]
\end{lem}
\begin{proof}[Beweis des Lemmas:]
 Sei $(a_i)_{i \in I} \in \prod_{i \in I} A_i$. Für alle $i \in I$ gibt es wegen $a_i \in A_i$ ein $j_i \in J_i$ mit $a_i \in B^i_{j_i}$. Daher ist
 \[
  (a_i)_{i \in I} \in \prod_{i \in I} B^i_{j_i}
  \subseteq \bigcup_{(j_i) \in \prod_{i \in I} J_i}\ \prod_{i \in I} B^i_{j_i},
 \]
 also ist
 \[
  \prod_{i \in I} A_i \subseteq \bigcup_{(j_i) \in \prod_{i \in I} J_i} \prod_{i \in I} B^i_{j_i}.
 \]
 
 Ist
 \[
  (a_i)_{i \in I} \in \bigcup_{(j_i) \in \prod_{i \in I} J_i} \prod_{i \in I} B^i_{j_i},
 \]
 so gibt es $(j_i)_{i \in I} \in \prod_{i \in I} J_i$ mit $a_i \in B^i_{j_i} \subseteq A_i$ für alle $i \in I$, also $(a_i)_{i \in I} \in \prod_{i \in I} A_i$. Daher ist
 \[
  \prod_{i \in I} A_i \supseteq \bigcup_{(j_i) \in \prod_{i \in I} J_i}\ \prod_{i \in I} B^i_{j_i}.
 \]
 
 Die Disjunktheit der Vereinigung ergibt sich aus der paarweisen Disjunktheit der $B^i_j$, $j \in J_i$, für alle $i \in I$: Sind $(j_i)_{i \in I}, (j'_i)_{i \in I} \in \prod_{i \in I} J_i$ so dass
 \[
  (a_i)_{i \in I} \in \prod_{i \in I} B^i_{j_i} \cap\ \prod_{i \in I} B^i_{j'_i}
 \]
 existiert, so muss $a_i \in B^i_{j_i}$ und $a_i \in B^i_{j'_i}$ für alle $i \in I$. Also muss $B^i_{j_i} \cap B^i_{j'_i} \neq \emptyset$, und damit $j_i = j'_i$ für alle $i \in I$.
\end{proof}





\section{(Volumen auf Quadern)}


\subsection{}
Seien $q, Q \in \mc{Q}$ mit $q \subseteq Q$ beliebig aber fest. Es ist $Q = \prod_{i=1}^n [A_i,B_i)$ mit $A_i \leq B_i$ für $i=1,\ldots,n$, und $q = \prod_{i=1}^n [a_i,b_i)$ mit $A_i \leq a_i \leq b_i \leq B_i$ für $i=1,\ldots,n$. Dass $[a_i,b_i) \subseteq [A_i, B_i)$ folgt für $i=1,\ldots,n$ daher, dass sich aus $q \subseteq Q$ für die Projektion
\[
 \pi_i : \R^n \rightarrow \R, (x_1, \ldots, x_n) \mapsto x_i
\]
ergibt, dass
\[
 [a_i,b_i) = \pi_i(q) \subseteq \pi(Q) = [A_i,B_i).
\]
Mit Lemma \ref{produktdisjunkt} ergibt sich nun, dass
\begin{align*}
 Q
 &= \prod_{i=1}^n [A_i,B_i)
 = \prod_{i=1}^n \left( \underbrace{[A_i,a_i)}_{:= C^i_1}\ \dotcup\ \underbrace{[a_i,b_i)}_{:= C^i_2}\ \dotcup\ \underbrace{[b_i,B_i)}_{:= C^i_3} \right) \\
 &= \bigdotcup_{(c_1, \ldots, c_n) \in \{1,2,3\}^n} \prod_{i=1}^n C^i_{c_i}
\end{align*}
die disjunkte Vereinigung von $3^n$ Quadern $q_1, \ldots, q_{3^n} \in \mc{Q}$ ist, wobei $q = \prod_{i=1}^n C^i_2$ einer von diesen ist. Aufgrund der Additivität von $\mu$ ist daher
\[
 \mu(Q)
 = \mu\left(\bigdotcup_{i=1}^{3^n} q_i\right)
 = \sum_{i=1}^{3^n} \mu(q_i)
 \geq \mu(q),
\]
was die Monotonie von $\mu$ zeigt.


\subsection{}
Wegen der Invarianz von $\mu$ kann für alle $Q \in \mc{Q}$ o.b.d.A. davon ausgegangen werden, dass $Q = \prod_{i=1}^n [0,a_i)$ mit $a_i \in \R_{\geq 0}$ für $i = 1,\ldots,n$. Auch folgt aus der Invarianz von $\mu$ bezüglich Transposition, der endlichen Additivität von $\mu$ und Lemma \ref{produktdisjunkt}, dass für alle $Q = \prod_{i=1}^n [0,a_i) \in \mc{Q}$ und $m_1, \ldots, m_n \in \N$
\[
 \mu\left( \prod_{i=1}^n [0,m_i a_i) \right) = m_1 \cdots m_n\ \mu(Q),
\]
da
\begin{align*}
 &\, \mu\left( \prod_{i=1}^n [0,m_i a_i) \right)
 = \mu\left( \prod_{i=1}^n \bigdotcup_{j=1}^{m_i} \underbrace{[(j-1) a_i,j a_i)}_{:=C^i_j} \right) \\
 &= \mu\left( \bigdotcup_{(c_1, \ldots, c_n) \in \prod_{i=1}^n \{1,\ldots,m_i\}} \prod_{i=1}^n C^i_{c_i} \right) 
 = \sum_{(c_1, \ldots, c_n) \in \prod_{i=1}^n \{1,\ldots,m_i\}} \mu\left(\prod_{i=1}^n C^i_{c_i}\right) \\
 &= \sum_{(c_1, \ldots, c_n) \in \prod_{i=1}^n \{1,\ldots,m_i\}} \mu\left(\prod_{i=1}^n [(c_i-1) a_i,c_i a_i)\right)
 = \sum_{(c_1, \ldots, c_n) \in \prod_{i=1}^n \{1,\ldots,m_i\}} \mu(Q) \\
 &= m_1 \cdots m_n\ \mu(Q).
\end{align*}
Insbesondere folgt daher aus der Normierung, dass
\[
\mu\left(\prod_{i=1}^n [0,m_i)\right) = \prod_{i=1}^n m_i \text{ für alle } m_1, \ldots, m_n \in \N.
\]
Sei nun $Q = \prod_{i=1}^n [0,a_i) \in \mc{Q}$ beliebig aber fest. Für alle $m \in \N$ gilt
\[
 \mu\left(\prod_{i=1}^n [0,ma_i)\right) = m^n \mu\left(\prod_{i=1}^n [0,a_i)\right) = m^n \mu(Q),
\]
sowie wegen der Monotonie von $\mu$ auch
\begin{align*}
 &\prod_{i=1}^n \floor{ma_i}
 = \mu\left(\prod_{i=1}^n [0,\floor{ma_i})\right)
 \leq \mu\left(\prod_{i=1}^n [0,ma_i)\right) \text{ und}\\
 &\mu\left(\prod_{i=1}^n [0,ma_i)\right)
 \leq \mu\left(\prod_{i=1}^n [0,\ceil{ma_i})\right)
 = \prod_{i=1}^n \ceil{ma_i},
\end{align*}
also für $m \geq 1$
\begin{equation}\label{eq:rundungsprodukte}
 \prod_{i=1}^n \frac{\floor{ma_i}}{m}
 = \frac{1}{m^n}\prod_{i=1}^n \floor{ma_i}
 \leq \mu(Q)
 \leq \frac{1}{m^n}\prod_{i=1}^n \ceil{ma_i}
 = \prod_{i=1}^n \frac{\ceil{ma_i}}{m}.
\end{equation}
Nun ist $\limes{m}{\infty} \frac{\floor{mx}}{m} = m$ für alle $x \in \R$, da
\[
 x-\frac{1}{m} = \frac{mx-1}{m} \leq \frac{\floor{mx}}{m} \leq \frac{mx}{m} = x
\]
für alle $m \geq 1$ und daher
\[
x \leq \limes{m}{\infty} \frac{\floor{mx}}{m} \leq x,
\]
also $\limes{m}{\infty} \frac{\floor{mx}}{m} = x$. Analog ergibt sich, dass auch $\limes{m}{\infty} \frac{\ceil{mx}}{m} = x$. Aus \eqref{eq:rundungsprodukte} ergibt sich damit, dass
\begin{align*}
 &\prod_{i=1}^n a_i
 = \prod_{i=1}^n \limes{m}{\infty} \frac{\floor{ma_i}}{m}
 = \limes{m}{\infty} \prod_{i=1}^n \frac{\floor{ma_i}}{m}
 \leq \mu(Q) \text{ und} \\
 &\mu(Q)
 \leq \limes{m}{\infty} \prod_{i=1}^n \frac{\ceil{ma_i}}{m}
 \leq \prod_{i=1}^n \limes{m}{\infty} \frac{\ceil{ma_i}}{m}
 = \prod_{i=1}^n a_i,
\end{align*}
also $\mu(Q) = \prod_{i=1}^n a_i = \prod_{i=1}^n (a_i-0)$.





\section{(Nullmengen)}


\subsection{}
Aus der Aufgabenstellung geht meiner Meinung nach nicht klar hervor, ob $A$ als
\begin{align*}
 \tilde{A}_1 := \{x \in X: x \in A_k \text{ für unendlich viele } k \in \N\} \text{ oder als}\\
 \tilde{A}_2 := \left\{x \in X: x \in M \text{ für unendlich viele } M \in \{A_k\}_{k \in \N} \right\}
\end{align*}
definiert wird.
Für den Beweis diesen Aufgabenteiles genügt es allerdings davon auszugehen, dass $A = \tilde{A}_1$, da $\tilde{A}_2 \subseteq \tilde{A}_1$: Ist $x \in \tilde{A}_2$, so gibt es unendlich viele $M \in \{A_k\}_{k \in \N}$ mit $x \in M$; insbesondere gibt es daher unendlich viele $n \in \N$ mit $x \in A_n$. Man bemerke jedoch, dass im Allgemeinen $\tilde{A}_1 \subsetneq \tilde{A}_2$: Ist etwa $(A_n)_{n \in \N}$ konstant mit $A_1 \neq \emptyset$, so ist $\tilde{A}_1 = A_1 \neq \emptyset$, aber $\tilde{A}_2 = \emptyset$.

Zunächst gilt es zu zeigen, dass $A$ messbar ist, d.h. dass $A \in \A$. Dies ist der Fall, da
\begin{align*}
 A = \tilde{A}
 &= \{x \in X: x \in A_k \text{ für unendlich viele } k \in \N\} \\
 &= \{x \in X : \text{für alle } n \in \N \text{ gibt es } m \geq n \text{ mit } x \in A_m\} \\
 &= \bigcap_{n \in \N} \bigcup_{m \geq n} A_m \in \A.
\end{align*}

Sei $\varepsilon > 0$ beliebig aber fest. Da $\mu(A_k) \geq 0$ für alle $k \in \N$ ist die Folge der Partialsummen $\left(\sum_{k=1}^n \mu(A_k)\right)_{n \in \N}$ monoton steigend. Da sie nach Annahme nach oben beschränkt ist, ist sie konvergent. Es gibt daher ein $N \in \N$ mit $\sum_{k=N}^\infty A_k < \varepsilon$. Da $\mu$ ein Maß ist folgt daraus, dass \[ \mu\left( \bigcup_{k = N}^\infty A_k \right) \leq \sum_{k=N}^\infty \mu(A_k) < \varepsilon. \] Da $A = \tilde{A}_1$ gibt es für alle $x \in A$ ein $n \in \N$ mit $n > N$ und $x \in A_n$. Insbesondere ist daher $A \subseteq \bigcup_{k=N}^\infty A_k$. Aufgrund der Monotonie von $\mu$ folgt, dass
\[
 0 \leq \mu(A) \leq \mu\left( \bigcup_{k=N}^\infty A_k \right) < \varepsilon.
\]
Aus der Beliebigkeit von $\varepsilon$ folgt damit, dass $\mu(A) = 0$.


\subsection{}
Es sei $(X,\A) := (\N,\mc{P}(\N))$ und $\mu$ das Zählmaß auf $\mc{P}(\N)$, sowie $A_k := \{1,\ldots,k\}$ für alle $k \in \N$. Es ist dann $\{1\} \subseteq A$, unabhängig davon ob $A = \tilde{A}_1$ oder $A = \tilde{A}_2$. Wegen der Monotonie des Maßes ist daher
\[
\mu(A) \geq \mu(\{1\}) = 1 \neq 0.
\]





\section{(Vom äußeren Maß zum Maß)}


\subsection{}
Es ist $\mu^*(\emptyset) = 0$, da für die Folge $(A_k)_{k \in \N}$ definiert als $A_k := \emptyset \in \A$ für alle $k \in \N$ offenbar $\emptyset = \bigcup_{k \in \N} A_k$, und daher
\[
 \mu^*(\emptyset) \leq \sum_{k \in \N} \mu(A_k) = \sum_{k \in \N} 0 = 0.
\]
Dass $\mu^*(0) \geq 0$ folgt direkt daraus, dass $\mu$ nur nicht-negative Werte annimmt, und die Menge
\[
 \left\{ (A_k)_{k \in \N} \in \A^\N : \emptyset \subseteq \bigcup_{k \in \N} A_k \right\}
 = \A^\N
\]
nichtleer ist, da $\A$ als Algebra nicht leer ist.

Seien $A,B \in \mc{P}(X)$ mit $A \subseteq B$. Die Monotonie von $\mu^*$, dass also $\mu^*(A) \leq \mu^*(B)$, ergibt sich daraus, dass für alle $(B_k)_{k \in \N}$ mit $B_k \in \A$ für alle $k \in \N$ und $B \subseteq \bigcup_{k \in \N} B_k$ auch $A \subseteq B \subseteq \bigcup_{k \in \N} B_k$. Also ist
\begin{align*}
 S_B := &\left\{ (B_k)_{k \in \N} \in \A^\N : B \subseteq \bigcup_{k \in \N} B_k \right\} \\
 \subseteq &\left\{ (A_k)_{k \in \N} \in \A^\N : A \subseteq \bigcup_{k \in \N} A_k \right\} =: S_A,
\end{align*}
und daher
\[
 \mu^*(A)
 = \inf_{(A_k) \in S_A} \sum_{k \in \N} \mu(A_k)
 \leq \inf_{(B_k) \in S_B} \sum_{k \in \N} \mu(B_k)
 = \mu^*(B).
\]

Die abzählbare Subadditivität von $\mu^*$ ergibt sich wie folgt: Sei $\varepsilon > 0$ beliebig aber fest und $(A_k)_{k \in \N}$ eine Folge auf $\mc{P}(X)$. Es gibt nach der Definition von $\mu^*$ für alle $k \in \N$ eine Folge $(B^k_n)_{n \in \N}$ auf $\A$ mit $A_k \subseteq \bigcup_{n \in \N} B^k_n$ und
\[
 \sum_{n \in \N} \mu\left(B^k_n\right) \leq \mu^*(A)+\frac{\varepsilon}{2^{k+1}}.
\]
Da daher
\[
 \bigcup_{k \in \N} A_k \subseteq \bigcup_{k,n \in \N} B^k_n
\]
ist nach der Definition von $\mu^*$
\[
 \mu^*\left( \bigcup_{k \in \N} A_k \right)
 \leq \sum_{k,n \in \N} \mu\left( B^n_k \right)
 \leq \sum_{k,n \in \N} \left( \mu^*(A_k) + \frac{\varepsilon}{2^{k+1}} \right)
 = \left( \sum_{k \in \N} \mu^*(A_k) \right) + \varepsilon.
\]
Aus der Beliebigkeit von $\varepsilon > 0$ folgt damit, dass
\[
 \mu^*\left( \bigcup_{k \in \N} A_k \right)
 \leq \sum_{k \in \N} \mu^*(A_k).
\]


\subsection{}
Es gilt zu zeigen, dass für alle $A \subseteq X$ und $B \in \A$
\[
 \mu^*(A) \geq \mu^*(A \cap B) + \mu^*(A \cap B^c).
\]
Sei $\varepsilon > 0$ beliebig aber fest, $A \subseteq X$ und $B \in \A$. Nach der Definition von $\mu^*$ gibt es eine Folge $(A_k)_{k \in \N}$ auf $\A$ mit $A \subseteq \bigcup_{k \in \N} A_k$ und
\begin{equation} \label{eq:algebramessbarepsilon}
 \mu^*(A) + \varepsilon \geq \sum_{k \in \N} \mu(A_k).
\end{equation}
Da $A \subseteq \bigcup_{k \in \N} A_k$ ist
\[
 A \cap B \subseteq \left( \bigcup_{k \in \N} A_k \right) \cap B = \bigcup_{k \in \N} (A_k \cap B),
\]
wobei $A_k \cap B \in \A$ für alle $k \in \N$, sowie analog auch $A \cap B^c \subseteq \bigcup_{k \in \N} (A_k \cap B^c)$ mit $A_k \cap B^c \in \A$ für alle $k \in \N$. Aufgrund der endlichen Additivität von $\mu$ und der Definition sowie Monotonie von $\mu^*$ ist daher
\begin{align*}
 \sum_{k \in \N} \mu(A_k)
 &= \sum_{k \in \N} \mu\left( (A_k \cap B)\ \dotcup\ (A_k \cap B^c) \right) \\
 &= \sum_{k \in \N} \left( \mu(A_k \cap B) + \mu(A_k \cap B^c) \right) \\
 &= \left( \sum_{k \in \N} \mu(A_k \cap B) \right) + \left( \sum_{k \in \N} \mu(A_k \cap B^c) \right) \\
 &\geq \mu^*\left( \bigcup_{k \in \N} (A_k \cap B) \right) + \mu^*\left( \bigcup_{k \in \N} (A_k \cap B^c) \right) \\
 &\geq \mu^*(A \cap B) + \mu^*(A \cap B^c).
\end{align*}
Zusammen mit \eqref{eq:algebramessbarepsilon} ergibt sich damit, dass
\[
 \mu^*(A) + \varepsilon \geq \mu^*(A \cap B) + \mu^*(A \cap B^c).
\]
Aus der Beliebigkeit von $\varepsilon > 0$ folgt damit, dass
\[
 \mu^*(A) \geq \mu^*(A \cap B) + \mu^*(A \cap B^c).
\]


\subsection{}
sei $\varepsilon > 0$ beliebig aber fest und $A \in \A$. Sei $(A_k)_{k \in \N}$ eine Folge auf $\A$ mit $A \subseteq \bigcup_{k \in \N} A_k$ und
\[
 \mu^*(A) + \varepsilon \geq \sum_{k \in \N} \mu(A_k).
\]
Für $k \in \N$ sei $B_k := A \cap A_k \in \A$. Es ist
\[
 A = A \cap \bigcup_{k \in \N} A_k = \bigcup_{k \in \N} (A \cap A_k) = \bigcup_{k \in \N} B_k,
\]
wegen der $\sigma$-Additivität und Monotonie von $\mu$ auf $\A$ also
\[
 \mu(A)
 = \mu\left( \bigcup_{k \in \N} B_k \right)
 \leq \sum_{k \in \N} \mu(B_k)
 \leq \sum_{k \in \N} \mu(A_k)
 \leq \mu^*(A) + \varepsilon.
\]
Aus der Beliebigkeit von $\varepsilon > 0$ folgt, dass $\mu(A) \leq \mu^*(A)$.

Dass andererseits $\mu(A) \geq \mu^*(A)$ folgt direkt aus der Definition von $\mu^*$ unter Betrachtung der Folge $(A_k)_{k \in \N}$ mit $A_0 := A$ und $A_k := \emptyset$ für alle $k \geq 1$, für die $A = \bigcup_{k \in \N} A_k$ und $\mu(A) = \sum_{k \in \N} \mu(A_k)$.


\subsection{}
Es bezeichne $\mc{M}_{\mu^*}$, wie in der Vorlesung, die Menge aller $\mu^*$-messbaren Mengen. Nach Aufgabenteil \textbf{b)} ist $\A \in \mc{M}_{\mu^*}$, und wie aus der Vorlesung bekannt ist $\mc{M}_{\mu^*}$ ein $\sigma$-Algebra. Folglich ist $\sigma(\A) \subseteq \mc{M}_{\mu^*}$. Wie aus der Vorlesung bekannt ist die Einschränkung von $\mu^*$ auf $\mc{M}_{\mu^*}$ ein Maß auf $\mc{M}_{\mu^*}$, also die Einschränkung von $\mu^*$ auf $\sigma(\A) \subseteq \mc{M}_{\mu^*}$ ein Maß auf $\sigma(\A)$. Dass dieses auf $\A$ mit $\mu$ übereinstimmt wurde in Aufgabenteil \textbf{c)} gezeigt.

















\section{(Verschiedene äußere Maße)}





\end{document}
