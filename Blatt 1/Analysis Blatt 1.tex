\documentclass[a4paper,10pt]{article}
%\documentclass[a4paper,10pt]{scrartcl}

\usepackage{xltxtra}
\usepackage[ngerman]{babel}
\usepackage{amsmath}
\usepackage{amssymb}
\usepackage{amsthm}
\usepackage{mathtools}
\usepackage{nicefrac}
\usepackage{enumerate}
\usepackage{leftidx}

\theoremstyle{definition}
\newtheorem*{lem}{Lemma}
\newtheorem*{beh}{Behauptung}
\newtheorem*{bem}{Bemerkung}
\newtheorem*{ia}{Induktionsanfang}
\newtheorem*{is}{Induktionsschritt}

\renewcommand{\thesection}{Aufgabe \arabic{section}.}
\renewcommand{\thesubsection}{\alph{subsection})}
\renewcommand{\thesubsubsection}{(\roman{subsubsection})}

\newcommand{\N}{\mathbb{N}}
\newcommand{\Z}{\mathbb{Z}}
\newcommand{\Q}{\mathbb{Q}}
\newcommand{\R}{\mathbb{R}}
\newcommand{\C}{\mathbb{C}}
\newcommand{\La}{\mathcal{L}}
\newcommand{\dx}{\,\text{d}x}
\newcommand{\dy}{\,\text{d}y}
\newcommand{\dt}{\,\text{d}t}
\newcommand{\du}{\,\text{d}u}
\newcommand{\mc}[1]{\mathcal{#1}}
\newcommand{\Real}{\operatorname{Re}}
\newcommand{\Imag}{\operatorname{Im}}
\newcommand{\sgn}{\operatorname{sgn}}
\newcommand{\limes}[2]{\lim_{#1 \rightarrow #2}}
\newcommand{\limessup}[1]{\limsup_{#1 \rightarrow \infty}}
\newcommand{\limesinf}[1]{\liminf_{#1 \rightarrow \infty}}
\newcommand{\vect}[1]{\begin{pmatrix}#1\end{pmatrix}}
\newcommand{\partd}[2]{\frac{\partial #1}{\partial #2}}
\newcommand{\op}[1]{\left\|#1\right\|_{\text{op}}}

\makeatletter
\renewcommand*\env@matrix[1][*\c@MaxMatrixCols c]{%
  \hskip -\arraycolsep
  \let\@ifnextchar\new@ifnextchar
  \array{#1}}
\makeatother

\setromanfont[Mapping=tex-text]{Linux Libertine O}
% \setsansfont[Mapping=tex-text]{DejaVu Sans}
% \setmonofont[Mapping=tex-text]{DejaVu Sans Mono}
\parindent 0pt

\title{Analysis 3 — Übung 1}
\author{Jendrik Stelzner}
\date{\today}

\begin{document}
\maketitle






\section{ (Push-Forward und Pull-Back von $\sigma$-Algebren) }

\subsection{}
Es gilt, die Axiome einer $\sigma$-Algebra für $f^*[\mc{A}]$ zu überprüfen.

Da $\mc{A}$ eine $\sigma$-Algebra auf $X$ ist, ist $X \in \mc{A}$, also $f^{-1}(Y) = X \in \mc{A}$ und somit $Y \in f^*[A]$. Für $B \in f^*[\mc{A}]$ ist $f^{-1}(B) \in \mc{A}$, und da $\mc{A}$ eine $\sigma$-Algebra ist somit auch $f^{-1}(B^c) = \left( f^{-1}(B) \right)^c \in \mc{A}$, also $B^c \in f^*[\mc{A}]$. Für eine Folge $(B_n)_{n \in \N}$ auf $f^*[\mc{A}]$ ist $f^{-1}(B_n) \in \mc{A}$ für alle $n \in \N$, da $\mc{A}$ unter abzählbaren Vereinigungen abgeschlossen ist, und daher $f^{-1}\left(\bigcup_{n \in \N} B_n\right) = \bigcup_{n \in \N} f^{-1}(B_n) \in \mc{A}$, und daher auch $\bigcup_{n \in \N} B_n \in f^*[\mc{A}]$.

Damit sind alle Axiome einer $\sigma$-Algebra für $f^*[\mc{A}]$ erfüllt.

\subsection{}
Es gilt, die Axiome einer $\sigma$-Algebra für $f_*[\mc{B}]$ zu überprüfen.

Da $\mc{B}$ eine $\sigma$-Algebra auf $Y$ ist, ist $Y \in \mc{B}$, und somit $X = f^{-1}(Y) \in f_*[\mc{B}]$. Für $A \in f_*[\mc{B}]$ gibt es $B \in \mc{B}$ mit $f^{-1}(B) = A$; da $\mc{B}$ eine $\sigma$-Algebra ist, ist damit auch $B^c \in \mc{B}$ und somit $A^c = \left(f^{-1}(B)\right)^c = f^{-1}(B^c) \in f_*[\mc{B}]$. Ist $(A_n)_{n \in N}$ eine Folge auf $f_*[\mc{B}]$, so gibt es für alle $n \in \N$ ein $B_n \in \mc{B}$ mit $A_n = f^{-1}(B_n)$; da $\mc{B}$ als $\sigma$-Algebra ist, gilt damit auch $\bigcup_{n \in \N} B_n \in \mc{B}$, und somit $\bigcup_{n \in \N} A_n = \bigcup_{n \in N} f^{-1}(B_n) = f^{-1} \left(\bigcup_{n \in \N} B_n\right) \in f_*[\mc{B}]$.

$f_*[\mc{B}]$ erfüllt also alle Axiome einer $\sigma$-Algebra.





\section{(Gegenbeispiele)}





\section{(Die $\sigma$-Algebren auf einer dreielementigen Menge)}

\subsection{}
Es ist \[\mc{P}(\{1,2,3\}) = \{\emptyset, \{1\}, \{2\}, \{3\}, \{1\}^c, \{2\}^c, \{3\}^c, \{1,2,3\}\},\] wobei $\{1\}, \{2\}, \{3\}$ alle einelementigen, und $\{1\}^c, \{2\}^c, \{3\}^c$ alle zweielementigen Teilmengen von $\{1,2,3\}$ sind.
Die verschiedenen $\sigma$-Algebren auf $\{1,2,3\}$ lassen sich durch die jeweilige Anzahl der einelementigen Mengen klassifizieren. Sei $\mc{A}$ eine $\sigma$-Algebra auf $\{1,2,3\}$. Wir bemerken, dass aufgrund der Abgeschlossenheit von $\mc{A}$ unter Komplementbildung für alle $x \in \{1,2,3\} :\{x\} \in \mc{A} \Leftrightarrow \{x\}^c \in \mc{A}$.

Enthält $\mc{A}$ keine einelementige Menge, so enthält $\mc{A}$ nach der Bemerkung auch keine zweielementige Menge, es ist also $\mc{A} = \{\emptyset, \{1,2,3\}\}$.

Für alle $y \in \{1,2,3\}$ gibt es offenbar die $\sigma$-Algebra $\{\emptyset, \{y\}, \{y\}^c, \{1,2,3\}\}$ auf $\{1,2,3\}$. Dies sind die einzigen $\sigma$-Algebren, die genau eine einelementige Menge enthalten: Ist $\{x\} \in \mc{A}$ für genau ein $x \in \{1,2,3\}$, so ist auch $\{x\}^c \in \mc{A}$. Für jede zweielementige Menge $\{y\}^c \in \mc{A}$, muss auch $\{y\} \in \mc{A}$, und damit $y = x$ und $\{y\}^c = \{x\}^c$.

Gilt $\{x\}, \{y\} \in \mc{A}$ für zwei verschiedene $x,y \in \{1,2,3\}$, so ist auch $\{x,y\}^c = (\{x\} \cup \{y\})^c \in \mc{A}$, also ist $\{z\} \in \mc{A}$ für alle $z \in \{1,2,3\}$. Es gibt also keine $\sigma$-Algebra auf $\{1,2,3\}$ die genau zwei einelementige Mengen beinhaltet.

Ist $\{x\} \in \mc{A}$ für alle $x \in \{1,2,3\}$, so ist wegen \[\mc{P}(\{1,2,3\}) = \sigma(\{\{1\},\{2\},\{3\}\}) \subseteq \mc{A} \subseteq \mc{P}(\{1,2,3\})\] bereits $\mc\{A\} = \mc{P}(\{1,2,3\})$.

Die einzigen $\sigma$-Algebren auf $\{1,2,3\}$ sind somit $\{\emptyset,\{1,2,3\}\}, \mc{P}(\{1,2,3\})$ sowie $\{\emptyset, \{x,\}, \{x\}^c, \{1,2,3\}\}$ für die verschiedenen $x \in \{1,2,3\}$.






\end{document}
