\documentclass[a4paper,10pt]{article}
%\documentclass[a4paper,10pt]{scrartcl}

\usepackage{xltxtra}
\usepackage[ngerman]{babel}
\usepackage{amsmath}
\usepackage{amssymb}
\usepackage{amsthm}
\usepackage{mathtools}
\usepackage{nicefrac}
\usepackage{enumerate}
\usepackage{leftidx}

\makeatletter
\g@addto@macro\th@definition{\thm@headpunct{:}}
\makeatother
\newcounter{saetze}
\newtheorem{lem}[saetze]{Lemma}
\newtheorem{beh}[saetze]{Behauptung}
\newtheorem{bem}[saetze]{Bemerkung}
\newtheorem*{ia}{Induktionsanfang}
\newtheorem*{is}{Induktionsschritt}

\renewcommand{\thesection}{Aufgabe \arabic{section}.}
\renewcommand{\thesubsection}{\alph{subsection})}
\renewcommand{\thesubsubsection}{(\roman{subsubsection})}

\makeatletter
\def\moverlay{\mathpalette\mov@rlay}
\def\mov@rlay#1#2{\leavevmode\vtop{%
   \baselineskip\z@skip \lineskiplimit-\maxdimen
   \ialign{\hfil$\m@th#1##$\hfil\cr#2\crcr}}}
\newcommand{\charfusion}[3][\mathord]{
    #1{\ifx#1\mathop\vphantom{#2}\fi
        \mathpalette\mov@rlay{#2\cr#3}
      }
    \ifx#1\mathop\expandafter\displaylimits\fi}
\makeatother

\newcommand{\N}{\mathbb{N}}
\newcommand{\Z}{\mathbb{Z}}
\newcommand{\Q}{\mathbb{Q}}
\newcommand{\R}{\mathbb{R}}
\newcommand{\C}{\mathbb{C}}
\newcommand{\A}{\mathcal{A}}
\newcommand{\La}{\mathcal{L}}
\newcommand{\dx}{\,\text{d}x}
\newcommand{\dy}{\,\text{d}y}
\newcommand{\dt}{\,\text{d}t}
\newcommand{\du}{\,\text{d}u}
\newcommand{\dmu}{\,\text{d}\mu}
\newcommand{\dlambda}{\,\text{d}\lambda}
\newcommand{\Img}{\operatorname{Im}}
\newcommand{\Real}{\operatorname{Re}}
\newcommand{\Imag}{\operatorname{Im}}
\newcommand{\sgn}{\operatorname{sgn}}
\newcommand{\Vol}{\operatorname{Vol}}
\newcommand{\dotcup}{\ensuremath{\mathaccent\cdot\cup}}
\newcommand{\bigdotcup}{\charfusion[\mathop]{\bigcup}{\cdot}}
\newcommand{\ceil}[1]{\left\lceil{#1}\right\rceil}
\newcommand{\floor}[1]{\left\lfloor{#1}\right\rfloor}
\newcommand{\mc}[1]{\mathcal{#1}}
\newcommand{\limes}[2]{\lim_{#1 \rightarrow #2}}
\newcommand{\limessup}[1]{\limsup_{#1 \rightarrow \infty}}
\newcommand{\limesinf}[1]{\liminf_{#1 \rightarrow \infty}}
\newcommand{\vect}[1]{\begin{pmatrix}#1\end{pmatrix}}
\newcommand{\partd}[2]{\frac{\partial #1}{\partial #2}}
\newcommand{\op}[1]{\left\|#1\right\|_{\text{op}}}

\makeatletter
\renewcommand*\env@matrix[1][*\c@MaxMatrixCols c]{%
  \hskip -\arraycolsep
  \let\@ifnextchar\new@ifnextchar
  \array{#1}}
\makeatother

\setromanfont[Mapping=tex-text]{Linux Libertine O}
% \setsansfont[Mapping=tex-text]{DejaVu Sans}
% \setmonofont[Mapping=tex-text]{DejaVu Sans Mono}
\parindent 0pt

\title{\sc Analysis III \\ \Large 7. Aufgabenblatt}
\author{Jendrik Stelzner}
\date{\today}

\begin{document}
\maketitle





\section{(Eine Integralformel)}





\section{(Funktionen mit verschwindendem Integral auf Anfangsstücken)}

\subsection{}
Wir zeigen, dass $\int_U f \dlambda = 0$ für alle offenen Mengen $U \subseteq (a,b)$. Zunächst bemerken wir, dass für alle $x \in (a,b)$
\[
 0
 = \int_{(a,b)} f \dlambda
 = \int_{(a,x)} f \dlambda + \int_{(x,b)} f \dlambda
 = \int_{(x,b)} f \dlambda,
\]
also für alle $x,y \in (a,b)$ mit $x < y$
\[
 0
 = \int_{(a,b)} f \dlambda
 = \int_{(a,x)} f \dlambda + \int_{(x,y)} f \dlambda + \int_{(y,b)} f \dlambda
 = \int_{(x,y)} f \dlambda.
\]

Sei nun $U \subseteq (a,b)$ offen. Wie bereits letzte Woche gezeigt können wir $U$ schreiben als $U = \bigdotcup_{n \in \N} U_n$, wobei $U_n = \emptyset$ oder $U_n$ ein offenes Intervall ist für alle $n \in \N$. Für $g_n = \sum_{k=1}^n f \chi_{U_k}$ und $g = f\chi_U$ ist $g_n \rightarrow g$ punktweise auf ganz $(a,b)$. Da $f$ integrierbar ist, ist es auch $|f|$, und wegen $|g_n| \leq |f|$ für alle $n \in \N$ nach dem Satz über dominierte Konvergenz daher
\begin{align*}
 \int_U f \dlambda
 &= \int_{(a,b)} f \chi_U \dlambda
 = \limes{n}{\infty} \int_{(a,b)} \sum_{k=0}^n f \chi_{U_k} \dlambda \\
 &= \sum_{k=0}^\infty \int_{(a,b)} f \chi_{U_k} \dlambda
 = \sum_{k=0}^\infty \int_{U_k} f\dlambda
 = 0.
\end{align*}

Es sei nun $E_+ := \{x \in (a,b) : f(x) > 0\}$ mit $\lambda(E_+) \leq b-a < \infty$. Wegen der Regularität des Lebesgue-Maßes und da $\lambda_(E_+) < \infty$ finden wir eine Folge $(U_n)_{n \in \N}$ offener Mengen mit $E_+ \subseteq U_n$ und $\lambda(U_n \setminus E_+) \leq \frac{1}{n}$ für alle $n$. Wir können o.B.d.A. davon ausgehen, dass die Folge $(U_n)_{n \in \N}$ fallend ist, da wir sonst die Folge $\left( \bigcap_{k=1}^n U_k \right)_{n \in \N}$ betrachten. Es sei $U := \bigcap_{n \in \N} U_n$.

Es ist $\int_U f \dlambda$: Es ist $f\chi_{U_n} \rightarrow f \chi_U$ punktweise auf ganz $(a,b)$ und $|f\chi_{U_n}| \leq |f|$ für alle $n \in \N$, nach dem Satz über dominierte Konvergenz also
\[
 \int_U f \dlambda
 = \int_{(a,b)} f \chi_U \dlambda
 = \limes{n}{\infty} \int_{(a,b)} f \chi_{U_n} \dlambda
 = \limes{n}{\infty} \int_{U_n} f \dlambda
 = 0.
\]

Es ist $E_+ \subseteq U$ und, da $\mu(U_0) < \infty$ und die Folge $(U_n)_{n \in \N}$ fallend ist, 
\begin{align*}
 \mu(U \setminus E_+)
 &= \mu\left( \left( \bigcap_{n \in \N} U_n \right) \setminus E_+\right)
 = \mu\left( \bigcap_{n \in \N} (U_n \setminus E_+) \right) \\
 &= \limes{n}{\infty} \mu(U_n \setminus E_+)
 = \limes{n}{\infty} \frac{1}{n}
 = 0.
\end{align*}

Es ist0also $\int_{E_+} f \dlambda = \int_U f \dlambda = 0$. Da $f_{|E_+} > 0$ ist daher, wie aus der Vorlesung bekannt, $f(x) = 0$ für $\lambda$-fast alle $x \in E_+$. Nach der Definition von $E_+$ muss daher $\lambda(E_+) = 0$. Analog ergibt sich für $E_- := \{x \in (a,b) : f(x) < 0\}$, dass auch $\lambda(E_-) = 0$. Also ist
\[
 \lambda(\{x \in (a,b) : f(x) \neq 0\}) = \lambda(E_-\ \dotcup\ E_+) = \lambda(E_-) + \lambda(E_+) = 0.
\]








\section{(Ein Konvergenzresultat)}





\section{(Gegenbeispiele)}











\end{document}
