\documentclass[a4paper,10pt]{article}
%\documentclass[a4paper,10pt]{scrartcl}

\usepackage{xltxtra}
\usepackage[ngerman]{babel}
\usepackage{amsmath}
\usepackage{amssymb}
\usepackage{amsthm}
\usepackage{mathtools}
\usepackage{nicefrac}
\usepackage{enumerate}
\usepackage{leftidx}

\makeatletter
\g@addto@macro\th@definition{\thm@headpunct{:}}
\makeatother
\newcounter{saetze}
\newtheorem{lem}[saetze]{Lemma}
\newtheorem{beh}[saetze]{Behauptung}
\newtheorem{bem}[saetze]{Bemerkung}
\newtheorem*{ia}{Induktionsanfang}
\newtheorem*{is}{Induktionsschritt}

\renewcommand{\thesection}{Aufgabe \arabic{section}.}
\renewcommand{\thesubsection}{\alph{subsection})}
\renewcommand{\thesubsubsection}{(\roman{subsubsection})}

\makeatletter
\def\moverlay{\mathpalette\mov@rlay}
\def\mov@rlay#1#2{\leavevmode\vtop{%
   \baselineskip\z@skip \lineskiplimit-\maxdimen
   \ialign{\hfil$\m@th#1##$\hfil\cr#2\crcr}}}
\newcommand{\charfusion}[3][\mathord]{
    #1{\ifx#1\mathop\vphantom{#2}\fi
        \mathpalette\mov@rlay{#2\cr#3}
      }
    \ifx#1\mathop\expandafter\displaylimits\fi}
\makeatother

\newcommand{\N}{\mathbb{N}}
\newcommand{\Z}{\mathbb{Z}}
\newcommand{\Q}{\mathbb{Q}}
\newcommand{\R}{\mathbb{R}}
\newcommand{\C}{\mathbb{C}}
\newcommand{\A}{\mathcal{A}}
\newcommand{\La}{\mathcal{L}}
\newcommand{\dx}{\,\text{d}x}
\newcommand{\dy}{\,\text{d}y}
\newcommand{\dt}{\,\text{d}t}
\newcommand{\du}{\,\text{d}u}
\newcommand{\dmu}{\,\text{d}\mu}
\newcommand{\dlambda}{\,\text{d}\lambda}
\newcommand{\Img}{\operatorname{Im}}
\newcommand{\Real}{\operatorname{Re}}
\newcommand{\Imag}{\operatorname{Im}}
\newcommand{\sgn}{\operatorname{sgn}}
\newcommand{\Vol}{\operatorname{Vol}}
\newcommand{\dotcup}{\ensuremath{\mathaccent\cdot\cup}}
\newcommand{\bigdotcup}{\charfusion[\mathop]{\bigcup}{\cdot}}
\newcommand{\ceil}[1]{\left\lceil{#1}\right\rceil}
\newcommand{\floor}[1]{\left\lfloor{#1}\right\rfloor}
\newcommand{\mc}[1]{\mathcal{#1}}
\newcommand{\limes}[2]{\lim_{#1 \rightarrow #2}}
\newcommand{\limessup}[1]{\limsup_{#1 \rightarrow \infty}}
\newcommand{\limesinf}[1]{\liminf_{#1 \rightarrow \infty}}
\newcommand{\vect}[1]{\begin{pmatrix}#1\end{pmatrix}}
\newcommand{\partd}[2]{\frac{\partial #1}{\partial #2}}
\newcommand{\op}[1]{\left\|#1\right\|_{\text{op}}}

\makeatletter
\renewcommand*\env@matrix[1][*\c@MaxMatrixCols c]{%
  \hskip -\arraycolsep
  \let\@ifnextchar\new@ifnextchar
  \array{#1}}
\makeatother

\setromanfont[Mapping=tex-text]{Linux Libertine O}
% \setsansfont[Mapping=tex-text]{DejaVu Sans}
% \setmonofont[Mapping=tex-text]{DejaVu Sans Mono}
\parindent 0pt

\title{\sc Analysis III \\ \Large 7. Aufgabenblatt}
\author{Jendrik Stelzner}
\date{\today}

\begin{document}
\maketitle





\section{(Eine Integralformel)}


\subsection{}
Aus der Monotonie von $\mu$ folgt, dass für $a,b \in [0,\infty)$ mit $a \leq b$
\[
 g(a)
 = \mu(\{x \in X : f(x) \geq a\})
 \geq \mu(\{x \in X : f(x) \geq b\})
 = g(b).
\]
$g$ ist also monoton fallend und somit bekanntermaßen Borel-messbar.


\subsection{}
Zunächst zeigen wir die Aussage für einfache Funktionen: Sei $f$ eine einfache Funktion mit Werten $0 \leq \alpha_1 < \alpha_2 < \ldots, < \alpha_n$, die auf Mengen $A_1, \ldots, A_n$ angenommen werden; wir setzen $\alpha_0 := 0$.

Ist $\int_X f \dmu = \infty$, so gibt es $1 \leq i \leq n$ mit $\alpha_i \neq 0$ und $\mu(A_i) = \infty$. Es ist daher $g(t) = \mu\{x \in X: f(x) \geq t\} = \infty$ für alle $t \in [0,\alpha_i]$, und somit auch $\int_{[0,\infty)} g = \infty$.

Ansonsten nimmt $g$ für $i=1,\ldots,n$ auf dem Intervall $(\alpha_{i-1},\alpha_i]$ den konstanten Wert $\sum_{j=1}^n \mu(A_j)$, und auf $(\alpha_n, \infty)$ den Wert $0$ an. Es ist dann
\begin{align*}
 \int_{[0,\infty)} g \dlambda
 &= \int_{(0,\infty)} g \dlambda
 = \sum_{i=1}^n \lambda((\alpha_{i-1},\alpha_i) \sum_{j=i}^n \mu(A_j) \\
 &= \sum_{i=1}^n (\alpha_i - \alpha_{i-1})\sum_{j=i}^n \mu(A_j)
 = \sum_{i=1}^n \sum_{j=i}^n (\alpha_i - \alpha_{i-1}) \mu(A_j) \\
 &= \sum_{j=1}^n \sum_{i=1}^j (\alpha_j - \alpha_{j-1}) \mu(A_j)
 = \sum_{j=1}^n (\alpha_j - \alpha_0) \mu(A_j) \\
 &= \sum_{j=1}^n \alpha_j \mu(A_j)
 = \int_X f \dmu.
\end{align*}





\section{(Funktionen mit verschwindendem Integral auf Anfangsstücken)}
Wir zeigen, dass $\int_U f \dlambda = 0$ für alle offenen Mengen $U \subseteq (a,b)$. Zunächst bemerken wir, dass für alle $x \in (a,b)$
\[
 0
 = \int_{(a,b)} f \dlambda
 = \int_{(a,x)} f \dlambda + \int_{(x,b)} f \dlambda
 = \int_{(x,b)} f \dlambda,
\]
also für alle $x,y \in (a,b)$ mit $x < y$
\[
 0
 = \int_{(a,b)} f \dlambda
 = \int_{(a,x)} f \dlambda + \int_{(x,y)} f \dlambda + \int_{(y,b)} f \dlambda
 = \int_{(x,y)} f \dlambda.
\]

Sei nun $U \subseteq (a,b)$ offen. Wie bereits letzte Woche gezeigt können wir $U$ schreiben als $U = \bigdotcup_{n \in \N} U_n$, wobei $U_n = \emptyset$ oder $U_n$ ein offenes Intervall ist für alle $n \in \N$. Für $g_n = \sum_{k=1}^n f \chi_{U_k}$ und $g = f\chi_U$ ist $g_n \rightarrow g$ punktweise auf ganz $(a,b)$. Da $f$ integrierbar ist, ist es auch $|f|$, und wegen $|g_n| \leq |f|$ für alle $n \in \N$ nach dem Satz über dominierte Konvergenz daher
\begin{align*}
 \int_U f \dlambda
 &= \int_{(a,b)} f \chi_U \dlambda
 = \limes{n}{\infty} \int_{(a,b)} \sum_{k=0}^n f \chi_{U_k} \dlambda \\
 &= \sum_{k=0}^\infty \int_{(a,b)} f \chi_{U_k} \dlambda
 = \sum_{k=0}^\infty \int_{U_k} f\dlambda
 = 0.
\end{align*}

Es sei nun $E_+ := \{x \in (a,b) : f(x) > 0\}$ mit $\lambda(E_+) \leq b-a < \infty$. Wegen der Regularität des Lebesgue-Maßes und da $\lambda_(E_+) < \infty$ finden wir eine Folge $(U_n)_{n \in \N}$ offener Mengen mit $E_+ \subseteq U_n$ und $\lambda(U_n \setminus E_+) \leq \frac{1}{n}$ für alle $n$. Wir können o.B.d.A. davon ausgehen, dass die Folge $(U_n)_{n \in \N}$ fallend ist, da wir sonst die Folge $\left( \bigcap_{k=1}^n U_k \right)_{n \in \N}$ betrachten. Es sei $U := \bigcap_{n \in \N} U_n$.

Es ist $\int_U f \dlambda$: Es ist $f\chi_{U_n} \rightarrow f \chi_U$ punktweise auf ganz $(a,b)$ und $|f\chi_{U_n}| \leq |f|$ für alle $n \in \N$, nach dem Satz über dominierte Konvergenz also
\[
 \int_U f \dlambda
 = \int_{(a,b)} f \chi_U \dlambda
 = \limes{n}{\infty} \int_{(a,b)} f \chi_{U_n} \dlambda
 = \limes{n}{\infty} \int_{U_n} f \dlambda
 = 0.
\]

Es ist $E_+ \subseteq U$ und, da $\mu(U_0) < \infty$ und die Folge $(U_n)_{n \in \N}$ fallend ist, 
\begin{align*}
 \mu(U \setminus E_+)
 &= \mu\left( \left( \bigcap_{n \in \N} U_n \right) \setminus E_+\right)
 = \mu\left( \bigcap_{n \in \N} (U_n \setminus E_+) \right) \\
 &= \limes{n}{\infty} \mu(U_n \setminus E_+)
 = \limes{n}{\infty} \frac{1}{n}
 = 0.
\end{align*}

Es ist0also $\int_{E_+} f \dlambda = \int_U f \dlambda = 0$. Da $f_{|E_+} > 0$ ist daher, wie aus der Vorlesung bekannt, $f(x) = 0$ für $\lambda$-fast alle $x \in E_+$. Nach der Definition von $E_+$ muss daher $\lambda(E_+) = 0$. Analog ergibt sich für $E_- := \{x \in (a,b) : f(x) < 0\}$, dass auch $\lambda(E_-) = 0$. Also ist
\[
 \lambda(\{x \in (a,b) : f(x) \neq 0\}) = \lambda(E_-\ \dotcup\ E_+) = \lambda(E_-) + \lambda(E_+) = 0.
\]





\section{(Ein Konvergenzresultat)}
Wir können o.B.d.A. davon ausgehen, dass $f_n \rightarrow f$ auf ganz $X$, da die Menge $\{x \in X : f_n \not\rightarrow f\}$ eine $\mu$-Nullmenge ist, also für das Integral $\int_X |f_n-f| \dmu$ nicht von Bedeutung ist. Wegen $f_n \geq 0$ für alle $n$ folgt damit insbesondere, dass $f \geq 0$.

Da $f$ und $f_n$ für alle $n$ integrierbar sind, ist es auch $f-f_n$ und daher auch $|f-f_n|$ für alle $n$. Nach der Dreiecksungleichung ist dabei $|f_n-f| \leq f+f_n$ für alle $n$, also $0 \leq f_n + f - |f_n - f|$ für alle $n$. Daher ist nach dem Lemma von Fatou
\begin{equation}\label{eq: lemma von fatou}
 \int_X \limesinf{n} f_n+f-|f_n-f| \dmu
 \leq \limesinf{n} \int_X f_n+f-|f_n-f| \dmu.
\end{equation}
Dabei ist
\[
 \int_X \limesinf{n} f_n+f-|f_n-f| \dmu
 = \int_X \limesinf{n} 2f \dmu
 = 2 \int_X f \dmu
\]
und
\begin{align*}
  &\,\limesinf{n} \int_X f_n+f-|f_n-f| \dmu \\
 =&\, \limesinf{n} \int_X f_n \dmu + \limesinf{n} \int f \dmu + \limesinf{n} -\int_X |f_n-f| \dmu \\
 =&\, 2 \int_X f \dmu - \limessup{n} \int_X |f_n-f| \dmu.
\end{align*}
Aus \eqref{eq: lemma von fatou} ergibt sich damit, dass
\[
 0 \leq - \limessup{n} \int_X |f_n-f| \dmu,
\]
also
\[
 0
 \geq \limessup{n} \int_X |f_n-f| \dmu
 \geq \limesinf{n} \int_X |f_n-f| \dmu
 \geq 0,
\]
und daher $\limes{n}{\infty} \int_X |f_n-f| \dmu = 0$.





\section{(Gegenbeispiele)}


\subsection{}
Es sei $f = 0$ und $f_n = \frac{1}{n} \chi_{[-n,n]}$ für alle $n \geq 1$. Dann ist $f_n \rightarrow f$ gleichmäßig auf $\R$, aber für alle $n \geq 1$
\[
 \int_\R f_n \dlambda = 2 \neq 0 = \int_\R f \dlambda.
\]
Inbesondere konvergiert $\int_\R f_n \dlambda$ nicht gegen $\int_\R f \lambda$.


\subsection{}
Diese Aussage ist totaler Blödsinn: Aus ihr folgt insbesondere, dass zwei Funktionen genau dann das gleich Integral haben, wenn sie fast überall gleich sind. Ein einfaches und offensichtliches Gegenbeispiel ist $f_n = \chi_{[-1,0]}$ unabhängig von $n \in \N$ und $f = \chi_{[0,1]}$.


\subsection{}
Man betrachte $f=0$ und die Funktionenfolge $(f_n)_{n \in \N}$ definiert als
\[
 f_{2^n-1+k} := \chi_{\left[k/2^n,(k+1)/2^n\right]}
\]
für alle $n \in \N$ und $k=0,\ldots,n-1$, d.h. $f_0 = \chi_{[0,1]}$, $f_1 = \chi_{[0,1/2]}$, $f_2 = \chi_{[1/2,1]}$, $f_{3+k} = \chi_{[k/4,(k+1)/4]}$ für $k=0,\ldots,3$, usw.

Es ist offenbar
\[
 \int_\R f_n \dlambda \rightarrow 0 = \int_\R f \dlambda,
\]
aber $f_n \not\rightarrow f$ auf ganz $[0,1]$.





\end{document}
