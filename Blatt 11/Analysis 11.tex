\documentclass[a4paper,10pt]{article}
%\documentclass[a4paper,10pt]{scrartcl}

\usepackage{xltxtra}
\usepackage[ngerman]{babel}
\usepackage{amsmath}
\usepackage{amssymb}
\usepackage{amsthm}
\usepackage{mathtools}
\usepackage{nicefrac}
\usepackage{enumerate}
\usepackage{leftidx}
\usepackage{stmaryrd} % for \longarrownot

\makeatletter
\g@addto@macro\th@definition{\thm@headpunct{:}}
\makeatother
\newcounter{saetze}
\newtheorem{lem}[saetze]{Lemma}
\newtheorem{beh}[saetze]{Behauptung}
\newtheorem{bem}[saetze]{Bemerkung}
\theoremstyle{definition}
\newtheorem*{ia}{Induktionsanfang}
\newtheorem*{is}{Induktionsschritt}

\renewcommand{\thesection}{Aufgabe \arabic{section}.}
\renewcommand{\thesubsection}{\alph{subsection})}
\renewcommand{\thesubsubsection}{(\roman{subsubsection})}

\makeatletter
\def\moverlay{\mathpalette\mov@rlay}
\def\mov@rlay#1#2{\leavevmode\vtop{%
   \baselineskip\z@skip \lineskiplimit-\maxdimen
   \ialign{\hfil$\m@th#1##$\hfil\cr#2\crcr}}}
\newcommand{\charfusion}[3][\mathord]{
    #1{\ifx#1\mathop\vphantom{#2}\fi
        \mathpalette\mov@rlay{#2\cr#3}
      }
    \ifx#1\mathop\expandafter\displaylimits\fi}
\makeatother

\newcommand{\N}{\mathbb{N}}
\newcommand{\Z}{\mathbb{Z}}
\newcommand{\Q}{\mathbb{Q}}
\newcommand{\R}{\mathbb{R}}
\newcommand{\C}{\mathbb{C}}
\newcommand{\A}{\mathcal{A}}
\newcommand{\cL}{\mathcal{L}}
\newcommand{\dr}{\,\text{d}r}
\newcommand{\dt}{\,\text{d}t}
\newcommand{\du}{\,\text{d}u}
\newcommand{\dx}{\,\text{d}x}
\newcommand{\dy}{\,\text{d}y}
\newcommand{\dmu}{\,\text{d}\mu}
\newcommand{\dlambda}{\,\text{d}\lambda}
\newcommand{\dvarphi}{\,\text{d}\varphi}
\newcommand{\Img}{\operatorname{Im}}
\newcommand{\Real}{\operatorname{Re}}
\newcommand{\Imag}{\operatorname{Im}}
\newcommand{\sgn}{\operatorname{sgn}}
\newcommand{\diag}{\operatorname{diag}}
\newcommand{\Vol}{\operatorname{Vol}}
\newcommand{\dotcup}{\ensuremath{\mathaccent\cdot\cup}}
\newcommand{\bigdotcup}{\charfusion[\mathop]{\bigcup}{\cdot}}
\newcommand{\nlongrightarrow}{\longarrownot\longrightarrow}
\newcommand{\ceil}[1]{\left\lceil{#1}\right\rceil}
\newcommand{\floor}[1]{\left\lfloor{#1}\right\rfloor}
\newcommand{\mc}[1]{\mathcal{#1}}
\newcommand{\limes}[2]{\lim_{#1 \to #2}}
\newcommand{\limessup}[1]{\limsup_{#1 \to \infty}}
\newcommand{\limesinf}[1]{\liminf_{#1 \to \infty}}
\newcommand{\vect}[1]{\begin{pmatrix}#1\end{pmatrix}}
\newcommand{\partd}[2]{\frac{\partial #1}{\partial #2}}
\newcommand{\op}[1]{\left\|#1\right\|_{\text{op}}}

\makeatletter
\renewcommand*\env@matrix[1][*\c@MaxMatrixCols c]{%
  \hskip -\arraycolsep
  \let\@ifnextchar\new@ifnextchar
  \array{#1}}
\makeatother

\setromanfont[Mapping=tex-text]{Linux Libertine O}
% \setsansfont[Mapping=tex-text]{DejaVu Sans}
% \setmonofont[Mapping=tex-text]{DejaVu Sans Mono}
\parindent 0pt

\title{\sc Analysis III \\ \Large 11. Aufgabenblatt}
\author{Jendrik Stelzner}
\date{\today}

\begin{document}
\maketitle





\section{(Dualität)}
Für den Fall $f \in \cL^p(\Omega)$, mit $p \in [1,\infty]$, ist die Aussage bereits aus der Vorlesung bekannt. Es ist nämlich bekannt, dass die Abbildung
\[
 \psi : L^p(\Omega) \to \left(L^{p'}(\Omega)\right)^* \text{ mit }
 \psi(f)(g) = \int_\Omega f g \dmu \text{ für alle } g \in L^{p'}(\Omega)
\]
wegen der $\sigma$-Endlichkeit des Maßraums $(\Omega, \A, \mu)$ für alle $p \in [1,\infty]$ eine Isometrie ist, dass also
\begin{align*}
 \|f\|_p
 = \|\psi(f)\|
 &= \sup \left\{ |\psi(f)(g)| : g \in L^{p'}(\Omega), \|g\|_{p'} \leq 1 \right\} \\
 &= \sup \left\{ \left|\int_\Omega f g \dmu\right| : g \in L^{p'}(\Omega), \|g\|_{p'} \leq 1 \right\} \\
 &= \sup \left\{ \int_\Omega f g \dmu : g \in L^{p'}(\Omega), \|g\|_{p'} \leq 1 \right\}.
\end{align*}

Sei nun $f : \Omega \to [-\infty, \infty]$ messbar und $p \in [1,\infty]$ mit $f \not\in \cL^p(\Omega)$. Wir unterscheiden zwischen den drei Fällen $p = 1$, $p = \infty$ und $1 < p < \infty$.

Im Fall $p = 1$ lässt sich unabhängig von $M \in \N$ die Testfunktion $g_M = \sgn(f)$ wählen. Es ist klar, dass $g_M \in \cL^\infty(\Omega)$ mit $\|g_M\|_\infty = 1$. Da $f \not\in \cL^1(\Omega)$ ist
\[
 \int_\Omega f g_M \dmu
 = \int_\Omega |f| \dmu
 = \infty \geq M.
\]

Ist $p = \infty$ so setzen wir für $M \in \N$
\[
 A_M := \{x \in \Omega : |f(x)| \geq M\}.
\]
Offenbar ist $A_M \in \A$, und da $f \not\in \cL^\infty(\Omega)$ ist $\mu(A_M) > 0$. Wegen der $\sigma$-Endlichkeit des Maßraums $(\Omega, \A, \mu)$ gibt es eine Menge $B_M \subseteq A_M$ mit $0 < \mu(B_M) < \infty$. Für die Funktion
\[
 g_M := \frac{\chi_{B_M}}{\mu(B_M)} \sgn(f)
\]
ist $g_M \in \cL^1(\Omega)$ mit $\|g_M\|_1 = 1$, da
\[
 \int_\Omega |g_M| \dmu
 = \frac{1}{\mu(B_M)} \int_{B_M} 1 \dmu
 = 1.
\]
Auch ist
\[
 \int_\Omega f g \dmu
 = \frac{1}{\mu(B_M)} \int_{B_M} |f| \dmu
 \geq \frac{1}{\mu(B_M)} \int_{B_M} M \dmu
 = M.
\]

Sei nun $1 < p < \infty$. Wegen der $\sigma$-Endlichkeit des Maßraumes $(\Omega, \A, \mu)$ gibt es eine Folge $(B_n)_{n \in \N}$ auf $\A$ mit $\Omega = \bigcup_{n \in \N} B_n$, so dass $B_n \subseteq B_{n+1}$ und $\mu(B_n) < \infty$ für alle $n \in \N$.

Da $|f| : \Omega \to [0,\infty]$ messbar ist gibt es eine Folge $(f_k)_{k \in \N}$ einfacher, messbarer Funktionen mit $f_k \geq 0$ und $f_k \leq f_{k+1}$ für alle $k \in \N$ und $\limes{k}{\infty} f_k = |f|$ punktweise überall. Aufgrund von Monotonie und Stetigkeit ist auch $(f_k^p)_{k \in N}$ eine Folge einfacher, messbarer Funktionen mit $f_k^p \geq 0$, $f_k^p \leq f_{k+1}^p$ für alle $k \in \N$ und $\limes{k}{\infty} f_k^p = |f|^p$.

Mehrfache Anwendung des Satzes über Monotone Konvergenz ergibt, dass
\[
 \infty
 = \int_\Omega |f|^p \dmu
 = \limes{n}{\infty} \int_{B_n} |f|^p \dmu
 = \limes{n}{\infty} \limes{k}{\infty} \int_{B_n} f_k^p \dmu.
\]
Es gibt daher $N \in \N$ mit
\[
 \limes{k}{\infty} \int_{B_N} f_k^p \dmu \geq (M+1)^p + 1,
\]
also ein $K \in \N$ mit
\[
 \int_{B_N} f_K^p \dmu \geq (M+1)^p.
\]
Wir setzen
\[
 h := \chi_{B_N} f_K
\]
und bemerken auch direkt, dass $|f| \geq h \geq 0$.


Da $\mu(B_N) < \infty$ und $f_K^p$ als einfache Funktion beschränkt ist, ist
\[
 \int_\Omega |h|^p \dmu
 = \int_{B_N} f_K^p \dmu
 < \infty.
\]
Es ist also $h \in \cL^p(\Omega)$ mit
\[
 \|h\|_p
 = \left( \int_\Omega |h|^p \dmu \right)^{1/p}
 = \left( \int_{B_N} f_K^p \dmu \right)^{1/p}
 \geq M+1.
\]

Da $h \in \cL^p(\Omega)$ ist
\[
 \|h\|_p
 = \sup \left\{ \int_\Omega h g \dmu : g \in \cL^{p'}(\Omega), \|g\|_{p'} \leq 1 \right\}.
\]
Es gibt also ein $g \in \cL^{p'}(\Omega)$ mit $\|g\|_{p'} \leq 1$, so dass
\[
 \int_\Omega h g \dmu \geq \|h\|_p -1 \geq M.
\]
Für $g_M = \sgn(f) |g|$ ist $|g_M| = |g|$ und deshalb $g_M \in \cL^{p'}(\Omega)$ mit $\|g_M\|_{p'} = 1$. Da $|f| \geq h \geq 0$ ist schließlich
\[
 \int_\Omega f g_M \dmu
 = \int_\Omega |f| |g| \dmu
 \geq \int_\Omega h |g| \dmu
 \geq \int_\Omega h g \dmu
 \geq M.
\]





\section{(Die Hölder-Ungleichung)}


\subsection{}
Wir betrachten zunächst den Fall $r = \infty$. Da
\[
 \frac{1}{p} + \frac{1}{q} = \frac{1}{r} = 0
\]
ist $p = q = \infty$. Es ist klar, dass für alle $f,g \in \cL^\infty(\Omega)$
\[
 |f(x)g(x)| = |f(x)| |g(x)| \leq \|f\|_\infty \|g\|_\infty \text{ für $\mu$-fast alle } x \in \Omega.
\]
Das zeigt, dass $fg \in \cL^\infty(\Omega)$ und die Ungleichung gilt.

Ist $r \in [1,\infty)$, so folgt aus $f \in \cL^p(\Omega)$, dass $|f|^r \in \cL^{p/r}(\Omega)$ mit $\||f|^r\|_{p/r} = \|f\|_p^r$, und aus $g \in \cL^q(\Omega)$, dass $|g|^r \in \cL^{q/r}(\Omega)$ mit $\||g|^r\|_{q/r} = \|f\|_q^r$. Dass dabei $p/r, q/r \geq 1$ folgt daraus, dass $1/p, 1/q \leq 1/r$, also $p,q \geq r$.

Da
\[
 \frac{1}{p/r} + \frac{1}{q/r}
 = \frac{r}{p} + \frac{r}{q}
 = r\left(\frac{1}{p} + \frac{1}{q}\right)
 = r\,\frac{1}{r}
 = 1
\]
ist nach der Hölder-Ungleichung
\[
 \int_\Omega |fg|^r \dmu
 = \int_\Omega |f|^r |g|^r \dmu
 \leq \||f|^r\|_{p/r} \||g|^r\|_{q/r}
 = \|f\|_p^r \|g\|_q^r.
\]
Dass die rechte Seite der Gleichung endlich ist, zeigt, dass $fg \in \cL^r(\Omega)$. Außerdem folgt, dass
\[
 \|fg\|_r
 = \left( \int_\Omega |fg|^r \dmu \right)^{1/r}
 \leq \|f\|_p \|g\|_q.
\]


\subsection{}
Wir zeigen die Aussage per Induktion über $k \geq 1$.
\begin{ia}
 Für $k=1$ ist nichts zu zeigen, und für $k=2$ handelt es sich um die Aussage des vorherigen Aufgabenteils.
\qed\end{ia}
\begin{is}
 Es sei nun $k \geq 3$ und es gelte die Aussage für $k-1$ und $2$. Wegen $\sum_{i=1}^k 1/p_i = 1/r$ ist
 \[
  \sum_{i=1}^{k-1} \frac{1}{p_i} = \frac{1}{r} - \frac{1}{p_k}.
 \]
 
 Ist $1/r - 1/p_k = 0$, so ist $r = p_k$ und $p_1 = \ldots = p_{k-1} = \infty$. Für $1 \leq r < \infty$ ist dann
 \[
  \int_\Omega \left|\prod_{i=1}^k f_i\right|^r \dmu
  = \int_\Omega \prod_{i=1}^k |f_i|^r \dmu
  \leq \prod_{i=1}^{k-1} \|f_i\|_\infty^r \int_\Omega |f_k|^r \dmu
  = \prod_{i=1}^{k} \|f_i\|_{p_i}^r,
 \]
 also $\prod_{i=1}^k f_i \in \cL^r(\Omega)$ mit $\|\prod_{i=1}^k f_i\|_r \leq \prod_{i=1}^k \|f_i\|_{p_i}$.
 Für $r = \infty$ ist
 \[
  \left| \prod_{i=1}^k f_i \right|
  = \prod_{i=1}^k \left| f_i \right|
  \leq \prod_{i=1}^k \|f_i\|_\infty
  \text{ $\mu$-fast überall},
 \]
 also $\prod_{i=1}^k f_i \in \cL^\infty(\Omega) = \cL^r(\Omega)$ mit $\|\prod_{i=1}^k f_i\|_\infty \leq \prod_{i=1}^k \|f_i\|_\infty = \prod_{i=1}^k \|f_i\|_{p_i}$.
 
 Im Fall $1/r - 1/p_k > 0$ ist, da $1/r - 1/p_k \leq 1/r$,
 \[
  s := \frac{1}{\frac{1}{r} - \frac{1}{p_k}} \geq r \geq 1.
 \]
 Da $\sum_{i=1}^{k-1} 1/p_i = 1/s$ ist nach der Induktionsvoraussetzung $\prod_{i=1}^{k-1} f_i \in \cL^s(\Omega)$ mit
 \[
  \left\| \prod_{i=1}^{k-1} f_i \right\|_s \leq \prod_{i=1}^{k-1} \|f_i\|_{p_i}.
 \]
 Da $p_k, r, s \in [1,\infty]$ mit $1/p_k + 1/s = 1/r$, und $f_k \in \cL^{p_k}(\Omega)$ und $\prod_{i=1}^{k-1} f_i \in \cL^s(\Omega)$, ist nach Induktionsvoraussetzung
 \begin{gather*}
  \prod_{i=1}^k f_i
  = \left( \prod_{i=1}^{k-1} f_i \right) f_k \in \cL^r(\Omega)
 \shortintertext{mit}
  \left\| \prod_{i=1}^k f_i \right\|_r
  \leq \left\| \prod_{i=1}^{k-1} f_i \right\|_s \|f_k\|_{p_k}
  \leq \prod_{i=1}^k \|f_i\|_{p_i}.
 \end{gather*}
\qed\end{is}


\subsection{}
Im Folgenden wird häufig der Satz von Tonelli genutzt. Die entsprechenden Umformungen werden mit $(T)$ markiert.

Wir betrachten zunächst den Fall, dass $p = \infty$, $q = \infty$ oder $r = \infty$. Ist etwa $p = \infty$, so muss $q = r = 1$. Es ist dann
\begin{align*}
     &\, \int_{\R^n \times \R^n} |f(x) g(y-x) h(y)| \dlambda_{2n}(x,y) \\
 \leq&\, \|f\|_\infty \int_{\R^n \times \R^n} |g(y-x) h(y)| \dlambda_{2n}(x,y) \\
 \underset{(T)}=&\, \|f\|_\infty \int_{\R^n} \int_{\R^n} |g(y-x) h(y)| \dlambda_n(x) \dlambda_n(y) \\
    =&\, \|f\|_\infty \int_{\R^n} |h(y)| \int_{\R^n} |g(y-x)| \dlambda_n(x) \dlambda_n(y) \\
    =&\, \|f\|_\infty \|g\|_1 \int_{\R^n} |h(y)| \dlambda_n(y)
    =    \|f\|_\infty \|g\|_1 \|h\|_1.
\end{align*}
Die Fälle $q = \infty$ und $r = \infty$ lassen sich analog abhandeln.

Als Nächstes betrachten wir den Fall, dass $p = 1$, $q = 1$ oder $r = 1$. Ist etwa $p = 1$, so ist $1/q + 1/r = 1$ und daher
\begin{align*}
     &\, \int_{\R^n \times \R^n} |f(x) g(y-x) h(y)| \dlambda_{2n}(x,y) \\
     \underset{(T)}=&\, \int_{\R^n} \int_{\R^n} |f(x) g(y-x) h(y)| \dlambda_n(y) \dlambda_n(x) \\
     =&\, \int_{\R^n} |f(x)| \int_{\R^n} |g(y-x) h(y)| \dlambda_n(y) \dlambda_n(x) \\
     \underset{(*)}\leq&\, \|g\|_q \|h\|_r \int_{\R^n} |f(x)| \dlambda_n(x)
     = \|f\|_1 \|g\|_q \|h\|_r.
\end{align*}
Dabei wird bei $(*)$ die Hölder-Ungleichung genutzt. Die Fälle $q = 1$ und $r = 1$ lassen sich analog abhandeln. Dabei muss man im Fale $q = 1$ allerdings den Transformationssatz nutzen: Es ist dann $1/p + 1/r = 1$, und es ergibt sich, dass
\begin{align*}
 &\, \int_{\R^n \times \R^n} |f(x)g(y-x)h(y)| \dlambda_{2n}(x,y) \\
 \underset{(T)}=&\, \int_{\R^n} \int_{\R^n} |f(x)g(y-x)h(y)| \dlambda_n(y) \dlambda_n(x) \\
 =&\, \int_{\R^n} |f(x)| \int_{\R^n} |g(y-x)h(y)| \dlambda_n(y) \dlambda_n(x) \\
 \underset{(*)}=&\, \int_{\R^n} |f(x)| \int_{\R^n} |g(y)h(y+x)| \dlambda_n(y) \dlambda_n(x) \\
 =&\, \int_{\R^n} \int_{\R^n} |f(x)g(y)h(y+x)| \dlambda_n(y) \dlambda_n(x) \\
 \underset{(T)}=&\, \int_{\R^n} \int_{\R^n} |f(x)g(y)h(y+x)| \dlambda_n(x) \dlambda_n(y) \\
 =&\, \int_{\R^n} |g(y)|\int_{\R^n} |f(x)h(y+x)| \dlambda_n(x) \dlambda_n(y) \\
 \underset{(**)}\leq&\, \|f\|_p \|h\|_r \int_{\R^n} |g(y)| \dlambda_n(y)
 = \|f\|_p \|g\|_1 \|h\|_r.
\end{align*}
Dabei wird bei $(*)$ der Transformationssatz genutzt (bezüglich des Diffeomorphismus \mbox{$\varphi_x: \R^n \to \R^n, y \mapsto y+x$}) und bei $(**)$ die Hölder-Ungleichung.


Zuletzt betrachten wir Fall, dass $1 < p,q,r < \infty$. Es ist dann auch \mbox{$1 < p', q', r' < \infty$}. Da
\[
 3
 = \frac{1}{p} + \frac{1}{p'} + \frac{1}{q} + \frac{1}{q'} + \frac{1}{r} + \frac{1}{r'}
 = 2 + \frac{1}{p'} + \frac{1}{q'} + \frac{1}{r'}
\]
ist
\[
 \frac{1}{p'} + \frac{1}{q'} + \frac{1}{r'} = 1.
\]

Wir betrachten die Hilfsfunktionen $\alpha, \beta, \gamma : \R^n \times \R^n \to [0,\infty]$ mit
\begin{align*}
 \alpha(x,y) &:= |g(y-x)|^{q/p'} |h(y)|^{r/p'}, \\
  \beta(x,y) &:= |f(x)|^{p/q'} |h(y)|^{r/q'} \text{ und} \\
 \gamma(x,y) &:= |f(x)|^{p/r'} |g(y-x)|^{q/r'}.
\end{align*}
Offenbar sind $\alpha$, $\beta$ und $\gamma$ messbar.

Es ist $\alpha \in \cL^{p'}(\R^n \times \R^n)$ mit $\|\alpha\|_{p'} = \|g\|_q^{q/p'} \|h\|_r^{r/p'}$, denn
\begin{align*}
 &\, \int_{\R^n \times \R^n} |\alpha(x,y)|^{p'} \dlambda_{2n}(x,y) \\
 =&\, \int_{\R^n \times \R^n} |g(y-x)|^q |h(y)|^r \dlambda_{2n}(x,y) \\
 \underset{(T)}=&\, \int_{\R^n} \int_{\R^n} |g(y-x)|^q |h(y)|^r \dlambda_n(x) \dlambda_n(y) \\
 =&\, \int_{\R^n} |h(y)|^r \int_{\R^n} |g(y-x)|^q \dlambda_n(x) \dlambda_n(y) \\
 =&\, \|g\|_q^q \int_{\R^n} |h(y)|^r \dlambda_n(y)
 = \|g\|_q^q \|h\|_r^r.
\end{align*}
Analog ergibt sich auch, dass $\beta \in \cL^{q'}(\R^n \times \R^n)$ mit $\|\beta\|_{q'} = \|f\|_p^{p/q'} \|h\|_r^{r/q'}$, und dass $\gamma \in \cL^{r'}(\R^n \times \R^n)$ mit $\|\gamma\|_{r'} = \|f\|_p^{p/r'} \|g\|_q^{q/r'}$.

Wir bemerken nun, dass
\begin{gather*}
 \frac{p}{q'} + \frac{p}{r'}
 = p\left( \frac{1}{q'} + \frac{1}{r'} \right)
 = p\left( 1-\frac{1}{p'} \right)
 = p \, \frac{1}{p}
 = 1, \text{ sowie} \\
 \frac{q}{p'} + \frac{q}{r'}
 = 1 \text{ und }
 \frac{r}{p'} + \frac{r}{q'}
 = 1.
\end{gather*}


Damit ergibt sich aus dem vorherigen Aufgabenteil, dass
\begin{align*}
 &\,     \int_{\R^n \times \R^n} |f(x) g(y-x) h(y)| \dlambda_{2n}(x,y) \\
 =&\,    \int_{\R^n \times \R^n} |f(x)|^{p/q'+p/r'} |g(y-x)|^{q/p'+q/r'} |h(y)|^{r/p'+r/q'} \dlambda_{2n}(x,y) \\
 =&\     \int_{\R^n \times \R^n} |\alpha \beta \gamma| \dlambda_{2n}
 \leq \|\alpha\|_{p'} \|\beta\|_{q'} \|\gamma\|_{r'} \\
 =&\,    \|f\|_p^{p/q' + p/r'} \|g\|_q^{q/p' + q/r'} \|h\|_r^{r/p' + r/q'}
 =    \|f\|_p \|g\|_q \|h\|_r.
\end{align*}





\section{(Die Dichtheit von einfachen Funktionen in $L^p$)}

\begin{bem}\label{bem: abschätzung}
 Für $p \in [1,\infty)$ und $x, y \geq 0$ mit $x \leq y$ ist
 \[
  (x+y)^p \geq x^p + y^p
 \]
 und
 \[
  (y-x)^p \leq y^p - x^p.
 \]
\end{bem}
\begin{proof}
 Wir betrachten die Funktion $\psi : [0,\infty) \to \R$ mit
 \[
  \psi(t) = (x+t)^p -x^p -t^p \text{ für alle } t \in [0,\infty).
 \]
 Es ist $\psi \in C^1([0,\infty))$ mit
 \[
  \psi'(t) = p(x+t)^{p-1} -pt^{p-1} = p( (x+t)^{p-1} - t^{p-1} ).
 \]
 Es ist $\psi(0) = 0$ und $\psi'(t) \geq 0$ für alle $t \in (0,\infty)$. Also ist $\psi(t) \geq 0$ für alle $t \in [0,\infty)$. Insbesondere ist
 \[
  (x+y)^p -x^p -y^p = \psi(y) \geq 0,
 \]
 was die erste Gleichung zeigt. Da $y-x \geq 0$ ist
 \[
  y^p = (y-x+x)^p \geq (y-x)^p + x^p,
 \]
 was die zweite Gleichung zeigt.
\end{proof}

Wir betrachten die beiden Fälle $p \in [0,\infty)$ und $p = \infty$ getrennt. Zunächst betrachten wir den Fall $p \in [0,\infty)$.

Sei zunächst $f \in \cL^p(\Omega)$ mit $f \geq 0$. Wie wir wissen gibt es eine Folge $(f_n)_{n \in \N}$ einfacher, messbarer Funktionen mit $f_n \geq 0$ und $f_n \leq f_{n+1}$ für alle $n \in \N$, sowie $\limes{n}{\infty} f_n(x) = f(x)$ für alle $x \in \Omega$. Aufgrund von Monotonie ist $0 \leq f^p$, sowie $0 \leq f_n^p$ und $f_n^p \leq f_{n+1}^p$ für alle $n \in \N$, und aufgrund von Stetigkeit ist
\[
 \limes{n}{\infty} f_n(x)^p = f(x)^p \text{ für alle } x \in \Omega.
\]
Nach dem Satz über monotone Konvergenz ist daher
\[
 \limes{n}{\infty} \int_\Omega f_n^p \dmu = \int_\Omega f^p \dmu.
\]
Da $f \in \cL^p(\Omega)$ und $f \geq 0$ ist die rechte Seite der Gleichung endlich, wegen der Monotonie von $(f^p_n)_{n \in \N}$ also auch auch die linke für alle $n \in \N$. Es ist daher $(f_n)_{n \in \N}$ eine Folge einfacher Funktionen auf $\cL^p(\Omega)$ mit
\[
 \limes{n}{\infty} \int_\Omega f^p - f_n^p \dmu
 = \int_\Omega f^p \dmu - \limes{n}{\infty} \int_\Omega f_n^p \dmu
 = 0.
\]
Nach Bemerkung \ref{bem: abschätzung} ist daher
\[
 0
 \leq \limes{n}{\infty} \int_\Omega (f - f_n)^p \dmu
 \leq \limes{n}{\infty} \int_\Omega f^p - f_n^p \dmu = 0,
\]
also
\[
 \limes{n}{\infty} \int_\Omega (f - f_n)^p \dmu = 0
\]
und damit $\limes{n}{\infty} \|f-f_n\|_p = 0$.

Sei nun $f \in \cL^p(\Omega)$ beliebig. Es ist auch $|f| \in \cL^p(\Omega)$, und es gibt eine Folge einfacher Funktionen $(f'_n)_{n \in \N}$ auf $\cL^p(\Omega)$ mit $\limes{n}{\infty} \||f|-f'_n\|_p = 0$. Wir definieren die Folge einfacher Funktionen $(f_n)_{n \in \N}$ auf $\cL^p(\Omega)$ als
\[
 f_n := \sgn(f)f'_n \text{ für alle } n \in \N.
\]
Dass $f_n \in \cL^p(\Omega)$ für alle $n \in \N$ folgt direkt daraus, dass $|f'_n| = |f_n|$. Auch ist
\[
 |f(x)-f_n(x)| \leq ||f(x)|-f'_n(x)| \text{ für alle } x \in \Omega, n \in \N.
\]
Es ist daher
\[
 \|f-f_n\|_p
 = \left( \int_\Omega |f-f_n|^p \dmu \right)^{1/p}
 \leq \left( \int_\Omega ||f|-f'_n|^p \right)^{1/p}
 = \||f|-f'_n\|_p
\]
für alle $n \in \N$. Aus $\limes{n}{\infty} \||f|-f'_n\|_p = 0$ folgt daher, dass
\[
 \limes{n}{\infty} \|f-f_n\|_p = 0.
\]

Sei nun $p = \infty$. Sei zunächst $f \in \cL^\infty(\Omega)$ mit $f \geq 0$. Wir definieren die Folge einfacher Funktionen $(f_n)_{n \geq 1}$ auf $\cL^\infty(\Omega)$ als
\[
 f_n(x) := \sup \left\{ \frac{k}{n} \leq f(x) : k \in \N \right\} \text{ für alle } x \in \
 \Omega, n \geq 1.
\]
Dass $f_n$ für alle $n \in \N$ eine einfache Funktion ist, und damit inbesondere $f_n \in \cL^\infty(\Omega)$ für alle $n \in \N$, folgt direkt daraus, dass \mbox{$f \in \cL^\infty(\Omega)$}. Es ist auch klar, dass
\[
 \sup_{x \in \Omega} |f(x) - f_n(x)| < \frac{1}{n} \text{ für alle } n \geq 1.
\]
Es ist daher $\|f-f_n\|_\infty \leq 1/n$ für alle $n \geq 1$, also  $\limes{n}{\infty} \|f-f_n\|_\infty = 0$.

Sei nun $f \in \cL^\infty(\Omega)$ beliebig. Offenbar ist auch $|f| \in \cL^\infty(\Omega)$, und wir finden eine Folge einfacher Funktionen $(f'_n)_{n \in \N}$ auf $\cL^\infty(\Omega)$ mit $\limes{n}{\infty} \||f|-f'_n\|_\infty = 0$. Wir definieren die Folge $(f_n)_{n \in \N}$ einfacher Funktionen auf $\cL^\infty(\Omega)$ als
\[
 f_n := \sgn(f)f'_n.
\]
Für alle $n \in \N$ folgt $f_n \in \cL^\infty(\Omega)$ direkt daraus, dass $|f_n| = |f'_n|$. Da
\[
 |f(x)-f_n(x)| \leq ||f(x)|-f'_n(x)| \text{ für alle } x \in \Omega, n \in \N
\]
ist $\|f-f_n\|_\infty \leq \||f|-f'_n\|_\infty$ für alle $n \in \N$. Aus $\limes{n}{\infty} \||f|-f'_n\|_\infty = 0$ folgt daher, dass $\limes{n}{\infty} \|f-f_n\|_\infty = 0$.





\section{(Schwache Konvergenz)}


\subsection{}
Sei $g \in \cL^{p'}(\Omega)$ beliebig aber fest. Da $f, f_k \in \cL^p(\Omega)$ für alle $k \in \N$ ist nach der Hölder-Ungleichung $f g, f_k g \in \cL^1(\Omega)$ für alle $k \in \N$. Es ist daher
\[
 \limes{k}{\infty} \int_\Omega f_k g \dmu = \int_\Omega f g \dmu
\]
genau dann, wenn
\[
 0
 = \limes{k}{\infty} \int_\Omega f_k g \dmu - \int_\Omega f g \dmu
 = \limes{k}{\infty} \int_\Omega (f_k - f)g \dmu.
\]
Da auch $f_k - f \in \cL^p(\Omega)$ folgt mithilfe der Hölder-Ungleichung, dass für alle $k \in \N$
\[
 \left| \int_\Omega (f_k -f)g \right|
 \leq \int_\Omega |(f_k -f)g| \dmu
 \leq \|f_k-f\|_p \|g\|_{p'}.
\]
Da $f_k \longrightarrow f$ in $L^p(\Omega)$ ist $\limes{k}{\infty} \|f_k -f\|p = 0$, also auch
\[
 \limes{k}{\infty} \int_\Omega (f_k -f)g \dmu = 0.
\]


\subsection{}
Aus der Eindeutigkeit des Grenzwertes auf $\R$ folgt aus $f_k \rightharpoonup f$ und $f_k \rightharpoonup h$, dass für alle $g \in \cL^{p'}(\Omega)$
\[
 \int_\Omega f g \dmu = \limes{k}{\infty} \int_\Omega f_k g \dmu = \int_\Omega h g \dmu.
\]
Da nach der Hölder-Ungleichung beide Seiten der Gleichung endlich sind, folgt, dass
\[
 \int_\Omega (f-h) g \dmu = 0 \text{ für alle } g \in \cL^{p'}(\Omega).
\]
Daher ergibt sich aus \textbf{Aufgabe 1} wegen der $\sigma$-Endlichkeit des Maßraumes, und da $f-h \in \cL^p(\Omega)$, dass
\[
 \|f-h\|_p
 = \sup\left\{ \int_\Omega (f-h) g \dmu : g \in \cL^{p'}(\Omega), \|g\|_{p'} \leq 1 \right\}
 = 0.
\]
Also ist $f = h$ in $L^p(\Omega)$, da $\|\cdot\|_p$ eine Norm auf $L^p$ ist.


\subsection{}
\begin{bem} \label{bem: liminf sup vertauschen}
 Sei $I$ eine Indexmenge und für alle $i \in I$ sei $(a_n^i)_{n \in \N}$ eine Folge auf $\R$. Dann ist
 \[
  \sup_{i \in I} \limesinf{n} a_n^i
  \leq \limesinf{n} \sup_{i \in I} a_n^i.
 \]
\end{bem}
\begin{proof}
 Für alle $j \in I$ und $n \in \N$ ist
 \[
  a_n^j \leq \sup_{i \in I} a_n^i.
 \]
 Also ist für alle $j \in I$
 \[
  \limesinf{n} a_n^j \leq \limesinf{n} \sup_{i \in I} a_n^i
 \]
 und deshalb auch
 \[
  \sup_{j \in I} \limesinf{n} a_n^j \leq \limesinf{n} \sup_{i \in I} a_n^i.
 \]
\end{proof}

Zusammenfügen von \textbf{Aufgabe 1} und Bemerkung \ref{bem: liminf sup vertauschen} ergibt, dass
\begin{align*}
 \|f\|_p
 &= \sup \left\{ \int_\Omega f g \dmu : g \in \cL^{p'}(\Omega), \|g\|_{p'} \leq 1 \right\} \\
 &= \sup \left\{ \limesinf{k} \int_\Omega f_k g \dmu : g \in \cL^{p'}(\Omega), \|g\|_{p'} \leq 1 \right\} \\
 &\leq \limesinf{k} \sup \left\{ \int_\Omega f_k g \dmu : g \in \cL^{p'}(\Omega), \|g\|_{p'} \leq 1 \right\}
 = \limesinf{k} \|f_k\|_p.
\end{align*}




























\end{document}
