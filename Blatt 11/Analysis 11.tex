\documentclass[a4paper,10pt]{article}
%\documentclass[a4paper,10pt]{scrartcl}

\usepackage{xltxtra}
\usepackage[ngerman]{babel}
\usepackage{amsmath}
\usepackage{amssymb}
\usepackage{amsthm}
\usepackage{mathtools}
\usepackage{nicefrac}
\usepackage{enumerate}
\usepackage{leftidx}
\usepackage{stmaryrd} % for \longarrownot

\makeatletter
\g@addto@macro\th@definition{\thm@headpunct{:}}
\makeatother
\newcounter{saetze}
\newtheorem{lem}[saetze]{Lemma}
\newtheorem{beh}[saetze]{Behauptung}
\newtheorem{bem}[saetze]{Bemerkung}
\newtheorem*{ia}{Induktionsanfang}
\newtheorem*{is}{Induktionsschritt}

\renewcommand{\thesection}{Aufgabe \arabic{section}.}
\renewcommand{\thesubsection}{\alph{subsection})}
\renewcommand{\thesubsubsection}{(\roman{subsubsection})}

\makeatletter
\def\moverlay{\mathpalette\mov@rlay}
\def\mov@rlay#1#2{\leavevmode\vtop{%
   \baselineskip\z@skip \lineskiplimit-\maxdimen
   \ialign{\hfil$\m@th#1##$\hfil\cr#2\crcr}}}
\newcommand{\charfusion}[3][\mathord]{
    #1{\ifx#1\mathop\vphantom{#2}\fi
        \mathpalette\mov@rlay{#2\cr#3}
      }
    \ifx#1\mathop\expandafter\displaylimits\fi}
\makeatother

\newcommand{\N}{\mathbb{N}}
\newcommand{\Z}{\mathbb{Z}}
\newcommand{\Q}{\mathbb{Q}}
\newcommand{\R}{\mathbb{R}}
\newcommand{\C}{\mathbb{C}}
\newcommand{\A}{\mathcal{A}}
\newcommand{\La}{\mathcal{L}}
\newcommand{\dr}{\,\text{d}r}
\newcommand{\dt}{\,\text{d}t}
\newcommand{\du}{\,\text{d}u}
\newcommand{\dx}{\,\text{d}x}
\newcommand{\dy}{\,\text{d}y}
\newcommand{\dmu}{\,\text{d}\mu}
\newcommand{\dlambda}{\,\text{d}\lambda}
\newcommand{\dvarphi}{\,\text{d}\varphi}
\newcommand{\Img}{\operatorname{Im}}
\newcommand{\Real}{\operatorname{Re}}
\newcommand{\Imag}{\operatorname{Im}}
\newcommand{\sgn}{\operatorname{sgn}}
\newcommand{\diag}{\operatorname{diag}}
\newcommand{\Vol}{\operatorname{Vol}}
\newcommand{\dotcup}{\ensuremath{\mathaccent\cdot\cup}}
\newcommand{\bigdotcup}{\charfusion[\mathop]{\bigcup}{\cdot}}
\newcommand{\nlongrightarrow}{\longarrownot\longrightarrow}
\newcommand{\ceil}[1]{\left\lceil{#1}\right\rceil}
\newcommand{\floor}[1]{\left\lfloor{#1}\right\rfloor}
\newcommand{\mc}[1]{\mathcal{#1}}
\newcommand{\limes}[2]{\lim_{#1 \to #2}}
\newcommand{\limessup}[1]{\limsup_{#1 \to \infty}}
\newcommand{\limesinf}[1]{\liminf_{#1 \to \infty}}
\newcommand{\vect}[1]{\begin{pmatrix}#1\end{pmatrix}}
\newcommand{\partd}[2]{\frac{\partial #1}{\partial #2}}
\newcommand{\op}[1]{\left\|#1\right\|_{\text{op}}}

\makeatletter
\renewcommand*\env@matrix[1][*\c@MaxMatrixCols c]{%
  \hskip -\arraycolsep
  \let\@ifnextchar\new@ifnextchar
  \array{#1}}
\makeatother

\setromanfont[Mapping=tex-text]{Linux Libertine O}
% \setsansfont[Mapping=tex-text]{DejaVu Sans}
% \setmonofont[Mapping=tex-text]{DejaVu Sans Mono}
\parindent 0pt

\title{\sc Analysis III \\ \Large 11. Aufgabenblatt}
\author{Jendrik Stelzner}
\date{\today}

\begin{document}
\maketitle





\addtocounter{section}{1}





\section{(Die Hölder-Ungleichung)}


\subsection{}
Ist $r = \infty$, so ist auch $p = q = \infty$. Für $f,g \in \La^\infty(\Omega)$ ist dann
\[
 |f(x)g(x)| = |f(x)| |g(x)| \leq \|f\|_\infty \|g\|_\infty \text{ für $\mu$-fast alle } x \in \Omega.
\]
Das zeigt, dass $fg \in \La^\infty(\Omega)$ und die geforderte Ungleichung gilt.

Ist $r \in [1,\infty)$, so folgt aus $f \in \La^p(\Omega)$, dass $|f|^r \in \La^{p/r}(\Omega)$ mit $\||f|^r\|_{p/r} = \|f\|_p^r$, und aus $g \in \La^q(\Omega)$, dass $|g|^r \in \La^{q/r}(\Omega)$ mit $\||g|^r\|_{q/r} = \|f\|_q^r$. Da
\[
 \frac{r}{p} + \frac{r}{q} = r\left(\frac{1}{p} + \frac{1}{q}\right) = r\,\frac{1}{r} = 1
\]
ist nach der Hölder-Ungleichung
\[
 \int_\Omega |fg|^r \dmu
 = \int_\Omega |f|^r |g|^r \dmu
 \leq \||f|^r\|_{p/r} \||g|^r\|_{q/r}
 = \|f\|_p^r \|g\|_q^r.
\]
Dass die rechte Seite der Gleichung endlich ist, zeigt, dass $fg \in \La^r(\Omega)$. Außerdem folgt, dass
\[
 \|fg\|_r
 = \left( \int_\Omega |fg|^r \dmu \right)^{1/r}
 \leq \|f\|_p \|g\|_q.
\]





\section{(Die Dichtheit von einfachen Funktionen in $L^p$)}

\begin{bem}\label{bem: abschätzung}
 Für $p \in [1,\infty)$ und $x, y \geq 0$ mit $x \leq y$ ist
 \[
  (x+y)^p \geq x^p + y^p
 \]
 und
 \[
  (y-x)^p \leq y^p - x^p.
 \]
\end{bem}
\begin{proof}
 Wir betrachten die Funktion $\psi : [0,\infty) \to [0,\infty)$ mit
 \[
  \psi(t) = (x+t)^p -x^p -t^p \text{ für alle } t \in [0,\infty).
 \]
 Es ist $\psi \in C^1([0,\infty))$ mit
 \[
  \psi'(t) = p(x+t)^{p-1} -pt^{p-1} = p( (x+t)^{p-1} - t^{p-1} ).
 \]
 Es ist $\psi(0) = 0$ und $\psi'(t) \geq 0$ für alle $t \in (0,\infty)$. Also ist $\psi(t) \geq 0$ für alle $t \in [0,\infty)$. Insbesondere ist
 \[
  (x+y)^p -x^p -y^p = \psi(y) \geq 0,
 \]
 was die erste Gleichung zeigt. Da $y-x \geq 0$ ist
 \[
  y^p = (y-x+x)^p \geq (y-x)^p + x^p,
 \]
 was die zweite Gleichung zeigt.
\end{proof}

Wir betrachten die beiden Fälle $p \in [0,\infty)$ und $p = \infty$ getrennt. Zunächst betrachten wir den Fall $p \in [0,\infty)$.

Sei zunächst $f \in \La^p(\Omega)$ mit $f \geq 0$. Wie wir wissen gibt es eine Folge $(f_n)_{n \in \N}$ einfacher, messbarer Funktionen mit $f_n \geq 0$ und $f_n \leq f_{n+1}$ für alle $n \in \N$, sowie $\limes{n}{\infty} f_n(x) = f(x)$ für alle $x \in \Omega$. Aufgrund von Monotonie ist $0 \leq f^p$, sowie $0 \leq f_n^p$ und $f_n^p \leq f_{n+1}^p$ für alle $n \in \N$, und aufgrund von Stetigkeit ist
\[
 \limes{n}{\infty} f_n(x)^p = f(x)^p \text{ für alle } x \in \Omega.
\]
Nach dem Satz über monotone Konvergenz ist daher
\[
 \limes{n}{\infty} \int_\Omega f_n^p \dmu = \int_\Omega f^p \dmu.
\]
Da $f \in \La^p(\Omega)$ und $f \geq 0$ ist die rechte Seite der Gleichung endlich, wegen der Monotonie von $(f^p_n)_{n \in \N}$ also auch auch die linke für alle $n \in \N$. Es ist daher $(f_n)_{n \in \N}$ eine Folge einfacher Funktionen auf $\La^p(\Omega)$ mit
\[
 \limes{n}{\infty} \int_\Omega f^p - f_n^p \dmu
 = \int_\Omega f^p \dmu - \limes{n}{\infty} \int_\Omega f_n^p \dmu
 = 0.
\]
Nach Bemerkung \ref{bem: abschätzung} ist daher
\[
 0 \leq \limes{n}{\infty} \int_\Omega (f - f_n)^p \dmu \leq \limes{n}{\infty} f^p - f_n^p \dmu = 0,
\]
also
\[
 \limes{n}{\infty} \int_\Omega (f - f_n)^p \dmu = 0
\]
und damit $\limes{n}{\infty} \|f-f_n\|_p = 0$.

Sei nun $f \in \La^p(\Omega)$ beliebig. Es ist auch $|f| \in \La^p(\Omega)$, und es gibt eine Folge einfacher Funktionen $(f'_n)_{n \in \N}$ auf $\La^p(\Omega)$ mit $\limes{n}{\infty} \||f|-f'_n\|_p = 0$. Wir definieren die Folge einfacher Funktionen $(f_n)_{n \in \N}$ auf $\La^p(\Omega)$ als
\[
 f_n := \sgn(f)f'_n \text{ für alle } n \in \N.
\]
Dass $f_n \in \La^p(\Omega)$ für alle $n \in \N$ folgt direkt daraus, dass $|f'_n| = |f_n|$. Auch ist
\[
 |f(x)-f_n(x)| \geq ||f(x)|-f'_n(x)| \text{ für alle } x \in \Omega \text{ für alle } n \in \N.
\]
Es ist daher
\[
 \|f-f_n\|_p
 = \left( \int_\Omega |f-f_n|^p \dmu \right)^{1/p}
 \leq \left( \int_\Omega ||f|-f'_n|^p \right)^{1/p}
 = \||f|-f'_n\|_p
\]
für alle $n \in \N$. Aus $\limes{n}{\infty} \||f|-f'_n\|_p = 0$ folgt daher, dass $\limes{n}{\infty} \|f-f_n\|_p = 0$.

Sei nun $p = \infty$. Sei zunächst $f \in \La^\infty(\Omega)$ mit $f \geq 0$. Für $n \in \N$ definieren wir die Folge einfacher Funktionen $(f_n)_{n \geq 1}$ auf $\La^\infty(\Omega)$ als
\[
 f_n(x) := \sup \left\{ \frac{k}{n} \leq f(x) : k \in \N \right\}.
\]
Dass $n \in \N$ $f_n \in \La^\infty(\Omega)$ und $f_n$ einfach ist, folgt für alle $n \in \N$ direkt daraus, dass \mbox{$f \in \La^\infty(\Omega)$}. Es ist auch klar, dass
\[
 \sup_{x \in \Omega} |f(x) - f_n(x)| < \frac{1}{n} \text{ für alle } n \geq 1.
\]
Es ist daher $\|f-f_n\|_\infty \leq 1/n$ für alle $n \geq 1$, also  $\limes{n}{\infty} \|f-f_n\|_\infty = 0$.

Sei nun $f \in \La^\infty(\Omega)$ beliebig. Offenbar ist auch $|f| \in \La^\infty(\Omega)$, und wir finden eine Folge einfacher Funktionen $(f'_n)_{n \in \N}$ auf $\La^\infty(\Omega)$ mit $\limes{n}{\infty} \||f|-f'_n\|_\infty = 0$. Wir definieren die Folge $(f_n)_{n \in \N}$ einfacher Funktionen auf $\La^\infty(\Omega)$ als
\[
 f_n := \sgn(f)f'_n.
\]
Dass $f_n \in \La^\infty(\Omega)$ für alle $n \in \N$ folgt direkt daraus, dass $|f_n| = |f'_n|$ für alle $n \in \N$. Da
\[
 |f(x)-f_n(x)| \leq ||f(x)|-f'_n(x)| \text{ für alle } x \in \Omega \text{ für alle } n \in \N
\]
ist $\|f-f_n\|_\infty \leq \||f|-f'\|_\infty$ für alle $n \in \N$. Aus $\limes{n}{\infty} \||f|-f'_n\|_\infty = 0$ folgt daher, dass $\limes{n}{\infty} \|f-f_n\|_\infty = 0$.





\section{(Schwache Konvergenz)}


\subsection{}
Sei $g \in \La^{p'}(\Omega)$ beliebig aber fest. Da $f, f_k \in \La^p(\Omega)$ für alle $k \in \N$ ist nach der Hölder-Ungleichung $f g, f_k g \in \La^1(\Omega)$ für alle $k \in \N$. Es ist daher
\[
 \limes{k}{\infty} \int_\Omega f_k g \dmu = \int_\Omega f g \dmu
\]
genau dann, wenn
\[
 0
 = \limes{k}{\infty} \int_\Omega f_k g \dmu - \int_\Omega f g \dmu
 = \limes{k}{\infty} \int_\Omega (f_k - f)g \dmu.
\]
Da auch $f_k - f \in \La^p(\Omega)$ folgt mithilfe der Hölder-Ungleichung, dass für alle $k \in \N$
\[
 \left| \int_\Omega (f_k -f)g \right|
 \leq \int_\Omega |f_k -f| |g| \dmu
 \leq \||f_k-f|\|_p \||g|\|_{p'}
 = \|f_k-f\|_p \|g\|_{p'}.
\]
Da $f_k \longrightarrow f$ in $L^p(\Omega)$ ist $\limes{k}{\infty} \|f_k -f\|p = 0$, also auch
\[
 \limes{k}{\infty} \int_\Omega (f_k -f)g \dmu = 0.
\]


\subsection{}
Aus der Eindeutigkeit des Grenzwertes auf $\R$ folgt aus $f_k \rightharpoonup f$ und $f_k \rightharpoonup h$, dass für alle $g \in \La^{p'}(\Omega)$
\[
 \int_\Omega f g \dmu = \limes{k}{\infty} \int_\Omega f_k g \dmu = \int_\Omega h g \dmu.
\]
Da nach der Hölder-Ungleichung beide Seiten der Gleichung endlich sind, folgt, dass
\[
 \int_\Omega (f-h) g \dmu = 0 \text{ für alle } g \in \La^{p'}(\Omega).
\]
Daher ergibt sich aus \textbf{Aufgabe 1} wegen der $\sigma$-Endlichkeit des Maßraumes, und da $f-h \in L^p(\Omega)$, dass
\[
 \|f-h\|_p
 = \sup\left\{ \int_\Omega (f-h) g \dmu : g \in \La^{p'}(\Omega) \text{ mit } \|g\|_{p'} \leq 1 \right\}
 = 0.
\]
Also ist $f = h$ in $L^p(\Omega)$, da $\|\cdot\|_p$ eine Norm auf $L^p$ ist.
































\end{document}
