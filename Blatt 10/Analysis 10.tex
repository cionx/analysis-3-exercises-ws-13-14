\documentclass[a4paper,10pt]{article}
%\documentclass[a4paper,10pt]{scrartcl}

\usepackage{xltxtra}
\usepackage[ngerman]{babel}
\usepackage{amsmath}
\usepackage{amssymb}
\usepackage{amsthm}
\usepackage{mathtools}
\usepackage{nicefrac}
\usepackage{enumerate}
\usepackage{leftidx}

\makeatletter
\g@addto@macro\th@definition{\thm@headpunct{:}}
\makeatother
\newcounter{saetze}
\newtheorem{lem}[saetze]{Lemma}
\newtheorem{beh}[saetze]{Behauptung}
\newtheorem{bem}[saetze]{Bemerkung}
\newtheorem*{ia}{Induktionsanfang}
\newtheorem*{is}{Induktionsschritt}

\renewcommand{\thesection}{Aufgabe \arabic{section}.}
\renewcommand{\thesubsection}{\alph{subsection})}
\renewcommand{\thesubsubsection}{(\roman{subsubsection})}

\makeatletter
\def\moverlay{\mathpalette\mov@rlay}
\def\mov@rlay#1#2{\leavevmode\vtop{%
   \baselineskip\z@skip \lineskiplimit-\maxdimen
   \ialign{\hfil$\m@th#1##$\hfil\cr#2\crcr}}}
\newcommand{\charfusion}[3][\mathord]{
    #1{\ifx#1\mathop\vphantom{#2}\fi
        \mathpalette\mov@rlay{#2\cr#3}
      }
    \ifx#1\mathop\expandafter\displaylimits\fi}
\makeatother

\newcommand{\N}{\mathbb{N}}
\newcommand{\Z}{\mathbb{Z}}
\newcommand{\Q}{\mathbb{Q}}
\newcommand{\R}{\mathbb{R}}
\newcommand{\C}{\mathbb{C}}
\newcommand{\A}{\mathcal{A}}
\newcommand{\La}{\mathcal{L}}
\newcommand{\dr}{\,\text{d}r}
\newcommand{\dt}{\,\text{d}t}
\newcommand{\du}{\,\text{d}u}
\newcommand{\dx}{\,\text{d}x}
\newcommand{\dy}{\,\text{d}y}
\newcommand{\dmu}{\,\text{d}\mu}
\newcommand{\dlambda}{\,\text{d}\lambda}
\newcommand{\dvarphi}{\,\text{d}\varphi}
\newcommand{\Img}{\operatorname{Im}}
\newcommand{\Real}{\operatorname{Re}}
\newcommand{\Imag}{\operatorname{Im}}
\newcommand{\sgn}{\operatorname{sgn}}
\newcommand{\diag}{\operatorname{diag}}
\newcommand{\Vol}{\operatorname{Vol}}
\newcommand{\dotcup}{\ensuremath{\mathaccent\cdot\cup}}
\newcommand{\bigdotcup}{\charfusion[\mathop]{\bigcup}{\cdot}}
\newcommand{\ceil}[1]{\left\lceil{#1}\right\rceil}
\newcommand{\floor}[1]{\left\lfloor{#1}\right\rfloor}
\newcommand{\mc}[1]{\mathcal{#1}}
\newcommand{\limes}[2]{\lim_{#1 \rightarrow #2}}
\newcommand{\limessup}[1]{\limsup_{#1 \rightarrow \infty}}
\newcommand{\limesinf}[1]{\liminf_{#1 \rightarrow \infty}}
\newcommand{\vect}[1]{\begin{pmatrix}#1\end{pmatrix}}
\newcommand{\partd}[2]{\frac{\partial #1}{\partial #2}}
\newcommand{\op}[1]{\left\|#1\right\|_{\text{op}}}

\makeatletter
\renewcommand*\env@matrix[1][*\c@MaxMatrixCols c]{%
  \hskip -\arraycolsep
  \let\@ifnextchar\new@ifnextchar
  \array{#1}}
\makeatother

\setromanfont[Mapping=tex-text]{Linux Libertine O}
% \setsansfont[Mapping=tex-text]{DejaVu Sans}
% \setmonofont[Mapping=tex-text]{DejaVu Sans Mono}
\parindent 0pt

\title{\sc Analysis III \\ \Large 10. Aufgabenblatt}
\author{Jendrik Stelzner}
\date{\today}

\begin{document}
\maketitle





\section{Gegenbeispiele}


\subsection{}
Wir betrachten die Funktion $f : (0,1) \rightarrow \R, x \mapsto 1/\sqrt{x}$. Diese ist auf auf jedem kompakten Teilintervall von $(0,1)$ Riemann-integrierbar und $f = |f|$ ist auf $(0,1)$ uneigentlich Riemann-integrierbar mit
\[
 \int_0^1 f(x) \dx
 = \int_0^1 \frac{1}{\sqrt{x}} \dx
 = \lim_{s \downarrow 0} \int_s^1 \frac{1}{\sqrt{x}} \dx
 = \lim_{s \downarrow 0} \left. 2 \sqrt{x} \right|_{x=s}^1
 = \lim_{s \downarrow 0} 2 - 2\sqrt{s}
 = 2.
\]
Also ist $f$ auf $(0,1)$ auch Lebesgue-integrierbar, weshalb $f \in \La((0,1))$. Es ist jedoch $f^2 \not\in \La^2((0,1))$: Für alle $x \in (0,1)$ ist $|f(x)^2| = f(x)^2 = 1/x$. Für die Funktionenfolge
\[
 h_n := \chi_{[1/n,1)} f^2
\]
gilt auf $(0,1)$ überall $h_n \leq h_{n+1}$ für alle $n \in \N$ und $\limes{n}{\infty} h_n = f^2$. Nach dem Satz über monotone Konvergenz ist also
\[
 \int_{(0,1)} |f|^2 \dlambda
 = \limes{n}{\infty} \int_{(0,1)} h_n \dlambda
 = \limes{n}{\infty} \int_{[1/n,1)} h_n \dlambda
 = \limes{n}{\infty} \int_{1/n}^{1} \frac{1}{x} \dx
 = \infty.
\]


\subsection{}
Wir betrachten die Funktion $f : \R \rightarrow \R, x \mapsto 1/(x+1)$. Es ist $f \not\in \La(\R)$: Für die Funktionenfolge
\[
 f_n := \chi_{[-n,n]} f
\]
ist auf $\R$ überall $f_n \leq f_{n+1}$ für alle $n \in \N$ und $\limes{n}{\infty} f_n = f = |f|$. Daher ist nach dem Satz über monotone Konvergenz
\begin{align*}
 \int_\R |f| \dlambda
 &= \limes{n}{\infty} \int_\R f_n \dlambda
 = \limes{n}{\infty} \int_{[-n,n]} f_n \dlambda \\
 &= \limes{n}{\infty} \int_{-n}^n \frac{1}{x+1} \dx
 = 2 \limes{n}{\infty} \int_1^{n+1} \frac{1}{x} \dx \\
 &= 2 \int_1^\infty \frac{1}{x} \dx
 = \infty.
\end{align*}

Es ist jedoch $f \in \La^2(\R)$: Für die Funktionenfolge
\[
 h_n := \chi_{[-n,n]} f^2
\]
ist auf $\R$ überall $h_n \leq h_{n+1}$ für alle $n \in \N$ und $\limes{n}{\infty} h_n = f^2 = |f|^2$. Also ist nach dem Satz über monotone Konvergenz
\begin{align*}
 \int_\R f^2 \dlambda
 &= \limes{n}{\infty} \int_\R h_n \dlambda
 = \limes{n}{\infty} \int_{[-n,n]} f^2 \dlambda \\
 &= \limes{n}{\infty} \int_{-n}^n \frac{1}{(x+1)^2} \dx
 = 2 \limes{n}{\infty} \int_1^{n+1} \frac{1}{x^2} \dx \\
 &= 2 \int_1^\infty \frac{1}{x^2}
 < \infty.
\end{align*}


















\end{document}
