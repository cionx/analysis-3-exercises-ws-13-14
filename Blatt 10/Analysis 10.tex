\documentclass[a4paper,10pt]{article}
%\documentclass[a4paper,10pt]{scrartcl}

\usepackage{xltxtra}
\usepackage[ngerman]{babel}
\usepackage{amsmath}
\usepackage{amssymb}
\usepackage{amsthm}
\usepackage{mathtools}
\usepackage{nicefrac}
\usepackage{enumerate}
\usepackage{leftidx}
\usepackage{stmaryrd} % for \longarrownot

\makeatletter
\g@addto@macro\th@definition{\thm@headpunct{:}}
\makeatother
\newcounter{saetze}
\newtheorem{lem}[saetze]{Lemma}
\newtheorem{beh}[saetze]{Behauptung}
\newtheorem{bem}[saetze]{Bemerkung}
\newtheorem*{ia}{Induktionsanfang}
\newtheorem*{is}{Induktionsschritt}

\renewcommand{\thesection}{Aufgabe \arabic{section}.}
\renewcommand{\thesubsection}{\alph{subsection})}
\renewcommand{\thesubsubsection}{(\roman{subsubsection})}

\makeatletter
\def\moverlay{\mathpalette\mov@rlay}
\def\mov@rlay#1#2{\leavevmode\vtop{%
   \baselineskip\z@skip \lineskiplimit-\maxdimen
   \ialign{\hfil$\m@th#1##$\hfil\cr#2\crcr}}}
\newcommand{\charfusion}[3][\mathord]{
    #1{\ifx#1\mathop\vphantom{#2}\fi
        \mathpalette\mov@rlay{#2\cr#3}
      }
    \ifx#1\mathop\expandafter\displaylimits\fi}
\makeatother

\newcommand{\N}{\mathbb{N}}
\newcommand{\Z}{\mathbb{Z}}
\newcommand{\Q}{\mathbb{Q}}
\newcommand{\R}{\mathbb{R}}
\newcommand{\C}{\mathbb{C}}
\newcommand{\A}{\mathcal{A}}
\newcommand{\La}{\mathcal{L}}
\newcommand{\dr}{\,\text{d}r}
\newcommand{\dt}{\,\text{d}t}
\newcommand{\du}{\,\text{d}u}
\newcommand{\dx}{\,\text{d}x}
\newcommand{\dy}{\,\text{d}y}
\newcommand{\dmu}{\,\text{d}\mu}
\newcommand{\dlambda}{\,\text{d}\lambda}
\newcommand{\dvarphi}{\,\text{d}\varphi}
\newcommand{\Img}{\operatorname{Im}}
\newcommand{\Real}{\operatorname{Re}}
\newcommand{\Imag}{\operatorname{Im}}
\newcommand{\sgn}{\operatorname{sgn}}
\newcommand{\diag}{\operatorname{diag}}
\newcommand{\Vol}{\operatorname{Vol}}
\newcommand{\dotcup}{\ensuremath{\mathaccent\cdot\cup}}
\newcommand{\bigdotcup}{\charfusion[\mathop]{\bigcup}{\cdot}}
\newcommand{\nlongrightarrow}{\longarrownot\longrightarrow}
\newcommand{\ceil}[1]{\left\lceil{#1}\right\rceil}
\newcommand{\floor}[1]{\left\lfloor{#1}\right\rfloor}
\newcommand{\mc}[1]{\mathcal{#1}}
\newcommand{\limes}[2]{\lim_{#1 \to #2}}
\newcommand{\limessup}[1]{\limsup_{#1 \to \infty}}
\newcommand{\limesinf}[1]{\liminf_{#1 \to \infty}}
\newcommand{\vect}[1]{\begin{pmatrix}#1\end{pmatrix}}
\newcommand{\partd}[2]{\frac{\partial #1}{\partial #2}}
\newcommand{\op}[1]{\left\|#1\right\|_{\text{op}}}

\makeatletter
\renewcommand*\env@matrix[1][*\c@MaxMatrixCols c]{%
  \hskip -\arraycolsep
  \let\@ifnextchar\new@ifnextchar
  \array{#1}}
\makeatother

\setromanfont[Mapping=tex-text]{Linux Libertine O}
% \setsansfont[Mapping=tex-text]{DejaVu Sans}
% \setmonofont[Mapping=tex-text]{DejaVu Sans Mono}
\parindent 0pt

\title{\sc Analysis III \\ \Large 10. Aufgabenblatt}
\author{Jendrik Stelzner}
\date{\today}

\begin{document}
\maketitle





\section{(Gegenbeispiele)}


\subsection{}
Wir betrachten die Funktion $f : (0,1) \to \R, x \mapsto 1/\sqrt{x}$. Diese ist auf auf jedem kompakten Teilintervall von $(0,1)$ Riemann-integrierbar und $f = |f|$ ist auf $(0,1)$ uneigentlich Riemann-integrierbar mit
\[
 \int_0^1 f(x) \dx
 = \int_0^1 \frac{1}{\sqrt{x}} \dx
 = \lim_{s \downarrow 0} \int_s^1 \frac{1}{\sqrt{x}} \dx
 = \lim_{s \downarrow 0} \left. 2 \sqrt{x} \right|_{x=s}^1
 = \lim_{s \downarrow 0} 2 - 2\sqrt{s}
 = 2.
\]
Also ist $f$ auf $(0,1)$ auch Lebesgue-integrierbar, weshalb $f \in \La((0,1))$. Es ist jedoch $f^2 \not\in \La^2((0,1))$: Für alle $x \in (0,1)$ ist $|f(x)^2| = f(x)^2 = 1/x$. Für die Funktionenfolge
\[
 h_n := \chi_{[1/n,1)} f^2
\]
gilt auf $(0,1)$ überall $h_n \leq h_{n+1}$ für alle $n \in \N$ und $\limes{n}{\infty} h_n = f^2$. Nach dem Satz über monotone Konvergenz ist also
\[
 \int_{(0,1)} |f|^2 \dlambda
 = \limes{n}{\infty} \int_{(0,1)} h_n \dlambda
 = \limes{n}{\infty} \int_{[1/n,1)} h_n \dlambda
 = \limes{n}{\infty} \int_{1/n}^{1} \frac{1}{x} \dx
 = \infty.
\]


\subsection{}
Wir betrachten die Funktion $f : \R \to \R, x \mapsto 1/(x+1)$. Es ist $f \not\in \La(\R)$: Für die Funktionenfolge
\[
 f_n := \chi_{[-n,n]} f
\]
ist auf $\R$ überall $f_n \leq f_{n+1}$ für alle $n \in \N$ und $\limes{n}{\infty} f_n = f = |f|$. Daher ist nach dem Satz über monotone Konvergenz
\begin{align*}
 \int_\R |f| \dlambda
 &= \limes{n}{\infty} \int_\R f_n \dlambda
 = \limes{n}{\infty} \int_{[-n,n]} f_n \dlambda \\
 &= \limes{n}{\infty} \int_{-n}^n \frac{1}{x+1} \dx
 = 2 \limes{n}{\infty} \int_1^{n+1} \frac{1}{x} \dx \\
 &= 2 \int_1^\infty \frac{1}{x} \dx
 = \infty.
\end{align*}

Es ist jedoch $f \in \La^2(\R)$: Für die Funktionenfolge
\[
 h_n := \chi_{[-n,n]} f^2
\]
ist auf $\R$ überall $h_n \leq h_{n+1}$ für alle $n \in \N$ und $\limes{n}{\infty} h_n = f^2 = |f|^2$. Also ist nach dem Satz über monotone Konvergenz
\begin{align*}
 \int_\R f^2 \dlambda
 &= \limes{n}{\infty} \int_\R h_n \dlambda
 = \limes{n}{\infty} \int_{[-n,n]} f^2 \dlambda \\
 &= \limes{n}{\infty} \int_{-n}^n \frac{1}{(x+1)^2} \dx
 = 2 \limes{n}{\infty} \int_1^{n+1} \frac{1}{x^2} \dx \\
 &= 2 \int_1^\infty \frac{1}{x^2}
 < \infty.
\end{align*}





\section{(Noch mehr Konvergenz)}


\subsection{}
Angenommen, es ist $f_n \longrightarrow f$ im Maß. Dann ist offenbar auch $f_{n_j} \longrightarrow f$ im Maß für jede Teilfolge $n_j$. Daher gibt es für jede Teilfolge $n_j$ eine Teilfolge $j_k$ so dass $f_{n_{j_k}} \longrightarrow f$ punktweise für $\mu$-fast alle $x \in \Omega$.


Die zu zeigende Umkehrung gilt im Allgemeinen nicht: Man betrachte etwa den Maßraum $(\R, \mc{B}(\R),\lambda)$ und die Funktionenfolge $f_n = \chi_{[n,\infty)}$ auf $\R$. Es ist $f_n \longrightarrow 0$ punktweise, also ist auch $f_{n_{j_k}} \longrightarrow 0$ punktweise für alle Teilfolgeen $n_j$ und $j_k$. Es ist jedoch nicht $f_n \longrightarrow f$ im Maß. Die Implikation gilt jedoch für endliche Maßräume, weshalb wir die Implikation unter der zusätzlichen Annahme $\mu(\Omega) < \infty$ zeigen.

Angenommen, es ist nicht $f_n \longrightarrow f$ im Maß. Dann gibt es ein $\delta > 0$, so dass
\[
 \mu(\{ x \in \Omega : |f_n(x)-f(x)| \geq \delta \}) \nlongrightarrow 0 \text{ für } n \longrightarrow \infty.
\]
Also gibt es ein $\varepsilon > 0$ und Teilfolge $n_j$ so dass
\begin{equation}\label{eq: größer epsilon}
 \mu(\{x \in \Omega : |f_{n_j}(x) - f(x)| \geq \delta\}) \geq \varepsilon \text{ für alle } j \in \N.
\end{equation}
Es gibt daher keine Teilfolge $j_k$ mit $f_{n_{j_k}} \longrightarrow f$ punktweise fast überall: Sonst gebe es wegen $\mu(\Omega) < \infty$ nämlich eine Teilfolge $k_l$ mit $f_{n_{j_{k_l}}} \longrightarrow f$ im Maß, also insbesondere
\[
 \limes{l}{\infty} \mu\left(\left\{x \in \Omega: \left|f_{n_{j_{k_l}}} - f(x)\right| \geq \delta \right\}\right) = 0,
\]
was im Widerspruch zu \eqref{eq: größer epsilon} steht.


\subsection{}
Es sei
\[
 A := \{x \in \Omega : f_n(x) \nlongrightarrow f(x) \text{ für } n \to \infty\}
\]
und für alle $\varepsilon > 0$ und $k \in \N$ sei
\begin{align*}
 A_{\varepsilon, k} := \{ x \in \Omega : |f_k(x)-f(x)| \geq \varepsilon \}.
\end{align*}
Offenbar ist $A_{\varepsilon, k}\in \A$ für alle $k \in \N$ und $\varepsilon > 0$. Daher ist auch
\begin{align*}
 A
 &= \{x \in \Omega: \text{ es gibt } \varepsilon > 0 \text{ mit } |f_n(x) - f(x)| \geq \varepsilon \text{ für unendlich viele } n \in \N\} \\
 &= \bigcup_{n \geq 1} \bigcap_{m \in \N} \bigcup_{k \geq m} A_{1/n,k} \in \A.
\end{align*}
Für alle $\varepsilon > 0$ und $k \in \N$ ist nach der Tschebyschow-Ungleichung
\[
 \mu(A_{\varepsilon, k})
 = \mu(\{x \in \Omega : |f_k(x)-f(x)| \geq \varepsilon\})
 \leq \frac{1}{\varepsilon} \int_\Omega |f_k-f| \dmu.
\]
Daher ist für alle $\varepsilon > 0$ und $m \in \N$
\begin{equation}\label{eq: Tschebyschow}
 \mu\left( \bigcup_{k \geq m} A_{\varepsilon, k}\right)
 \leq \sum_{k \geq m} \mu(A_{\varepsilon, k})
 \leq \frac{1}{\varepsilon} \sum_{k \geq m} \int_\Omega |f_k-f| \dmu.
\end{equation}
Da $\sum_{n \in \N} \int_\Omega |f_n-f| \dmu < \infty$ ist
\[
 \limes{m}{\infty} \sum_{k \geq m} \int_\Omega |f_k-f| \dmu = 0.
\]
Zusammen mit \eqref{eq: Tschebyschow} folgt damit, dass für alle $\varepsilon > 0$
\[
 \mu\left( \bigcap_{m=0}^\infty \bigcup_{k \geq m} A_{\varepsilon,k} \right)
 = \limes{m}{\infty} \mu\left( \bigcup_{k \geq m} A_{\varepsilon,k} \right)
 = 0,
\]
denn es ist klar, dass es sich bei $(\bigcup_{k \geq m} A_{\varepsilon, k})_{m \in \N}$ um eine fallende Folge handelt, und es ist $\mu(\bigcup_{k \geq m} A_{\varepsilon, k}) < \infty$ für $m$ groß genug. Also ist
\[
 \mu(A)
 = \mu\left( \bigcup_{n \geq 1} \bigcap_{m=0}^\infty \bigcup_{k \geq m} A_{1/n,k} \right)
 \leq \sum_{n \geq 1} \mu\left( \bigcap_{m=0}^\infty \bigcup_{k \geq m} A_{1/n,k} \right)
 = \sum_{n \geq 1} 0
 = 0.
\]
Daher ist $\mu(A) = 0$. Da nach Definition $f_n(x) \longrightarrow f(x)$ für alle $x \in A^c$ konvergiert also $f_n \rightarrow f$ punktweise $\mu$-fast überall.





\section{($\|f\|_\infty$ als Grenzwert von $\|f\|_p$)}
Angenommen, es ist $f \in \La^\infty(\Omega,\A,\mu)$. Dann ist für alle $p \in [1,\infty)$
\[
 \int_\Omega |f|^p \dmu
 \leq \int_\Omega \|f\|_\infty^p = \mu(\Omega) \|f\|_\infty^p < \infty,
\]
da $|f(x)| \leq \|f\|_\infty$ für $\mu$-fast alle $x \in \Omega$, und damit $f \in \La^p(\Omega,\A,\mu)$. Da für alle $p \in [1,\infty)$
\begin{equation}\label{eq: p leq inf}
 \|f\|_p
 = \left( \int_\Omega |f|^p \dmu \right)^{1/p}
 \leq \mu(\Omega)^{1/p} \|f\|_{\infty}
 \leq \max\{1,\mu(\Omega)\} \|f\|_{\infty}
\end{equation}
ist
\[
 \sup_{p \in [1,\infty)} \|f\|_p \leq \max\{1,\mu(\Omega)\} \|f\|_{\infty} < \infty.
\]

Angenommen, es ist $f \in \La^p(\Omega,\A,\mu)$ für alle $p \in [1,\infty)$ und $\sup_{p \in [1,\infty)} \|f\|_p < \infty$. Für $M := \sup_{p \in [1,\infty)} \|f\|_p$ ist wegen der Tschebyschow-Ungleichung für alle $p \in [1,\infty)$
\begin{align*}
 &\, \mu(\{ x \in \Omega : |f(x)| \geq M+1 \})
 = \mu(\{ x \in \Omega : |f(x)|^p \geq (M+1)^p \}) \\
 \leq&\, \frac{1}{(M+1)^p} \int_\Omega |f|^p \dmu
 = \left(\frac{\|f\|_p}{M+1}\right)^p
 \leq \left(\frac{M}{M+1}\right)^p.
\end{align*}
Da $0 \leq M/(M+1) < 1$ folgt damit, dass
\[
 \mu(\{ x \in \Omega : |f(x)| \geq M+1 \})
 \leq \limes{p}{\infty} \left(\frac{M}{M+1}\right)^p
 = 0,
\]
also
\[
 \mu(\{ x \in \Omega : |f(x)| \geq M + 1 \}) = 0.
\]
Es ist daher $f \in \La^\infty(\Omega,\A,\mu)$.

Wir zeigen nun, dass unter den obigen Bedingungen
\[
 \limes{p}{\infty} \|f\|_p = \|f\|_\infty.
\]
Ist $f(x) = 0$ für $\mu$-fast alle $x \in \Omega$, also $\|f\|_p = 0$ für ein $p \in [1,\infty]$, und damit auch $\|f\|_p = 0$ für alle $p \in [1,\infty]$, so ist nichts weiter zu zeigen. Es wird daher im Folgenden davon ausgegangen, dass nicht $f(x) = 0$ für $\mu$-fast alle $x \in \Omega$. Insbesondere ist dann auch $\mu(\Omega) > 0$.

Aus diesen Annahmen folgt, dass $\|f\|_\infty > 0$. Sei $\|f\|_\infty > \varepsilon > 0$ beliebig aber fest, und
\[
 A := \{x \in \Omega : |f(x)| \geq \|f\|_\infty-\varepsilon\}.
\]
Nach Definition von $\|f\|_\infty$ und $\varepsilon$ ist $\mu(A) > 0$. Für alle $p \in [1,\infty)$ ist
\begin{align*}
 \|f\|_p
 &= \left( \int_\Omega |f|^p \dmu \right)^{1/p}
 \geq \left( \int_A |f|^p \dmu \right)^{1/p} \\
 &\geq \left( \int_A (\|f\|_\infty-\varepsilon)^p \dmu \right)^{1/p}
 = \mu(A)^{1/p} (\|f\|_\infty - \varepsilon),
\end{align*}
und deshalb
\[
 \liminf_{p \to \infty} \|f\|_p
 \geq \liminf_{p \to \infty} \mu(A)^{1/p} (\|f\|_\infty - \varepsilon)
 = \|f\|_\infty - \varepsilon.
\]
Aus der Beliebigkeit von $0 < \varepsilon < \|f\|_\infty$ folgt damit, dass
\[
 \liminf_{p \to \infty} \|f\|_p \geq \|f\|_\infty.
\]
Andererseits folgt aus \eqref{eq: p leq inf}, dass
\[
 \limsup_{p \to \infty} \|f\|_p \leq \limsup_{p \to \infty} \mu(\Omega)^{1/p} \|f\|_\infty = \|f\|_\infty.
\]
Es ist also
\[
 \|f\|_\infty \leq \liminf_{p \to \infty} \|f\|_p \leq \limsup_{p \to \infty} \|f\|_p \leq \|f\|_\infty,
\]
und deshalb
\[
 \limes{p}{\infty} \|f\|_p = \|f\|_\infty.
\]





\section{}

\begin{lem}\label{lem: Lp trick}
 Sei $(\Omega, \A, \mu)$ ein Maßraum und $f \in \La^p(\Omega, \A, \mu)$ für $p \in [1,\infty]$. Für alle $\alpha \in [1,p]$, mit $\alpha \neq \infty$ falls $p = \infty$, ist $|f|^\alpha \in \La^{p/\alpha}(\Omega, \A, \mu)$ mit $\| |f|^\alpha \|_{p/\alpha} = \|f\|_p^\alpha$.
\end{lem}
\begin{proof}
 Ist $p = \infty$, so ist für alle $M \geq 0$
 \[
  \{x \in \Omega : |f(x)| \leq M\}
  = \{x \in \Omega : |f(x)|^\alpha \leq M^\alpha\}.
 \]
 Daher ist auch $|f|^\alpha \in \La^\infty(\Omega, \A, \mu)$ und $\||f|^\alpha\|_\infty = \|f\|_\infty^\alpha$. Ist $p < \infty$, so ist
 \[
  \int_\Omega \left( |f|^\alpha \right)^{p/\alpha} \dmu
  = \int_\Omega |f|^\alpha \dmu
  < \infty,
 \]
 also $|f|^\alpha \in \La^{p/\alpha}(\Omega, \A, \mu)$, und
 \[
  \| |f|^\alpha \|_{p/\alpha}
  = \left( \int_\Omega \left( |f|^\alpha \right)^{p/\alpha} \dmu \right)^{\alpha/p}
  = \left( \int_\Omega |f|^p \dmu \right)^{\alpha/p}
  = \|f\|_p^\alpha.
 \]
\end{proof}




\subsection{}
Für $p=q$ ist nichts zu zeigen, weshalb im Folgenden nur der Fall $p < q$ betrachtet wird. Ist $q = \infty$, so ergeben sich die Aussagen direkt aus \textbf{Aufgabe 3}, und aus \eqref{eq: p leq inf} insbesondere die Ungleichung $\|f\|_p \leq C\|f\|_q$ mit $C = \max \{\mu(\Omega), 1\}$. Daher wird im Folgenden nur der Fall $p < q < \infty$ betrachtet.

Sei $f \in \La^q(\Omega, \A, \mu)$. Nach Lemma \ref{lem: Lp trick} ist $|f|^p \in \La^{q/p}(\Omega, \A, \mu)$ mit $\| |f|^p \|_{q/p} = \|f\|_q^p$. Da $\mu(\Omega) < \infty$ ist $1 \in \La^{1/(1-p/q)}(\Omega, \A, \mu)$ mit
\[
 \|1\|_{1/(1-p/q)}
 = \left( \int_\Omega 1^{1/(1-p/q)} \dmu \right)^{1-p/q}
 = \left( \int_\Omega 1 \dmu \right)^{1-p/q}
 = \mu(\Omega)^{1-p/q}
\]
Nach der Hölder-Ungleichung ist daher
\[
 \int_\Omega |f|^p \dmu
 = \int_\Omega 1 \cdot |f|^p
 \leq \|1\|_{1/(1-p/q)} \big\| |f|^p \big\|_{q/p}
 = \mu(\Omega)^{1-p/q} \|f\|_q^p.
\]
Also ist $f \in \La^p(\Omega, \A, \mu)$ mit
\[
 \|f\|_p
 \leq \mu(\Omega)^{1/p-1/q} \|f\|_q.
\]


\subsection{}
Sei $f \in \La^1(\Omega, \A, \mu) \cap \La^\infty(\Omega, \A, \mu)$ und $p \in [1,\infty)$ beliebig aber fest. Für den Fall $p = 1$ ist nichts zu zeigen, weshalb im Folgenden nur der Fall $1 < p$ betrachtet wird.
Da $f \in \La^\infty(\Omega, \A, \mu)$ ist nach Lemma \ref{lem: Lp trick} auch $|f|^{p-1} \in \La^\infty(\Omega, \A, \mu)$ mit $\|f^{p-1}\|_\infty = \|f\|_\infty^{p-1}$. Da  $f \in \La^1(\Omega, \A, \mu)$ ist daher nach der Hölder-Ungleichung
\[
 \int_\Omega |f|^p \dmu
 = \int_\Omega |f| \cdot |f|^{p-1} \dmu
 \leq \|f\|_1 \|f^{p-1}\|_\infty
 = \|f\|_1 \|f\|_\infty^{p-1}.
\]
Es ist daher $f \in \La^p(\Omega, \A, \mu)$ und
\[
 \|f\|_p
 = \left( \int_\Omega |f|^p \dmu \right)^{1/p}
 \leq \|f\|_1^{1/p} \|f\|_\infty^{(p-1)/p}.
\]


\subsection{}
Ist $p = q$, so ist zwangsweise auch $r = p = q$ und die Aussagen gelten offenbar; auch wenn $p < q$ und $r = p$ oder $r = q$ sind sie offensichtlich. Es wird daher im Folgenden nur der Fall $p < r < q$ betrachtet.

Wir bemerken direkt, dass es in diesem Fall ein eindeutiges $\theta \in [0,1]$ mit $1/r = (1-\theta)/p + \theta/q$ gibt, wobei $\theta \in (0,1)$, da die Abbildung $[0,1] \rightarrow [1/q, 1/p], \theta \mapsto (1-\theta)/p + \theta/q$ eine Bijektion ist.

Da $f \in \La^p(\Omega, \A, \mu)$ und $f \in \La^q(\Omega,\A,\mu)$ ist nach Lemma \ref{lem: Lp trick} sowohl
\[
 |f|^{r(1-\theta)} \in \La^{p/(r(1-\theta))}(\Omega, \A, \mu) \text{ mit }
 \||f|^{r(1-\theta)}\|_{p/(r(1-\theta))} = \|f\|_p^{r(1-\theta)}
\]
als auch
\[
 |f|^{r\theta} \in \La^{q/(r\theta)}(\Omega, \A, \mu) \text{ mit }
 \|f^{r\theta}\|_{q/(r\theta)} = \|f\|_q^{r\theta}.
\]
Da
\[
 \frac{r(1-\theta)}{p} + \frac{r\theta}{q} = r\left(\frac{1-\theta}{p} + \frac{\theta}{q}\right) = 1
\]
ist nach der Hölderungleichung
\begin{align*}
 \int_\Omega |f|^r \dmu
 = \int_\Omega |f|^{r(1-\theta)} |f|^{r\theta} \dmu
 &\leq \||f|^{r(1-\theta)}\|_{p/(r(1-\theta))} \|f^{r\theta}\|_{q/(r\theta)} \\
 &= \|f\|_p^{r(1-\theta)} \|f\|_q^{r\theta}.
\end{align*}
Es ist also $f \in \La^r(\Omega, \A, \mu)$ mit
\[
 \|f\|_r
 = \left( \int_\Omega |f|^r \right)^{1/r}
 \leq \|f\|_p^{1-\theta} \|f\|_q^\theta.
\]


































\end{document}
