\documentclass[a4paper,10pt]{article}
%\documentclass[a4paper,10pt]{scrartcl}

\usepackage{xltxtra}
\usepackage[ngerman]{babel}
\usepackage{amsmath}
\usepackage{amssymb}
\usepackage{amsthm}
\usepackage{mathtools}
\usepackage{nicefrac}
\usepackage{enumerate}
\usepackage{leftidx}

\makeatletter
\g@addto@macro\th@definition{\thm@headpunct{:}}
\makeatother
\theoremstyle{definition}
\newtheorem{lem}{Lemma}
\newtheorem{beh}{Behauptung}
\newtheorem{bem}{Bemerkung}
\newtheorem*{ia}{Induktionsanfang}
\newtheorem*{is}{Induktionsschritt}

\renewcommand{\thesection}{Aufgabe \arabic{section}.}
\renewcommand{\thesubsection}{\alph{subsection})}
\renewcommand{\thesubsubsection}{(\roman{subsubsection})}

\makeatletter
\def\moverlay{\mathpalette\mov@rlay}
\def\mov@rlay#1#2{\leavevmode\vtop{%
   \baselineskip\z@skip \lineskiplimit-\maxdimen
   \ialign{\hfil$\m@th#1##$\hfil\cr#2\crcr}}}
\newcommand{\charfusion}[3][\mathord]{
    #1{\ifx#1\mathop\vphantom{#2}\fi
        \mathpalette\mov@rlay{#2\cr#3}
      }
    \ifx#1\mathop\expandafter\displaylimits\fi}
\makeatother

\newcommand{\N}{\mathbb{N}}
\newcommand{\Z}{\mathbb{Z}}
\newcommand{\Q}{\mathbb{Q}}
\newcommand{\R}{\mathbb{R}}
\newcommand{\C}{\mathbb{C}}
\newcommand{\A}{\mathcal{A}}
\newcommand{\La}{\mathcal{L}}
\newcommand{\dx}{\,\text{d}x}
\newcommand{\dy}{\,\text{d}y}
\newcommand{\dt}{\,\text{d}t}
\newcommand{\du}{\,\text{d}u}
\newcommand{\Img}{\operatorname{Im}}
\newcommand{\Real}{\operatorname{Re}}
\newcommand{\Imag}{\operatorname{Im}}
\newcommand{\sgn}{\operatorname{sgn}}
\newcommand{\Vol}{\operatorname{Vol}}
\newcommand{\dotcup}{\ensuremath{\mathaccent\cdot\cup}}
\newcommand{\bigdotcup}{\charfusion[\mathop]{\bigcup}{\cdot}}
\newcommand{\ceil}[1]{\left\lceil{#1}\right\rceil}
\newcommand{\floor}[1]{\left\lfloor{#1}\right\rfloor}
\newcommand{\mc}[1]{\mathcal{#1}}
\newcommand{\limes}[2]{\lim_{#1 \rightarrow #2}}
\newcommand{\limessup}[1]{\limsup_{#1 \rightarrow \infty}}
\newcommand{\limesinf}[1]{\liminf_{#1 \rightarrow \infty}}
\newcommand{\vect}[1]{\begin{pmatrix}#1\end{pmatrix}}
\newcommand{\partd}[2]{\frac{\partial #1}{\partial #2}}
\newcommand{\op}[1]{\left\|#1\right\|_{\text{op}}}

\makeatletter
\renewcommand*\env@matrix[1][*\c@MaxMatrixCols c]{%
  \hskip -\arraycolsep
  \let\@ifnextchar\new@ifnextchar
  \array{#1}}
\makeatother

\setromanfont[Mapping=tex-text]{Linux Libertine O}
% \setsansfont[Mapping=tex-text]{DejaVu Sans}
% \setmonofont[Mapping=tex-text]{DejaVu Sans Mono}
\parindent 0pt

\title{\sc Analysis III \\ \Large 5. Aufgabenblatt}
\author{Jendrik Stelzner}
\date{\today}

\begin{document}
\maketitle





\section{(Transformationsregel des Lebesguemaßes)}


\subsection{}
Sei $N \subseteq \R^n$ eine Lebesgue-Nullmenge und $\varepsilon > 0$ beliebig aber fest.
Aus der Definition des Lebesgue-Maßes folgt, dass es eine Folge $(I_k)_{k \in \N}$ von $n$-dimensionalen offenen Intervallen gibt, so dass $N \subseteq \bigcup_{k \in \N} I_k$ und $\lambda_n(N) \leq \sum_{k \in \N} \Vol(I_k)$. Dabei ist gerade $\Vol(I_k) = \lambda_n(I_k)$ für alle $k$.

Da jedes der $I_k$ offen ist, gibt es, wie aus der Vorlesung bekannt, für alle $k \in \N$ eine disjunkte Familie halboffener Würfel $(C^k_l)_{l \in \N}$ mit $I_k = \bigcup_{l \in \N} C^k_l$ für alle $k$. (Im Skript ist dies \textbf{Lemma 1.32}; auch wenn dort im Satz nur von halboffenen Quadern gesprochen wird, handelt es sich bei den genutzten Quadern offenbar um Würfel.)

Es ist also $(C^k_l)_{k,l \in \N}$ eine abzählbare Überdeckung von $N$ mit paarweise disjunkten, halboffenen Würfeln, wobei aufgrund der $\sigma$-Additivität von $\lambda_n$
\[
 \sum_{k,l \in \N} \Vol(C^k_l)
 = \sum_{k \in \N} \sum_{l \in \N} \lambda_n(C^k_l)
 = \sum_{k \in \N} \lambda_n \left(\bigcup_{l \in \N} C^k_l\right)
 = \sum_{k \in \N} \lambda_n(I_n)
 < \varepsilon.
\]
Das Umsortieren der Summanden ist deshalb möglich, da diese alle nicht-negativ sind.


\subsection{}
Ist $\alpha_i \geq 0$ für alle $i$, so ist offenbar
\begin{equation}\label{eq: skalierter würfel}
 T(Q) = T\left( [0,l)^n \right) = \prod_{i=1}^n [0,\alpha_i l),
\end{equation}
ein mehrdimensionales Intervall, also 
\begin{equation}\label{eq: volumenänderung würfel}
 \lambda_n(T(Q))
 = \prod_{i=1}^n |\alpha_i| l
 = \left(\prod_{i=1}^n |\alpha_i| \right) l^n
 = \left(\prod_{i=1}^n |\alpha_i| \right) \lambda_n(Q).
\end{equation}
Ist $\alpha_i < 0$ für ein $i$, so muss man in \eqref{eq: skalierter würfel} das Intervall $[0,\alpha_i l)$ durch $(\alpha_i l, 0]$ ersetzen. Die Aussage \eqref{eq: volumenänderung würfel} bleibt jedoch unverändert, da das Volumen eines Quaders, da es sich um ein mehrdimensionales Intervall handelt, nur von dessen Seitenlängen abhängt.

Es zu bemerken, dass aufgrund der Translationsinvarianz des Lebesgue-Maßes diese Gleichung \eqref{eq: volumenänderung würfel} für alle halboffenen Würfel, unabhängig von ihrer konkreten Position im $\R^n$ gilt.

Sei nun $N \subseteq \R^n$ eine Lebesgue-Nullmenge und $\varepsilon > 0$ beliebig aber fest. Wie im vorherigen Aufgabenteil gezeigt, gibt es eine Familie $(Q_k)_{k \in \N}$ von halboffenen Würfeln mit $N \subseteq \bigcup_{k \in \N} Q_k$ und $\sum_{k \in \N} \lambda_n(Q_k) < \varepsilon$. Es ist also
\begin{align*}
 \lambda_n( T(N) )
 &= \lambda_n\left( T\left( \bigcup_{k \in \N} Q_k \right) \right)
 = \sum_{k \in \N} \lambda_n(T(Q_k)) \\
 &= \sum_{k \in \N} \left(\prod_{i=1}^n |\alpha_i| \right) \lambda_n(Q_k)
 = \left(\prod_{i=1}^n |\alpha_i| \right) \sum_{k \in \N} \lambda_n(Q_k) \\
 &< \left(\prod_{i=1}^n |\alpha_i| \right) \varepsilon.
\end{align*}
Aus der Endlichkeit des Produkts $\prod_{i=1}^n |\alpha_i|$ und der Beliebigkeit von $\varepsilon > 0$ folgt damit, dass $\lambda_n(T(N)) = 0$.










\section{(Vervollständigung von Maßen)}


\subsection{}
Man bemerke, dass $\R^n \setminus \{0\} \in \mc{B}(\R^n)$ bezüglich $\mu$ eine Nullmenge ist. Da es für jedes $A \subseteq \R^n$ ein $B \subseteq \R^n \setminus \{0\}$ mit $A = \emptyset \cup B$ oder $A = \{0\} \cup \R^n \setminus\{0\}$ gibt, ist $\A_0 = \mc{P}(\R^n)$, denn $\emptyset, \{0\} \in \mc{B}(\R^n)$.


\subsection{}
Es sei
\begin{align*}
 \tilde{\A}_0
 :=& \bigcup_{A \subseteq \R^2} \left\{A \cup (0 \times \R), A \setminus (0 \times \R)\right\}.
\end{align*}
$\tilde{\A}_0$ enthält genau die Teilmengen von $\R^2$, die $0 \times \R$ ganz oder gar nicht beinhalten. Es gilt $\tilde{\A}_0 = \A$.

Sei $A \in \tilde{\A}_0$. Ist $(0,0) \not\in A$, so ist $A = \emptyset \cup A$, wobei $\emptyset \in \A$ und $A$ in der Nullmenge $\R^2 \setminus (0 \times \R) = (\R \setminus 0) \times \R \in \A$ enthalten ist, also $A \in \A_0$. Ist $(0,0) \in A$, so ist  $A = (0 \times \R) \cup (A \setminus (0 \times \R))$ analog in $\A_0$ enthalten. Also ist $\tilde{\A}_0 \subseteq \A_0$. 

Sei $A \in \A_0$. Es gibt es $B \in \A$ und $N \subseteq \mc{P}(\R^2)$ mit $N \subseteq M$ für eine Nullmenge $M \in \A$ sodass $A = B \cup N$. Da $M$ ein Nullmenge ist, ist $(0,0) \not\in M$, also $(0 \times \R) \cap M = \emptyset$. Daher ist $(0,0) \in A$ genau dann wenn $(0,0) \in B$. Dies gilt genau dann, wenn $(0 \times \R) \subseteq B$, und genau dann nicht, wenn $(0 \times \R) \cap B = \emptyset$. Es ist also $(0 \times \R)$ entweder ganz oder gar nicht in $A$, und daher $A \in \tilde{\A}_0$. Es ist also $\A_0 \subseteq \tilde{\A}_0$.





\section{(Fast stetige Funktionen)}


\subsection{}
Sei $t \in \R$ beliebig aber fest. Da $N$ abzählbar ist, ist $N \in \mc{B}(\R^n)$. (Jede abzählbare Menge $A$ lässt sich als \[ A = \bigcup_{a \in A} [a-1,a]\cap[a,a+1] \] darstellen.) Es ist also auch $M := \R^n \setminus N \in \mc{B}(\R^n)$. Nach Definition von $N$ ist $f_{|M} : M \rightarrow \R^n$ stetig. Es ist daher
\[
 A := \{x \in M : f_{|M}(x) \leq t\} \in \mc{B}(\R^n),
\]
da stetige Funktionen immer messbar sind. Da $N$ abzählbar ist, ist es auch
\[
 N' := \{x \in N : f(x) \leq t\},
\]
also $N' \in \mc{B}(\R^n)$. Daher ist auch
\[
 \{x \in N : f(x) \leq t\} = A \cup N' \in \mc{B}(\R^n),
\]
also $f$ Borell-messbar.


\subsection{}
Sei $t \in \R$ beliebig aber fest, und $A := \{x \in \R: f(x) < t\}$. Ist $A = \R$, so ist $A$ Borell-messbar. Ansonsten ist $A$ ein Intervall der Form $(a,\infty)$ oder $[a,\infty)$ mit $a \in \R$. Auch in diesen Fällen ist $A$ Borell-messbar.


\subsection{}
Sei $t \in \R$ beliebig aber fest.
Da $\lambda(N) = 0$ und das Lebesgue-Maß auf $(\R^n, \mc{M}_\lambda)$ vollständig ist, ist $N \in \mc{M}_\lambda$. Da $\R^n \in \mc{M}_\lambda$ ist daher auch $M = \R^n \setminus N \in \mc{M}_\lambda$. Nach Definition ist $f_{|M}$ stetig, also
\[
 A := \{x \in M : f_{|M}(x) \leq t\} \in \mc{M}_\lambda,
\]
Borell- und damit auch Lebesgue-messbar ist. Für
\[
 N' := \{x \in N : f(x) \leq t\}
\]
ist $N' \subseteq N$, wegen der Vollständigkeit des Lebesgue-Maßes also auch $\lambda(N') \in \mc{M}_\lambda$. Damit ist
\[
 \{x \in \R^n : f(x) \leq t\} = A \cup N' \in \mc{M}_\lambda,
\]
also $f$ Lebesgue-messbar.


\subsection{}
Es genügt zu zeigen, dass $\overline{A_{2 \varepsilon}} \subseteq A_\varepsilon$ für alle $\varepsilon > 0$. Denn dann ist
\[
 N
 = \bigcup_{\substack{q \in \Q \\ q > 0}} A_q
 \supseteq \bigcup_{\substack{q \in \Q \\ q > 0}} \overline{A_{2q}}
 = \bigcup_{\substack{q \in \Q \\ q > 0}} \overline{A_q}
 \supseteq \bigcup_{\substack{q \in \Q \\ q > 0}} A_q,
 = N
\]
also
\[
 N = \bigcup_{\substack{q \in \Q \\ q > 0}} \overline{A_q} \in F_{\sigma}.
\]

Sei $\varepsilon > 0$ beliebig aber fest. Es ist klar, dass $A_{2 \varepsilon} \subseteq A_\varepsilon$. O.b.d.A sei $a \in \partial A_{2 \varepsilon}$ und $\delta := \limsup_{y \rightarrow a} |f(y)-f(a)|$ (gibt es kein solches $a$, so ist nichts mehr zu zeigen). Angenommen, es ist $\delta < \varepsilon$. Dann gibt es ein $\omega > 0$, so dass $|f(y)-f(a)| < \delta$ für alle $y \in \R^n$ mit $|a-y| < \omega$. Da $a \in \partial A_{2 \varepsilon}$ gibt es ein $x \in A_{2 \varepsilon}$ mit $|x-a| < \omega$. Es ist
\[
 \limsup_{y \rightarrow x} |f(y)-f(x)|
 \leq \limsup_{y \rightarrow x} |f(y)-f(a)| + \limsup_{y \rightarrow x} |f(a)-f(x)|
 < 2 \delta
 < 2 \varepsilon,
\]
im Widerspuch zu $x \in A_{2 \varepsilon}$. Also muss $\delta \geq \varepsilon$, und somit $a \in A_{\varepsilon}$.





\section{(Reguläre Maße)}


\subsection{}
Sei $A \in B(\R^n)$ und $\varepsilon > 0$ beliebig aber fest. 

\subsubsection*{(ii) $\Rightarrow$ (iv)}
Es gibt eine offene Menge $U \subseteq \R^n$ mit $A^c \subseteq U$ und
\[
 \mu(U) \leq \mu(A^c) + \varepsilon.
\]
Es ergibt sich wegen der Endlichkeit von $\mu$, dass
\[
 \mu(U^c) = \mu(\R^n) - \mu(U) \geq \mu(\R^n) - \mu(A^c) - \varepsilon = \mu(A) - \varepsilon.
\]
Da $U^c \subseteq A$ abgeschlossen ist, und $\varepsilon > 0$ beliebig, zeigt dies (iv).

\subsection*{(iv) $\Rightarrow$ (ii)}
Sei $C \subseteq A^c$ abgeschlossen mit
\[
 \mu(C) \geq \mu(A^c) - \varepsilon.
\]
Es ist dann wegen der Endlichkeit von $\mu$
\[
 \mu(C^c) = \mu(\R^n) - \mu(C) \leq \mu(\R^n) - \mu(A^c) + \varepsilon = \mu(A) + \varepsilon.
\]
Da $C^c \supseteq A$ offen ist, und $\varepsilon > 0$, zeigt dies (ii).

\subsubsection*{(iii) $\Rightarrow$ (iv)}
Da jede kompakte Menge $K \subseteq \R^n$ auch abgeschlossen ist, gilt
\begin{align*}
 A
 &= \sup \{ \mu(K) : K \subseteq A, K \text{ kompakt} \} \\
 &\leq \sup \{ \mu(K) : C \subseteq A, C \text{ abgeschlossen} \}
 \leq A,
\end{align*}
also
\[
 A = \sup \{ \mu(K) : K \subseteq A, K \text{ abgeschlossen} \}.
\]

\subsubsection*{(iv) $\Rightarrow$ (iii)}
Es gibt ein $C \subseteq A$ abgeschlossen mit $\mu(C) + \frac{\varepsilon}{2} \geq \mu(A)$. Für alle $k \in \N$ sei
\[
 C_k := C \cap \overline{B_0(k)}.
\]
$C_k$ ist für alle $k$ abgeschlossen und beschränkt, also kompakt. Da $(C_k)_{k \in \N}$ eine wachsende Folge auf $\mc{B}(\R^n)$ ist, gilt
\[
 \limes{k}{\infty} \mu(C_k) = \mu\left( \bigcup_{k \in \N} C_k \right) = \mu(C).
\]
Es gibt also ein $N \in \N$ mit $\mu(C_N) + \frac{\varepsilon}{2} \geq \mu(C)$. Da daher
\[
 \mu(C_N) + \varepsilon \geq \mu(C) + \frac{\varepsilon}{2} \geq \mu(A),
\]
und $C_N \subseteq C \subseteq A$ abgeschlossen und beschränkt, also kompakt, ist, folgt (iii) aus der Beliebigkeit von $\varepsilon > 0$.


\subsection{}
Es sei
\[
 \A := \{A \in \mc{B}(\R^n): \text{(ii) und (iv) gelten für $A$}\}.
\]
$\A$ bildet eine $\sigma$-Algebra.

Es ist klar, dass $\R^n$ (ii) und (iv) erfüllt.

Für $A \in \A$ gilt (ii), d.h. es gibt für beliebiges aber festes $\varepsilon > 0$ eine offene Menge $U \supseteq A$ mit $\mu(U) \leq \mu(A)+\varepsilon$. Für $A^c$ gilt dann, dass $U^c \subseteq A^c$ eine abgeschlossene Menge mit
\[
 \mu(U^c) = \mu(\R^n) - \mu(U) \geq \mu(\R^n) - \mu(A) - \varepsilon = \mu(A^c) - \varepsilon.
\]
Wegen der Beliebigkeit von $\varepsilon > 0$ zeigt dies, dass (iv) für $A^c$ gilt. Da (ii) und (iv) äquivalent sind, ist daher auch $A^c \in \A$.

Sei $(A_k)_{k \in \N}$ eine Folge von Mengen mit $A_k \in \A$ für alle $k$ und $\varepsilon > 0$ beliebig aber fest. Es gibt für alle $k$ eine abgeschlossene Menge $C_k \subseteq \R^n$ mit $C_k \subseteq A_k$ und
\[
 \mu(A_k \setminus C_k) = \mu(A_k) - \mu(C_k) \leq \frac{\varepsilon}{2^{k+1}}
\]
Es ist daher
\[
 \bigcup_{k \in \N} C_k =: C \subseteq A := \bigcup_{k \in \N} A_k.
\]
eine abgeschlossene Teilmenge mit
\[
 A \setminus C
 = \bigcup_{k \in \N} A_k \setminus \bigcup_{k \in \N} C_k
 \subseteq \bigcup_{k \in \N} (A_k \setminus C_k).
\]
Es gilt
\begin{align*}
 \mu(A) - \mu(C)
 &= \mu(A \setminus C)
 \leq \mu\left( \bigcup_{k \in \N} A_k \setminus C_k \right) \\
 &\leq \sum_{k \in \N} \mu(A_k \setminus C_k)
 \leq \sum_{k \in \N} \frac{\varepsilon}{2^{k+1}}
 = \varepsilon,
\end{align*}
also $\mu(A) \leq \mu(C) + \varepsilon$. Wegen der Beliebigkeit von $\varepsilon > 0$ erfüllt $A$ daher (iv), also auch (ii), ist also in $\A$ enthalten.

Trivialerweise erfüllt jede offene Menge Bedingung (ii), und damit auch (iv). Also sind alle offenen Mengen in $\A$ enthalten. Da $\mc{B}(\R^n)$ die kleinste $\sigma$-Algebra mit dieser Eigenschaft ist, muss $\mc{B}(\R^n) \subseteq \A$. Da nach Definition $\A \subseteq \mc{B}(\R^n)$ ist daher $\mc{B}(\R^n) = \A$. Es erfüllen also alle $A \in \mc{B}(\R^n)$ die Bedingungen (ii) und (iv), also auch (iii). Da Bedingung (i) direkt aus der Endlichkeit von $\mu$ folgt, ist $\mu$ daher regulär.
























\end{document}
