\documentclass[a4paper,10pt]{article}
%\documentclass[a4paper,10pt]{scrartcl}

\usepackage{xltxtra}
\usepackage[ngerman]{babel}
\usepackage{amsmath}
\usepackage{amssymb}
\usepackage{amsthm}
\usepackage{mathtools}
\usepackage{nicefrac}
\usepackage{enumerate}
\usepackage{leftidx}

\makeatletter
\g@addto@macro\th@definition{\thm@headpunct{:}}
\makeatother
\theoremstyle{definition}
\newtheorem{lem}{Lemma}
\newtheorem{beh}{Behauptung}
\newtheorem{bem}{Bemerkung}
\newtheorem*{ia}{Induktionsanfang}
\newtheorem*{is}{Induktionsschritt}

\renewcommand{\thesection}{Aufgabe \arabic{section}.}
\renewcommand{\thesubsection}{\alph{subsection})}
\renewcommand{\thesubsubsection}{(\roman{subsubsection})}

\makeatletter
\def\moverlay{\mathpalette\mov@rlay}
\def\mov@rlay#1#2{\leavevmode\vtop{%
   \baselineskip\z@skip \lineskiplimit-\maxdimen
   \ialign{\hfil$\m@th#1##$\hfil\cr#2\crcr}}}
\newcommand{\charfusion}[3][\mathord]{
    #1{\ifx#1\mathop\vphantom{#2}\fi
        \mathpalette\mov@rlay{#2\cr#3}
      }
    \ifx#1\mathop\expandafter\displaylimits\fi}
\makeatother

\newcommand{\N}{\mathbb{N}}
\newcommand{\Z}{\mathbb{Z}}
\newcommand{\Q}{\mathbb{Q}}
\newcommand{\R}{\mathbb{R}}
\newcommand{\C}{\mathbb{C}}
\newcommand{\A}{\mathcal{A}}
\newcommand{\La}{\mathcal{L}}
\newcommand{\dx}{\,\text{d}x}
\newcommand{\dy}{\,\text{d}y}
\newcommand{\dt}{\,\text{d}t}
\newcommand{\du}{\,\text{d}u}
\newcommand{\Img}{\operatorname{Im}}
\newcommand{\Real}{\operatorname{Re}}
\newcommand{\Imag}{\operatorname{Im}}
\newcommand{\sgn}{\operatorname{sgn}}
\newcommand{\dotcup}{\ensuremath{\mathaccent\cdot\cup}}
\newcommand{\bigdotcup}{\charfusion[\mathop]{\bigcup}{\cdot}}
\newcommand{\ceil}[1]{\left\lceil{#1}\right\rceil}
\newcommand{\floor}[1]{\left\lfloor{#1}\right\rfloor}
\newcommand{\mc}[1]{\mathcal{#1}}
\newcommand{\limes}[2]{\lim_{#1 \rightarrow #2}}
\newcommand{\limessup}[1]{\limsup_{#1 \rightarrow \infty}}
\newcommand{\limesinf}[1]{\liminf_{#1 \rightarrow \infty}}
\newcommand{\vect}[1]{\begin{pmatrix}#1\end{pmatrix}}
\newcommand{\partd}[2]{\frac{\partial #1}{\partial #2}}
\newcommand{\op}[1]{\left\|#1\right\|_{\text{op}}}

\makeatletter
\renewcommand*\env@matrix[1][*\c@MaxMatrixCols c]{%
  \hskip -\arraycolsep
  \let\@ifnextchar\new@ifnextchar
  \array{#1}}
\makeatother

\setromanfont[Mapping=tex-text]{Linux Libertine O}
% \setsansfont[Mapping=tex-text]{DejaVu Sans}
% \setmonofont[Mapping=tex-text]{DejaVu Sans Mono}
\parindent 0pt

\title{\sc Analysis III \\ \Large 5. Aufgabenblatt}
\author{Jendrik Stelzner}
\date{\today}

\begin{document}
\maketitle





\section{(Transformationsregel des Lebesguemaßes)}





\section{(Vervollständigung von Maßen)}


\subsection{}
Man bemerke, dass $\R^n \setminus \{0\} \in \mc{B}(\R^n)$ bezüglich $\mu$ eine Nullmenge ist. Da es für jedes $A \subseteq \R^n$ ein $B \subseteq \R^n \setminus \{0\}$ mit $A = \emptyset \cup B$ oder $A = \{0\} \cup \R^n \setminus\{0\}$ gibt, ist $\mc{A}_0 = \mc{P}(\R^n)$, denn $\emptyset, \{0\} \in \mc{B}(\R^n)$.


\subsection{}
Es sei
\begin{align*}
 \tilde{\mc{A}}_0
 :=& \bigcup_{A \subseteq \R^2} \left\{A \cup (0 \times \R), A \setminus (0 \times \R)\right\}.
\end{align*}
$\tilde{\mc{A}}_0$ enthält genau die Teilmengen von $\R^2$, die $0 \times \R$ ganz oder gar nicht beinhalten. Es gilt $\tilde{\mc{A}}_0 = \mc{A}$.

Sei $A \in \tilde{\mc{A}}_0$. Ist $(0,0) \not\in A$, so ist $A = \emptyset \cup A$, wobei $\emptyset \in \mc{A}$ und $A$ in der Nullmenge $\R^2 \setminus (0 \times \R) = (\R \setminus 0) \times \R \in \mc{A}$ enthalten ist, also $A \in \mc{A}_0$. Ist $(0,0) \in A$, so ist  $A = (0 \times \R) \cup (A \setminus (0 \times \R))$ analog in $\mc{A}_0$ enthalten. Also ist $\tilde{\mc{A}}_0 \subseteq \mc{A}_0$. 

Sei $A \in \mc{A}_0$. Es gibt es $B \in \mc{A}$ und $N \subseteq \mc{P}(\R^2)$ mit $N \subseteq M$ für eine Nullmenge $M \in \mc{A}$ sodass $A = B \cup N$. Da $M$ ein Nullmenge ist, ist $(0,0) \not\in M$, also $(0 \times \R) \cap M = \emptyset$. Daher ist $(0,0) \in A$ genau dann wenn $(0,0) \in B$. Dies gilt genau dann, wenn $(0 \times \R) \subseteq B$, und genau dann nicht, wenn $(0 \times \R) \cap B = \emptyset$. Es ist also $(0 \times \R)$ entweder ganz oder gar nicht in $A$, und daher $A \in \tilde{\mc{A}}_0$. Es ist also $\mc{A}_0 \subseteq \tilde{\mc{A}}_0$.







\section{(Fast stetige Funktionen)}





\section{(Reguläre Maße)}


















\end{document}
