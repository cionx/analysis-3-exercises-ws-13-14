\documentclass[a4paper,10pt]{article}
%\documentclass[a4paper,10pt]{scrartcl}

\usepackage{xltxtra}
\usepackage[ngerman]{babel}
\usepackage{amsmath}
\usepackage{amssymb}
\usepackage{amsthm}
\usepackage{mathtools}
\usepackage{nicefrac}
\usepackage{enumerate}
\usepackage{leftidx}

\makeatletter
\g@addto@macro\th@definition{\thm@headpunct{:}}
\makeatother
\newcounter{saetze}
\newtheorem{lem}[saetze]{Lemma}
\newtheorem{beh}[saetze]{Behauptung}
\newtheorem{bem}[saetze]{Bemerkung}
\newtheorem*{ia}{Induktionsanfang}
\newtheorem*{is}{Induktionsschritt}

\renewcommand{\thesection}{Aufgabe \arabic{section}.}
\renewcommand{\thesubsection}{\alph{subsection})}
\renewcommand{\thesubsubsection}{(\roman{subsubsection})}

\makeatletter
\def\moverlay{\mathpalette\mov@rlay}
\def\mov@rlay#1#2{\leavevmode\vtop{%
   \baselineskip\z@skip \lineskiplimit-\maxdimen
   \ialign{\hfil$\m@th#1##$\hfil\cr#2\crcr}}}
\newcommand{\charfusion}[3][\mathord]{
    #1{\ifx#1\mathop\vphantom{#2}\fi
        \mathpalette\mov@rlay{#2\cr#3}
      }
    \ifx#1\mathop\expandafter\displaylimits\fi}
\makeatother

\newcommand{\N}{\mathbb{N}}
\newcommand{\Z}{\mathbb{Z}}
\newcommand{\Q}{\mathbb{Q}}
\newcommand{\R}{\mathbb{R}}
\newcommand{\C}{\mathbb{C}}
\newcommand{\A}{\mathcal{A}}
\newcommand{\La}{\mathcal{L}}
\newcommand{\dx}{\,\text{d}x}
\newcommand{\dy}{\,\text{d}y}
\newcommand{\dt}{\,\text{d}t}
\newcommand{\du}{\,\text{d}u}
\newcommand{\dmu}{\,\text{d}\mu}
\newcommand{\dlambda}{\,\text{d}\lambda}
\newcommand{\Img}{\operatorname{Im}}
\newcommand{\Real}{\operatorname{Re}}
\newcommand{\Imag}{\operatorname{Im}}
\newcommand{\sgn}{\operatorname{sgn}}
\newcommand{\Vol}{\operatorname{Vol}}
\newcommand{\dotcup}{\ensuremath{\mathaccent\cdot\cup}}
\newcommand{\bigdotcup}{\charfusion[\mathop]{\bigcup}{\cdot}}
\newcommand{\ceil}[1]{\left\lceil{#1}\right\rceil}
\newcommand{\floor}[1]{\left\lfloor{#1}\right\rfloor}
\newcommand{\mc}[1]{\mathcal{#1}}
\newcommand{\limes}[2]{\lim_{#1 \rightarrow #2}}
\newcommand{\limessup}[1]{\limsup_{#1 \rightarrow \infty}}
\newcommand{\limesinf}[1]{\liminf_{#1 \rightarrow \infty}}
\newcommand{\vect}[1]{\begin{pmatrix}#1\end{pmatrix}}
\newcommand{\partd}[2]{\frac{\partial #1}{\partial #2}}
\newcommand{\op}[1]{\left\|#1\right\|_{\text{op}}}

\makeatletter
\renewcommand*\env@matrix[1][*\c@MaxMatrixCols c]{%
  \hskip -\arraycolsep
  \let\@ifnextchar\new@ifnextchar
  \array{#1}}
\makeatother

\setromanfont[Mapping=tex-text]{Linux Libertine O}
% \setsansfont[Mapping=tex-text]{DejaVu Sans}
% \setmonofont[Mapping=tex-text]{DejaVu Sans Mono}
\parindent 0pt

\title{\sc Analysis III \\ \Large 8. Aufgabenblatt}
\author{Jendrik Stelzner}
\date{\today}

\begin{document}
\maketitle





\section{Vertauschung von Integralen}

\subsection{}
Für alle $x \in \R$ ist
\[
 \int_\R f(x,y) \dmu(y)
 = \int_\R \delta_{x,y} \dmu(y)
 = \mu(\{x\}) \cdot 1
 = 1,
\]
also
\[
 \int_\R \int_\R f(x,y) \dmu(y) \dlambda(x)
 = \int_\R 1 \dlambda
 = \infty.
\]
Andererseits ist für alle $y \in \R$
\[
 \int_\R f(x,y) \lambda(x)
 = \int_\R \delta_{x,y} \dlambda(x)
 = \lambda(\{y\}) \cdot 1
 = 0,
\]
also
\[
 \int_\R \int_\R f(x,y) \dlambda(x) \dmu(y)
 = \int_\R 0 \dmu
 = 0.
\]
Insbesondere sind die beiden Doppelintegrale verschieden.

\subsection{}
Es ist für alle $1 \in (0,1)$
\begin{equation*}
 \int_{(0,1)} g(x,y) \dlambda(y)
 = \int_0^1 \frac{x^2-y^2}{(x^2+y^2)^2} \dy,
\end{equation*}
da $g(x,y)$ für festes $x \in (0,1)$ auf $[0,1]$ stetig ist, also Lebesgue- und Riemannintegral übereinstimmen. Für obiges Integral ergibt sich
\begin{equation}\label{eq: integral auseinender}
 \begin{aligned}
  \int_0^1 \frac{x^2-y^2}{(x^2+y^2)^2} \dy
  &= \int_0^1 \frac{x^2+y^2-2y^2}{(x^2+y^2)^2} \dy \\
  &= \int_0^1 \frac{1}{x^2+y^2} \dy - \int_0^1 \frac{2y^2}{(x^2+y^2)^2} \dy.
 \end{aligned}
\end{equation}
Das Auseinanderziehen der Integrale ist möglich, da beide existieren, da die entsprechenden Funktionen auf $[0,1]$ stets stetig und beschränkt sind. Durch partielle Integration ergibt sich, dass
\begin{align*}
  \int_0^1 \frac{2y^2}{(x^2+y^2)^2} \dy
  &= \int_0^1 y \cdot \frac{2y}{(x^2+y^2)^2} \dy
  = \left. -y \frac{1}{x^2+y^2} \right|_{y=0}^1 + \int_0^1 \frac{1}{x^2+y^2} \\
  &= -\frac{1}{1+x^2} + \int_0^1 \frac{1}{x^2+y^2}.
\end{align*}
Zusammen mit \eqref{eq: integral auseinender} ergibt sich damit, dass
\[
 \int_{(0,1)} g(x,y) \dlambda(y)
 = \frac{1}{1+x^2}
\]
für alle $x \in (0,1)$.
Da $\frac{1}{1+x^2}$ auf $[0,1]$ stetig und beschränkt, und damit Riemann-integrierbar ist, gilt
\begin{align*}
 \int_{(0,1)} \int_{(0,1)} g(x,y) \dlambda(y) \dlambda(x)
 &= \int_{(0,1)} \frac{1}{1+x^2} \dlambda(x)
 = \int_0^1 \frac{1}{1+x^2} \dx \\
 &= \left. \arctan(x) \right|_{x=0}^1
 = \arctan(1) - \arctan(0)
 = \frac{\pi}{4}.
\end{align*}

Für das andere Doppelintegral ergibt sich, da $g(x,y) = -g(y,x)$ für alle $x,y \in (0,1)$, aus dem gerade berechneten Integral, dass
\[
 \int_{(0,1)} \int_{(0,1)} g(x,y) \dlambda(x) \dlambda(y)
 = -\int_{(0,1)} \int_{(0,1)} g(y,x) \dlambda(x) \dlambda(y)
 = -\frac{\pi}{4}.
\]
Insbesondere sind die beiden Integrale verschieden.

Um $\int_{(0,1)^2} |g| \dlambda_2$ zu bestimmen, bemerken wir, dass $|g| \geq 0$ auf $(0,1)^2$ stetig, also  Borell-messbar ist. Wäre $\int_{(0,1)^2} |g| \dlambda_2 < \infty$, so wäre $|g|$ und damit auch $g$ auf $(0,1)^2$ bezüglich $\lambda_2$ integrierbar. Da $(0,1)^2 = (0,1) \times (0,1)$ und $\lambda_2 = \lambda \times \lambda$ wären dann die beiden oben berechneten Integrale nach dem Satz von Fubini gleich. Da dies nicht der Fall ist, muss $\int_{(0,1)^2} |g| \dlambda_2 = \infty$.

Dies lässt sich auch nachrechnen: Es lässt sich auf $\int_{(0,1)^2} |g| \dlambda_2$ der Satz von Tonelli anwenden, weshalb
\[
 \int_{(0,1)^2} |g| \dlambda_2
 = \int_{(0,1)} \int_{(0,1)} |g(x,y)| \dlambda(y) \dlambda(x).
\]
Wir bemerken, dass, da $|g|$ stetig auf $[0,1] \setminus \{(0,0)\}$ ist, für alle $x \in (0,1)$
\[
 \int_{(0,1)} |g(x,y)| \dlambda(y)
 = \int_0^1 \frac{|x^2-y^2|}{(x^2+y^2)^2} \dy
 \geq \int_0^x \frac{x^2-y^2}{(x^2+y^2)^2} \dy.
\]
Analog zu den vorherigen Berechnungen ergibt sich, dass
\[
 \int_0^x \frac{x^2-y^2}{(x^2+y^2)^2} \dy
 = \left. y\frac{1}{x^2+y^2} \right|_{y=0}^x
 = \frac{1}{2x}.
\]
Da $\frac{1}{2x}$ auf $(0,1)$ stetig ist, und auf jedem kompakten Teilintervall von $(0,1)$ Riemann-integrierbar, ergibt sich damit zusammengefasst
\[
 \int_{(0,1)^2} |g| \dlambda_2
 \geq \int_{(0,1)} \frac{1}{2x} \dlambda
 = \int_0^1 \frac{1}{2x} \dx
 = \infty.
\]





\section{Fast überall differenzierbar}


\subsection{}
Es sei
\[
 N = \{x \in [0,1] : f \text{ ist nicht differenzierbar an } x\}.
\]
Da $f$ fast überall differenzierbar ist, ist $\lambda(N) = 0$ und $f$ auf $N^C \subseteq (0,1)$ differenzierbar. Wir zeigen zunächst, dass $g_k(x) \rightarrow f'(x)$ für $k \rightarrow \infty$ punktweise auf $N^C$. Hierfür unterscheiden wir zwischen zwei Fällen:

Ist $x = j/2^n$ für ein $n \in \N$ und $j \in \{0,1,\ldots,2^n-1\}$, so ist
\[
 g_k(x) = \frac{f(x+2^{-k})-f(x)}{2^{-k}} \text{ für alle } k \geq n,
\]
und daher, da $f$ an der Stelle $x$ differenzierbar ist,
\[
 \limes{k}{\infty} g_k(x)
 = \limes{k}{\infty} \frac{f(x+2^{-k})-f(x)}{2^{-k}}
 = f'(x).
\]

Andernfalls bezeichne für alle $k \in \N$ das Paar $x_k^L < x < x_k^R$ mit $x_k^R - x_k^L = 2^{-k}$ die Randpunkte des Intervalls, auf dem $g_k$ als konstant definiert ist. Es ergibt sich, dass für alle $k \in \N$
\begin{align*}
 g_k(x)
 &= \frac{f\left(x_k^R\right)-f\left(x_k^L\right)}{2^{-k}}
 = \frac{f\left(x_k^R\right)-f(x)}{2^{-k}} + \frac{f(x)-f\left(x_k^L\right)}{2^{-k}} \\
 &= \frac{f\left(x_k^R\right)-f(x)}{x_k^R-x}\frac{x_k^R-x}{2^{-k}}
    + \frac{f(x)-f\left(x_k^L\right)}{x-x_k^L}\frac{x-x_k^L}{2^{-k}},
\end{align*}
sowie
\begin{align*}
 f'(x)
 &= 1 \cdot f'(x)
 = \left(\frac{x_k^R-x}{2^{-k}}+\frac{x-x_k^L}{2^{-k}}\right)f'(x) \\
 &= \frac{x_k^R-x}{2^{-k}}f'(x) + \frac{x-x_k^L}{2^{-k}}f'(x),
\end{align*}
also für alle $k \in \N$
\begin{align*}
  &\, g_k(x)-f'(x) \\
 =&\, \left(\frac{f\left(x_k^R\right)-f(x)}{x_k^R-x}-f'(x)\right)\frac{x_k^R-x}{2^{-k}}
    + \left(\frac{f(x)-f\left(x_k^L\right)}{x-x_k^L}-f'(x)\right)\frac{x-x_k^L}{2^{-k}}.
\end{align*}
Durch die Dreiecksungleichung ergibt sich für alle $k \in \N$
\begin{equation}\label{eq: abschätzung lang}
 \begin{aligned}
      &\, |g_k(x)-f'(x)| \\
  \leq&\, \left|\frac{f\left(x_k^R\right)-f(x)}{x_k^R-x}-f'(x)\right|\frac{x_k^R-x}{2^{-k}}
          + \left|\frac{f(x)-f\left(x_k^L\right)}{x-x_k^L}-f'(x)\right|\frac{x-x_k^L}{2^{-k}}.
 \end{aligned}
\end{equation}
Sei nun $\varepsilon > 0$ beliebig aber fest. Da $\limes{k}{\infty} x_k^L = x = \limes{k}{\infty} x_k^R$ und $f$ an der Stelle $x$ differenzierbar ist, gibt es $n_L, n_R \in \N$ mit
\begin{align*}
 \left|\frac{f\left(x_k^R\right)-f(x)}{x_k^R-x}-f'(x)\right| &< \varepsilon \text{ für alle } k \geq n_L \text{ und} \\
 \left|\frac{f(x)-f\left(x_k^L\right)}{x-x_k^L}-f'(x)\right| &< \varepsilon \text{ für alle } k \geq n_R.
\end{align*}
Es ergibt sich damit aus \eqref{eq: abschätzung lang}, dass für alle $k \geq \max \{n_L, n_R\}$
\begin{align*}
 |g_k(x) - f'(x)|
 < \varepsilon \frac{x_k^R-x}{2^{-k}} + \varepsilon \frac{x-x_k^L}{2^{-k}}
 = \varepsilon \left(\frac{x_k^R-x}{2^{-k}}+\frac{x-x_k^L}{2^{-k}}\right)
 = \varepsilon.
\end{align*}
Wegen der Beliebigkeit von $\varepsilon > 0$ zeigt dies, dass $g_k(x) \rightarrow f'(x)$ für $k \rightarrow \infty$.

Da $f' : [0,1] \rightarrow \R$ nach Definition Lebesgue-messbar ist, und $g = f'$ auf $N^C$, folgt, da $N$ eine Lebesgue-Nullmenge und das Lebesgue-Maß vollständig ist, dass auch $g : [0,1) \rightarrow \R$ Lebesgue-messbar ist. (Vergleiche Satz 2.10.) Da $f$ streng monoton steigend ist, ist $g = f' \geq 0$ auf $N^c$. Das Integral $\int_{(0,1)} f' \dlambda = \int_{N^c} f' \dlambda$ ist daher wohldefiniert. Aus der Definition von $g_k$ ergibt sich auch direkt, dass für alle $k \in \N$.
\[
 \int_{N^c} g_k(x) \dlambda
 = \sum_{j=0}^{2^k-1} f\left((j+1)2^{-k}\right)-f\left(j2^{-k}\right) 
 = f(1)-f(0).
\]
Nach dem Lemma von Fatou ist daher
\[
 \int_{(0,1)} f' \dlambda
 = \int_{N^c} f' \dlambda
 = \int_{N^c} \limesinf{k} g_k \dlambda
 \leq \limesinf{k} \int_{N^c} g_k \dlambda
 = f(1)-f(0).
\]
Insbesondere ist $\int_{(0,1)} f' \dlambda < \infty$, also $f'$ über $(0,1)$ integrierbar bezüglich $\lambda$.


\subsection{}
Man betrachte die Kantormenge $C \subseteq [0,1]$ und die Kantorfunktion $f: [0,1] \rightarrow [0,1]$. Es ist bekannt, dass $f$ stetig und monoton steigend ist. $f$ ist auch $\lambda$-fast überall differenzierbar: Für alle $x \in [0,1] \setminus C$ gibt es ein $\varepsilon > 0$ mit $B_\varepsilon(x) \subseteq [0,1] \setminus C$. $f$ ist auf nach $B_\varepsilon(x)$ konstant, weshalb $f$ an $x$ differenzierbar mit $f'(x) = 0$ ist. Mit $\lambda(K) = 0$ ergibt sich die Behauptung.

Es ist insbesondere $\int_{(0,1)} f'(x) \dlambda = 0$, aber $f(1) - f(0) = 1 - 0 = 1$.








































\end{document}
